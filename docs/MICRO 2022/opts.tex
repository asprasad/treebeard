%\section{Optimizations}
\section{Optimizations on High-level IR}
\label{Sec:HIR}
This section describes tree tiling and tree reordering, two optimizations performed at the highest 
level of abstraction. Recall that at this level the \op{predictForest} operator is abstractly
represented by a set of decision trees. 

%\TODO{This section needs a better name!}
\subsection{Notation}
%TODO{Notation needs to be introduced in the background section, Tiling should be the first subsection}
We represent a decision tree by $\Tree = (V, E, r)$ where $V$ is the set of nodes, $E$ the set of edges and
$r \in V$ is the root. For each node $n \in V$, we define the following.
\begin{enumerate}
    \item $threshold(n) \in \mathbb{R}$, the threshold value for $n$.
    \item $featureIndex(n) \in \mathbb{N}$, the feature index for $n$.
    \item $left(n) \in V$, the left child of $n$ or $\emptyset$ if $n$ is a leaf. If $left(n) \neq \emptyset$, then $(n, left(n)) \in E$.
    \item $right(n) \in V$, the right child of $n$ or $\emptyset$ if $n$ is a leaf. If $right(n) \neq \emptyset$, then $(n, right(n)) \in E$.
\end{enumerate}
We use $L_{\Tree} \subseteq V$ to denote the set of leaves. % subseteq because tree could have single node

\subsection{Tiling}
\label{sec:Tiling}
% Treebeard vectorizes tree walks by grouping nodes of a decision tree into \textbf{\emph{tiles}}. The nodes in a tile are evaluated concurrently using vector instructions. Once the nodes of the current tile are evaluated, a look up table is used to compute which child of the current tile to move to next.

While a decision tree is naturally represented by a binary tree, 
it is not the best representation for tree traversal as it (i) requires many memory accesses, 
(ii) has poor branching structure and (iii) cannot make use of vector instructions. 
This section proposes a tiling optimization where we group multiple nodes in a decision tree into a single tile, effectively transforming a binary tree into an $n$-ary tiled tree. This not only allows the compiler to generate vectorized code (see Section~\ref{sec:Vectorization} ) to traverse trees but also enables spatial locality improvements by grouping nodes that are likely to be accessed together. We demonstrate this with two tiling heuristics later in the section. 



\CommentOut{
Treebeard groups nodes of the decision tree into \textbf{\emph{tiles}}. Tiling provides two benefits. 
\begin{enumerate}
  \item It allows the compiler to generate vector code to traverse trees. Section \ref{sec:Vectorization} describes how Treebeard does this.
  \item It enables spatial locality improvements by grouping together nodes that are likely to be accessed together. 
\end{enumerate}
Once nodes are grouped into tiles, an $n$-ary tree whose nodes are tiles is constructed. Treebeard then generates 
optimized code to walk this tree. The listing below shows at a high level how a tiled tree is walked (This is not 
true IR, but presented for clarity). 
}

Once trees are tiled Treebeard generates tree walks with the code structure shown below.
\begin{lstlisting}[style=c++]
  ResultType Prediction_Function(...) {
    // ...
    Tile t = getRootTile(tree)
    while (!isLeaf(tree, t)) do {
      // Evaluate predicates of all nodes in the tile
      predicates = evaluateTilePredicates(t, rows[i])
      
      // Move to the correct child of the current tile
      t = getChildTile(tree, t, predicates) 
    }
    treePrediction = getLeafValue(t)
    // ...
  }  
\end{lstlisting}
The code is just an abstract representation of a tiled tree walk that enables efficient lowering of specific steps in subsequent stages. 
\op{evaluateTilePredicates} (speculatively) computes the predicates on all nodes in a tile (line 6). Then \op{getChildTile} (line 9), uses the computed predicate values to determine which child of the current tile to move to. We defer a description of how these operators are lowered to a later section but focus on tiling algorithms in this section.

\CommentOut{
To compute the prediction of the tree, the predicates of all nodes in the tile are computed simultaneously (line 6). 
Then, the computed predicate values are used to determine which child of the current tile to move to (in the 
$n$-ary tree). This section presents the details of tiling and how Treebeard's general tiling infrastructure 
can be used to develop tiling algorithms with different objectives (sections \ref{sec:UnifTiling}) and \ref{sec:ProbTiling}).
The details of how Treebeard lowers predicate evaluation and moving to the correct child to use vector instructions 
are described in section \ref{sec:Vectorization}.

\subsection{Tiles and Tree Tiling}
\label{sec:ValidTiling}
}
\subsubsection*{Conditions for Efficient Tiling}
\label{sec:ValidTiling}
While any arbitrary partitioning of the nodes of a tree could be considered for tiling we impose a few intuitive constraints.
% to only allow nodes that are likely to be accessed together to be grouped into a tile. 
% \TODO{kr : replace $n_t$ with sz?}
Given a tree $\Tree = (V, E, r)$ and a tile size $n_t$ we impose the following constraints on the generated tiles $\{ T_1, T_2, ... ,T_m \}$ .
\begin{description}
    \item[Partitioning] $T_1 \cup T_2 ... \cup T_m = V$ and $T_i \cap T_j = \emptyset$ for all $i\neq j$
    \item[Connectedness] If $u, v \in T_i$, there is a (undirected) path connecting $u$ and $v$ fully contained in $T_i$.
   \item [Leaf seperation] $\forall l \in L_{\Tree}$ : $l \in T_i \rightarrow v \notin T_i \;\; \forall v \in V \backslash \{l\}$
  \item [Maximal tiling] if there are tiles such that. $|T_i| < n_t$ then there is no $v \in V\backslash \{ T_i \cup L_{\Tree} \}$ such that $(u, v) \in E$ for some $u \in T_i$. 
\end{description}
The \textbf{partitioning} and \textbf{maximal tiling} constraints together ensure that we group nodes into as few tiles as possible. {\textbf{Connectedness}} ensures that each tile is a sub-tree, a natural grouping of nodes that are likely to be accessed together. The {\textbf{Leaf separation}} constraint ensures that leafs are not tiled along with internal nodes. Leafs in a decision tree need special handling, they are used to check for walk termination and to determine the output (prediction). This constraint ensures that tiles are homogenous, this in-turn allows us to specialize the in-memory layout of trees and also simplifies code generation. We discuss leaf handling and tree layout in section ~\ref{Sec:MemoryRep}. We refer to any tiling that satisfies the above constraints as a \emph{valid} tilling.

%%% COMMENT %%%% 
\CommentOut{
\subsection{Tiled Trees}
A tiling transformation communicates the tiling to the Treebeard infrastructure by assigning a tile ID to each node in the decision tree. Using these tile IDs, Treebeard checks the validity of the tiling and then contructs a tree whose nodes are tiles. We call this tree the \textbf{\emph{tree of tiles}}. \TODO{We need a better name for this}
Figure \ref{Fig:ValidTilingTileSize3} shows a valid tiling with tile size 3 and the tree of tiles constructed by Treebeard. Three nodes are grouped into each of the tiles $t_1$ and $t_2$ as shown. Each tile is collapsed into a single node in the tree of tiles. However, each leaf in the original tree becomes a leaf in the tree of tiles.

\begin{figure}
  \centering
  \includegraphics[width=\linewidth]{figures/TiledTree_Size3.PNG}
  \caption{Example of a valid tree tiling with tile size $n_t=3$}
  \label{Fig:ValidTilingTileSize3}
\end{figure}

Treebeard maintains the following invariants.
\begin{enumerate}
  \item All tiles in a tree are the same size $n_t$. If the tiling produces any smaller tiles, these are padded by inserting dummy nodes to make them the required size.
  \item Nodes within tiles are always ordered in level order and left to right within a level. The numbering of the nodes in the above diagram shows this node order.
  \item Children of a node are numbered from left to right (regardless of level). For example, $l_1$ is the first child of $t_1$, $l_2$ is the second and so on.
\end{enumerate}
}
%%% END COMMENT %%%%

\subsection{Basic Tiling}
\label{sec:UnifTiling}
% The justification for basic tiling -- if we assume all paths are equally likely,
% we want to minimize the depth of speculation
Algorithm~\ref{Alg:UnifTilingAlgo} shows the basic tiling algorithm that produces a valid tiling. It 
attempts to minimize the depths of all constructed tiles. 
Tiling starts at the root and constructs a tile $Tile$ by performing
a level order traversal. The call \op{LevelOrderTraversal($\Tree$, $n_t$)} picks the next $n_t$ non-leaf nodes according to the standard 
level order tree traversal algorithm. Once the current tile is constructed, 
the tiling procedure is recursively performed on subtrees rooted at each node that is a destination of an edge going out of the constructed 
tile. It is easy to see that the tiling constructed by Algorithm \ref{Alg:UnifTilingAlgo} is valid.
% \TODO{Pass a modified tree without leaves to the level order traversal call. Explicitly add leaves as seperate tiles}
% Algorithm for basic tiling
\begin{algorithm}
  \caption{Basic tree tiling}
  \label{Alg:UnifTilingAlgo}
  \small{
  \begin{algorithmic}[1]
      \Procedure{LevelOrderTraversal}{$\Tree = (V, E, r)$, $n_t$}
        \State $queue = \{ r \}$
        \State $Tile = \emptyset$
        \While{$\neg queue.empty() \wedge |Tile|<n_t$}
          \State $n = queue.dequeue()$
          \If {$n \in L_{\Tree}$}
              \State \textbf{continue}
          \EndIf
          \State $Tile = Tile \cup \{ n \}$
          \State $queue.enqueue([left(n), right(n)])$
        \EndWhile
      \EndProcedure
      \State {}
      \Procedure{TileTree}{$\Tree = (V, E, r)$, $n_t$} 
          \If {$r \in L_{\Tree}$}
              \State \textbf{return} $\{ r \}$
          \EndIf
          % \State \textcolor{codegreen}{\textit{//Level order traversal to collect $n_t$ or fewer nodes. }}
          % \State \textcolor{codegreen}{\textit{//Leaves are not included in the constructed tile. }}
          \State $Tile \leftarrow LevelOrderTraveral(\Tree, n_t)$
          \State $Tiles =  \{ Tile \}$
          \For{$(u,v) \in Out(Tile)$}
              \State $Tiles \leftarrow Tiles \cup TileTree(S_v, n_t)$
          \EndFor
          \State \textbf{return} $Tiles$
      \EndProcedure
  \end{algorithmic}
  }
\end{algorithm}

One interesting property of this tiling algorithm is that it naturally reduces 
the imbalance in trees, especially at large tile sizes. As the algorithm 
traverses down to sparser levels of the tree, it naturally groups sub-trees 
containing chains of nodes, thus balancing the trees. While it is possible to 
further enhance the algorithm to explicitly balance tiled trees, we find that 
basic tiling suffices in practice.

%%%% COMMENT %%%%% 
\CommentOut{
\subsubsection{Further Opimization and Code Generation}
We found that most leaf tiles for a given tree are at the same depth when basic tiling is used. Furthermore, we see that deeper leaves 
are more likely to be reached.
%\footnote{Intuitively, this is true because training algorithms keep splitting nodes to maximize gain and gain
%will typically be maximized by splitting a large number of inputs.}.
Based on these observations, we (optionally) pad the tree of tiles generated with basic tiling so that all leaves are at the same depth.
This transformation is performed on the high level IR after basic tiling. 
Once the trees have been padded to make all leaves equal depth, the tree walks are fully unrolled to evaluate a fixed 
number of tiles and all leaf checks are omitted.

One other complication the code generator needs to handle is the fact that different trees in the model being 
compiled potentially have different depths. In order to handle
this, Treebeard sorts the trees by their depth. This ensures that all trees with equal depth are grouped together. Once this is done, 
the loop over the trees is fissed so that each of the resulting loops only walks trees of a single depth. Consider for example a 
forest with 4 trees $T_1$, $T_2$, $T_3$, and $T_4$ in that order. Further, assume that $T_1$ and $T_4$ have depth 2 while $T_2$ and $T_3$
have depth 3. First, Treebeard reorders the trees to be in the order $T_1$, $T_4$, $T_2$, $T_3$. Then, the loop over the trees is fissed
as shown in the following listing.

% loop transformations for basic tiling (splitting) 
\begin{lstlisting}{style=c++}
  forest = ensemble(...)
  for i = 0 to batchSize step 1 {
    prediction = 0
    for t = 0 to 2 step 1 {
      tree = getTree(forest, t) 
      node = getRoot(tree)
      node = traverseTreeTile(tree, node, rows[i])
      treePrediction = getLeafValue(tree, node)
      prediction = prediction + treePrediction
    }
    for t = 2 to 4 step 1 {
      tree = getTree(forest, t) 
      node = getRoot(tree)
      node = traverseTreeTile(tree, node, rows[i])
      node = traverseTreeTile(tree, node, rows[i])
      treePrediction = getLeafValue(tree, node)
      prediction = prediction + treePrediction
    }
    predictions[i] = prediction
  }  
\end{lstlisting}

\TODO{AP: This listing is unnecesarily long. Can we maybe leave out the loop bodies and say something like "depth 2 walk"? Should 
we point to the figure in the overview section instead?}
}
%%%% COMMENT END %%%%% 

\subsection{Probability Based Tiling}
\label{sec:ProbTiling}
%\subsubsection*{Motivation}

The next algorithm we propose exploits the inherent biases among the leaves of a decision tree. In typical machine learning models some leafs (equivalently outcomes or predictions) are more likely to be reached than others. In such settings, having balanced tiled trees is not sufficient to minimize expected inference time. 

Consider for example two machine learning models \op{airline-ohe} and \op{epsilon} (also used in our evaluation). 
Consider the graphs shown in   
figures \ref{Fig:AirlineOHEStats} and \ref{Fig:EpsilonStats} that are generated from training data. Each line in these graphs corresponds to a fixed fraction of the input (say $f$). 
A point on a line at coordinate $(x, y)$ means that a fraction $y$ of trees in the model could cover a fraction $f$ of all training inputs with a fraction $x$ of 
leaves. For example, the first point on the $f=0.9$ line in figure \ref{Fig:AirlineOHEStats} says that about 52\% of trees ($y$ value) need only 1\% of their
leaves ($x$ value) to cover 90\% of the training input. 
In general, Figure \ref{Fig:AirlineOHEStats} shows that very few leaves are needed to cover a very large fraction of inputs for the benchmark \op{airline-ohe}. 
This means that a small fraction of leaves are very likely. 
We call trees with a small number of extremely likely leaves \textbf{\emph{leaf biased}}.

On the other hand, for the benchmark \op{epsilon},
figure \ref{Fig:EpsilonStats} shows that a trees need a much larger fraction of their leaves to cover a significant fraction of the training input.
This means that most trees in \op{epsilon} are not leaf biased.

\CommentOut{
\subsubsection{More Notation}
In order to formulate the probability based tiling algorithm as an optimization problem, we define the following.
\begin{enumerate}
    \item For every leaf $l \in L$, we define $p_l$ as the probability that the leaf $l$ is reached.
    \item For each node $n \in V$, we define the absolute probability $p_v$ as
    \begin{equation}
        p_v = \begin{cases}
        p_l &\text{if $l \in L$}\\
        p_{left(v)} + p_{right(v)} &\text{otherwise}
        \end{cases}
    \end{equation}
    \item For any tree $T$, $\mathcal{C}(T)$ represents the set of all valid tilings of $T$.
    \item For every $v \in V$, we define $S_v$ as the subtree rooted at $v$.
    \item For every $v \in V$, we define $L_v$ as the set of leaves of $S_v$.
    \item For a every tile $T_i$, we define $root(T_i)$ as the node $v \in T_i$ such that $v$ has no incoming edges from any other node $u \in T_i$.
    \item For a tile $T_i$, $out(T_i) \subseteq E$ is the set of edges $(u, v)$ such that $u \in T_i$ and $v \notin T_i$.
\end{enumerate}
}

\begin{figure}
    \centering
    \includegraphics[width=\linewidth]{figures/airline-ohe.stats.train.txt.png}
    \caption{Statistical profile for airline-ohe}
    \label{Fig:AirlineOHEStats}
\end{figure}
\begin{figure}
    \centering
    \includegraphics[width=\linewidth]{figures/epsilon.stats.train.txt.png}
    \caption{Statistical profile for epsilon}
    \label{Fig:EpsilonStats}
\end{figure}
\CommentOut{
We say that an input row $r_i$ is \textbf{\emph{covered}} by a subset of leaves $L' \subseteq L$ of a tree $T$, if the leaf $l$ reached by 
walking $T$ for row $r_i$ is in $L'$. We show how different models (and even different trees within the same model) behave differently 
using models for two benchmarks, airline-ohe and epsilon. Consider the graphs shown in   
figures \ref{Fig:AirlineOHEStats} and \ref{Fig:EpsilonStats}. Each line in these graphs corresponds to a fixed fraction of the input (say $f$). 
A point on a line at coordinate $(x, y)$ means that a fraction $y$ of trees in the model could cover a fraction $f$ of all training inputs with a fraction $x$ of 
leaves. For example, the first point on the $f=0.9$ line in figure \ref{Fig:AirlineOHEStats} says that about 52\% of trees ($y$ value) need only 1\% of their
leaves ($x$ value) to cover 90\% of the training input. 
In general, Figure \ref{Fig:AirlineOHEStats} shows that very few leaves are needed to cover a very large fraction of inputs for the benchmark airline-ohe. 
This means that a small fraction of leaves are very likely. On the other hand, for the benchmark epsilon,
figure \ref{Fig:EpsilonStats} shows that a trees need a much larger fraction of their leaves to cover a significant fraction of the test input.
This means that most trees in the epsilon model are not leaf biased.
One other observation we make is that most models have some leaf biased trees while the rest of the trees have equally likely leaves.
\TODO{AP Maybe define a term for trees with roughly equally likely leaves?} We design the probability based tiling algorithm to take advantage of this property 
of decision tree ensembles. 
}
\subsubsection{The Optimization Problem}

%We assume that we are given the probabilities of each leaf node of the decision tree (these can easily be computed using the training data). For every leaf $l \in L$, we %are given the probability $p_l$ that the leaf $l$ is reached. 

Observe that the latency of one tree walk is proportional to the number of tiles that need to be evaluated to reach the leaf. It is easy to see that for a leaf biased tree, basic tilling does not optimize for this objective, it considers all leafs to be equally likely. 
 
The goal of probablistic tiling is to minimize the average inference latency, or equivalently the minimize the expected number of tiles that are evaluated to compute one tree prediction. More formally, the problem is to find a \emph{valid} (as defined in Section~\ref{sec:ValidTiling}) tiling $\mathcal{T}$ such that the following objective is minimized.
\[
    \min_{\mathcal{T} \in \mathcal{C}(T)}{\sum_{l \in L_{\Tree}} p_l.depth_{\mathcal{T}}(l)}
\]
where the minimization is over all valid tilings $\mathcal{T}$ of the tree $\Tree$, $depth_{\mathcal{T}}(l)$ is the depth of the leaf $l$ given tiling ${\mathcal{T}}$. $p_l$ is the probability of of reaching leaf $l$ as observed during training.

The above optimization problem can be solved optimally using dynamic programming. 
We leave this out in the interest of space. 
Instead, we use the simple greedy algorithm listed in algorithm \ref{Alg:GreedyTilingAlgo} to construct a valid tiling given the node probabilities\footnote{Probabilites for internal nodes can be computed from probablities for leafs by summing up the probabilities of all leafs that belong to the sub-tree rooted at the internal node. Leaf probabilities are collected during training.}.
The algorithm starts at the root and greedily keeps adding the most probable legal node to the current tile until the maximum tile size is reached.
Subsequently, the tiling procedure is recursively performed on all nodes that are destinations for edges going out of the constructed tile.

% \subsubsection{Dynamic Programming Formulation}


% For any node $v \in V$, we define
% \[
%     cost(v, \mathcal{T}) = \sum_{l \in L_v} p(l | v).depth_{\mathcal{T}}(l)
% \]
% where $\mathcal{T} \in \mathcal{C}(T_v)$.

% Then, the objective function, for the tree $T_v$, can be rewritten as 
% \[
%     opt\_cost(v) = \min_{\mathcal{T} \in \mathcal{C}(T_v)}{cost(v, \mathcal{T})}
% \]

% The objective function can then be rewritten in the following recursive form.
% \[
%     opt\_cost(v) = \min_{T_0 \in TileShapes(n_t, v)}{1 + \sum_{(n_1, n_2) \in out(T_0)} p(n_1 | v)p(n_2 | n1)opt\_cost(n_2)}
% \]
% where $TileShapes(n_t, v)$ is the set of all tile shapes of size $n_t$ with root $v$. A straight forward substitution argument shows why the solution to the subproblems (tiling all sub-trees) needs to to be optimal. The objective is now in a form that can solved using
% dynamic programming. 

% \subsubsection{Greedy Algorithm}

% Intuitively, it seems like the following greedy algorithm also gives the optimal tiling. The algorithm starts at the root and greedily keeps adding the most probable node to the current tile until the maximum tile size is reached.
\begin{algorithm}
    \caption{Greedy Probability Based Tree Tiling}
    \label{Alg:GreedyTilingAlgo}
    \begin{algorithmic}
        \Procedure{TileTree}{$\Tree = (V, E, r)$, $n_t$} 
            \If {$r \in L_{\Tree}$}
                \State \textbf{return} $\{ r \}$
            \EndIf
            \State $Tile \leftarrow \{ r \}$
            \While{$|Tile| < n_t$}
                \State $e = (u,v) \in Out(Tile)$ st $p(v)$ is max and $v \notin L$
                \If{$e = \emptyset$}
                    \State \textbf{break}
                \EndIf
                \State $Tile = Tile \cup \{ v \}$
            \EndWhile
            \State $Tiles =  \{ Tile \}$
            \For{$(u,v) \in Out(Tile)$}
                \State $Tiles \leftarrow Tiles \cup TileTree(S_v, n_t)$
            \EndFor
            \State \textbf{return} $Tiles$
        \EndProcedure
    \end{algorithmic}
\end{algorithm}

% Talk about problems with increasing number of tile shapes and only performing such tiling on skewed trees
\CommentOut{
When we tried to apply algorithm \ref{Alg:GreedyTilingAlgo} on all trees in our benchmarks, we found
that even minor variations in probability caused the tiling algorithm to generate a large 
number of tile shapes. This in turn caused a loss in performance because the large size of the 
lookup table needed (section \ref{sec:LookupTable}) caused increased L1 cache misses. In order to 
alleviate this, we only perform probability based tiling on trees that are leaf biased.
}
We find probability based tiling is only beneficial for leaf biased trees\footnote{Turns out that for trees that are not leaf biased, probability based tiling produces many more tile shapes (see Section~\ref{sec:tileShapes}) which direclty impacts the cost of \op{getChildTile} making it more expensive than basic tiling.}.  Recall that   
a tree to be leaf biased if a small fraction of leaves, say $\alpha$, can cover a large fraction of training inputs, say $\beta$.
We only perform probability based tiling on trees with thresholds $\alpha=0.05$ and $\beta=0.9$ and fall back to uniform tiling otherwise. 





\subsection{A Note on Implementation}
\TODO{Kr : Not sure if this is needed}
The tiling algorithms generate a \op{TileId} attribute per tree. The \op{TileId} attribute contains a mapping from a Node to the TileId asigned to it.
This information is used when lowering to the mid level abstraction in the form of loops. A sample of the lowered MIR code is shown in Figure~\ref{Fig:Overview}.	
\subsection{Loop Rewriting}
\label{Sec:LoopReordering}
\Treebeard{} supports a wide range of loop rewrites in the process of 
lowering from the high level IR to the mid level IR. \Treebeard{} 
supports fissing and permuting of the loop nest 
over the tree, input row pairs. Figure~\ref{Fig:Overview} shows the specific
case of permuting loops. Two versions of MIR generated for two different 
loop orders are shown -- ``one tree at a time'' that walks one tree for 
all input rows before moving to the next tree and ``one row at a time'' that
walks all trees for an input row before moving to the next row. The 
structure of the loop nest to be generated in MIR is decided at the HIR level and 
communicated to the lowering pass through attributes (as mention in 
section~\ref{Sec:Overview}).
\subsection{Tree ordering}
\label{sec:treeorder}	
%Another objective at the HIR is to reduce the number of trees that need to be specialized by the lower levels.   
\TODO{kr : Not sure if i got this para right}
Specializing the code for each tree in a model comes at a cost. First the code generator needs to generate different code for different trees potentially increasing the size of the generated code. Second some cross tree optimizations (applied at the lower levels of abstraction) like tree walk interleaving require that the multiple trees share identical code. 

In order to handle
this, \Treebeard{} pads trees with dummy nodes to make them balanced and then sorts the trees by their depth, so that trees of same depth can share code. Padding is only done for almost balanced trees (as generated by basic tiling), this is ensured by only adding up to a fixed fraction of dummy nodes.  
\TODO{Kr : Can we add a threshold say 10\%?}
%This ensures that all trees with equal depth are grouped together.
 Once this is done, 
the loop over the trees is fissed so that each of the resulting loops only walks trees of a single depth. Consider for example a 
forest with 4 trees $T_1$, $T_2$, $T_3$, and $T_4$ in that order. Further, assume that $T_1$ and $T_4$ have depth 2 while $T_2$ and $T_3$
have depth 3. First, Treebeard reorders the trees to be in the order $T_1$, $T_4$, $T_2$, $T_3$. Then, the loop over the trees is fissed
as shown in the following listing.

% loop transformations for basic tiling (splitting) 
\begin{lstlisting}{style=c++}
  forest = ensemble(...)
  for i = 0 to batchSize step 1 {
    prediction = 0
    for t = 0 to 2 step 1 {
      tree = getTree(forest, t) 
      node = getRoot(tree)
      node = traverseTreeTile(tree, node, rows[i])
      treePrediction = getLeafValue(tree, node)
      prediction = prediction + treePrediction
    }
    for t = 2 to 4 step 1 {
      tree = getTree(forest, t) 
      node = getRoot(tree)
      node = traverseTreeTile(tree, node, rows[i])
      node = traverseTreeTile(tree, node, rows[i])
      treePrediction = getLeafValue(tree, node)
      prediction = prediction + treePrediction
    }
    predictions[i] = prediction
  }  
\end{lstlisting}


\section{Optimizations on mid-level IR}
\label{Sec:MIR}
In this section, we present various tree walk optimizations performed by
\Treebeard{} on the mid-level IR.  

\subsection{Tree Walk Interleaving}
A key bottleneck we found when we profiled code generated from tiled walks (even 
after vectorization) was that true dependencies between instructions were still 
causing a significant number of processor stalls. Performing a walk with a 
single input-tree pair did not provide enough independent instructions to keep 
the processor busy. In order to address this, \Treebeard{} applies an 
unroll-and-jam transformation on the innermost loops of the loop nest. This has 
the effect of walking multiple tree and input row pairs in an interleaved 
fashion.  This mitigates the dependency stalls by enabling scheduling of 
instructions from independent tree walks. 

This optimization is performed in two steps. First, a pass on the mid-level IR 
transforms the loop structure.  It unrolls the innermost loops of the loop nest 
a specified number of times and jams together tree walks from the different 
iterations.  The following listing shows the mid-level IR when the inner loop 
over the input rows is unrolled by a factor of two and the two resulting tree 
walks are jammed together.

\begin{lstlisting}[style=c++]
  for t = 0 to numTrees step 1 {
    for i = 0 to batchSize step 2 {
      tree = getTree(forest, t)
      pred1, pred2 = InterleavedWalk((tree, rows[i]),
                                     (tree, rows[i+1]))
    }
  }
\end{lstlisting}
Next when lowering, the operations to traverse each of the tree, input row pairs 
(the arguments to the \texttt{InterleavedWalk}) are interleaved. One step of the interleaved 
walk is listed below. 
\begin{lstlisting}[style=c++]
  // ... 
  tile1 = tile2 = getRoot(tree)
  // ...
  threshold1 = loadThresholds(tree, tile1)
  threshold2 = loadThresholds(tree, tile2)
  featureIndex1 = loadFeatureIndices(tree, tile1)
  featureIndex2 = loadFeatureIndices(tree, tile2)
  feature1 = rows[i][featureIndex1]
  feature2 = rows[i][featureIndex2]
  pred1 = feature1 < threshold1
  pred2 = feature2 < threshold2
  tile1 = getChildTile(tile1, pred1)
  tile2 = getChildTile(tile2, pred2)
  // ...
\end{lstlisting}

\CommentOut{
These transformations are fairly general and are not aware of the in-memory representation of the model. Therefore, they 
are reusable across different in-memory representations - the ones that are currently built into Treebeard or ones that 
maybe added in the future.
\TODO{AP I feel the way this section is currently written makes the optimization seem extremely trivial. Is there a different 
way to present it?}
}
\subsection{Tree Walk Peeling and Tree Walk Unrolling}
\Treebeard{} splits the loop that performs a tree walk into two parts. It peels 
and introduces a \op{prologue} loop that walks down the tree a constant number 
of steps (for example, up to the depth of the first leaf) and then performs the 
rest of the tree walk in a separate loop.

%% As can be made aware (through a simple pass on the IR) of the depth of the first leaf in a tree, 
%First it identifies the leaf with the minimum depth through a simple pass.  Then

% Next \Treebeard{} unrolls the prologue completely, avoiding all traversal induced branching in it.

Several rewrites of the peeled loop are possible. \Treebeard{} completely 
unrolls the prologue if the peeled loop walks the tree upto the depth of the 
first leaf. In cases where \Treebeard{} has already padded and balanced the tree 
(Section~\ref{sec:treeorder}), unrolling the prologue loop completely avoids all 
traversal induced branching.  In the case of probability-based tiling, the 
prologue loop enables specialization of leaf checks so that these checks are 
faster (and less general) for the most probable leaves.

% \TODO{kr: Add example?}

\subsection{Parallelization}
Currently, \Treebeard{} performs a na\"ive parallelization of the inference computation. When parallelism is enabled, the 
loop over the input rows is parallelized using MLIR's OpenMP support. \Treebeard{} rewrites 
the mid-level IR by tiling the loop over the input rows with a tile size equal to the number of cores. 
As a concrete example, consider the case where we intend to perform inference 
using a model with four trees on a batch of 64 rows. Further, assume that we 
wish to parallelize this computation across 8 cores.  \Treebeard{} then 
generates the following IR:
\begin{lstlisting}[style=c++]
  parallel.for i0 = 0 to 64 step 8 {
    for i1 = 0 to 8 step 1 {
      i = i0 + i1
      prediction = 0
      for t = 0 to 4 step 1 {
        tree = getTree(forest, t) 
        treePrediction = WalkDecisionTree(tree, rows[i])
        prediction = prediction + treePrediction
      }
      predictions[i] = prediction
    }
  }
\end{lstlisting}
Currently, \Treebeard{} does not perform other standard parallelization 
optimizations as they are generic and independent of the 
problem domain. We leave a more thorough exploration of parallelizing decision 
trees to future work.


\section{Optimizations on Low-level IR}
\label{Sec:LIR}
\subsection{Vectorization}
\label{sec:Vectorization}
Vectorization performed by Treebeard is enabled by the tiling transformations described in section \ref{sec:Tiling}. 
When the low level IR is translated to LLVM IR, Treebeard generates LLVM instructions that operate on the threholds and feature indices 
of nodes within a tile in a vector fashion. Therefore, thresholds and feature indices are loaded using vector loads and predicates are 
evaluated using vector comparisons. These vector LLVM IR instructions are then converted to vector instructions in the target ISA by 
the LLVM JIT.

The below listing shows some of the details of a vectorized tree walk. 
\begin{lstlisting}[style=c++]
  // A lookup table that determines the child index of
  // the next tile given the tile shape and the outcome
  // of the vector comparison on the current tile
  int16_t LUT[NUM_TILE_SHAPES, pow(2, TileSize)]
  
  ResultType Prediction_Function(...) {
    // ...
    Node n = getRoot(tree)
    while (isLeaf(tree, n)==false) do {
      thresholds = loadThresholds(tree, n)
      featureIndices = loadFeatureIndices(tree, n)
      // Gather required feature from the current row
      features = rows[i][featureIndices]
      // Vector comparison of features and thresholds
      comparison = features < thresholds
      
      // Pack bits in comparison vector into an integer
      comparisonIndex = combineBitsIntoInt(comparison)
      
      // Get child index of tile we need to move to
      tileShape = loadTileShape(tree, n)
      childIndex = LUT[tileShapeID, comparisonIndex]
      
      // Move to the correct child of the current node
      n = getChildNode(tree, n, childIndex) 
    }
    ThresholdType prediction = getLeafValue(n)
    // ...
  }  
\end{lstlisting}
To evaluate the current tile, the vector of thresholds is first loaded (\texttt{loadThresholds}). This vector contains the thresholds of all nodes in the tile. Then, the features required for comparison are gathered into a vector (lines 11 and 13). The feature vector is compared to the threshold vector and the child tile to move to is determined (lines 15 to 25). More details about tile shapes and the look up table are presented in subsequent sections.

\subsubsection{Tile Shapes}
Informally, the \textbf{\emph{tile shape}} is the shape of the region that encloses all nodes in a tile in a diagram of the decision tree. More formally, for a tile size $n_t$, each unique legal binary tree containing $n_t$ nodes (nodes being indistinguishable) corresponds to a tile shape.

Figure \ref{Fig:TileSize3Shapes} enumerates all tile shapes with a tile size of 3. There are a total of 5 tile shapes with size 3. The number of tile shapes with a tile size $n_t$, denoted by $NTS(n_t)$ is given by the following equation. 

\begin{equation}
  NTS(n) = \sum_{k=0}^{n-1} NTS(k) \times NTS(n-k-1)
\end{equation}

where $NTS(0) = NTS(1) = 1$.

\begin{figure}
  \centering
  \includegraphics[width=\linewidth]{figures/TileShapes_Size3.PNG}
  \caption{All possible tile shapes with tile size $n_t=3$}
  \label{Fig:TileSize3Shapes}
\end{figure}

\subsubsection{Tile Shapes and Decision Tree Inference}
\label{Sec:TileShapesAndDecisionTreeInference}
Treebeard uses vector instructions to accelerate decision tree walks. Vector instructions are used to evaluate the predicates of all the nodes in a tile simultaneously. However, once the predicates of all the nodes in the tile are evaluated, computing the next tile to move to, given the outcome of the comparison depends on the tile shape of the current tile. To illustrate this problem, consider the case of the tiles of size 3 shown in figure \ref{Fig:TileTraversalTileSize3}. 
\begin{figure}
  \centering
  \includegraphics[width=\linewidth]{figures/TileTraversal_Size3.PNG}
  \caption{Example tile traversals with tile size $n_t=3$}
  \label{Fig:TileTraversalTileSize3}
\end{figure}
The diagram shows 3 of the 5 possible tile shapes for a tile size of 3. The nodes drawn in black are members of the tile $t_1$. The nodes in blue are the entry nodes of the children tiles of $t_1$. \TODO{Define entry nodes}

% To traverse a tile on an input row, first, the predicate of each node in the tile is computed. Subsequently, we need to determine which of the child tiles to move to next. Note that a true predicte (bit value 1) on a node implies a move to the left child and a false predicate (bit value 0) implies a move to the right child.

In the diagram, the bit strings (written in red) show which child we need to move to given the outcomes of the comparison. The bits represent the comparison outcomes of nodes and are in the order of the nodes in the tile -- marked 0, 1 and 2 in the diagram, i.e., the MSB is the predicate outcome of node 0 and the LSB the predicate outcome of node 2. For example, for the first tile shape, if the predicates of all nodes are true (i.e. the comparison outcome is 111), the next node to evaluate is $a$. 
% However, if the predicate of node 1 is false, then we need to move to $d$ regardless of the outcomes of nodes 2 and 3.
It is easy to see from the diagram that, depending on the tile shape, the same predicate outcomes can mean moving to different children. For example, for the outcome "011", the next tile is the 4th child (node $d$) for the first two tile shapes while it is the 3rd child for the other tile shape (node $c$).

\subsubsection{Lookup Table}
\label{sec:LookupTable}
A lookup table (LUT) is used to solve the problem described in section \ref{Sec:TileShapesAndDecisionTreeInference}, i.e. given the outcome of the comparisons of all nodes in a tile, determine the child tile we should evaluate next. The LUT is indexed by the tile shape and the comparison outcome. Formally, the LUT is a map.
\[
LUT : (TileShape, < Boolean \times n_t >) \rightarrow [0, n_t] \subset \mathbb{N}
\]

where $n_t$ is the tile size, $< Boolean \times n_t >$ is a vector of $n_t$ booleans. The value returned by the LUT is the index of the child of the current tile that should be evaluated next. For example, if we are evaluating the first tile $t$ in figure \ref{Fig:TileTraversalTileSize3}, and the result of the comparison is 110, then $LUT(TileShape(t), 110)=1$ since the tile we need to evaluate next is the tile with node $b$, which is the second child of the current tile.

In order to realize this LUT in generated code, Treebeard associates a non-negative integer ID with every unique tile shape of the given tile size. The result of the comparison, a vector of booleans, can be interpreted as a 64-bit integer. Therefore, the LUT can be implemented as a 2 dimensional array.
% \begin{lstlisting}{style=c++}
%   int16_t LUT[NTS(n_t), pow(2, n_t)]  
% \end{lstlisting}
Treebeard computes the values in the LUT statically as the tile size is a compile time constant.
\TODO{AP What comes after subsubsection?}

\subsection{In Memory Representation of Tiled Trees}
Treebeard currently has two in memory representations for tiled trees - an array based representation and a sparse representation. Both representations use an array of structs to represent all tiles of the model. 

\subsubsection{Array Based Representation}
\label{sec:ArrayBased}
Each tree in the model is represented as an array of tiles using the standard representation of trees as arrays. The root node is at index 0 and for a node at index $n$ in the array, the index of its $i^{th}$ child is given by $(n_t + 1) n + (i + 1)$ (nodes in the tree of tiles have $n_t + 1$ children). A tile is represented by an object of the following struct.
\begin{lstlisting}{style=c++}
  struct Tile {
    // A vector of TileSize elements
    <ThresholdType x TileSize> thresholds; 
    <FeatureIndexType x TileSize> featureIndices;
    // Integer that identifies the tile shape
    TileShapeIDType tileShapeID; 
  };  
\end{lstlisting}
\TODO{AP Is this level of detail really needed? Also, the vector type notation needs to be introduced somewhere.}
Even though this representation is simple and efficient for small models, the memory required for bigger models is very large. 
%The memory footprint is up to 20X that of the scalar representation.
This memory bloat causes performance problems because the span of the L1 TLB is not sufficient to efficiently translate 
addresses for the whole model. Storing leaves as full tiles (even though leaves just have to represent one value) and the
empty space introduced due to the array based representation of trees that are not complete account for most
of the increase.
%The sparse representation described next tries to address these issues. 

\subsubsection{Sparse Representation}
\label{sec:SparseRep}
The sparse representation tries to address the large memory footprint of the array based representation by doing the following.

\begin{itemize}
  \item We add a child pointer to each tile to eliminate the wasted space in the array representation. This points to the first child of the tile. All children of a tile are stored contiguously.
  \item Leaves are stored as a separate array of scalar values. Across all our benchmarks, after tiling a large fraction of 
  leaves are such that all their siblings are also leaves. Such leaves are directly moved into the leaves array. For leaves
  for which this property does not hold, an extra ``hop'' is added by making the original leaf tile a normal tile. All its
  children are made leaves with the same value as the original leaf.
\end{itemize}

\begin{figure}
  \centering
  \includegraphics[width=\linewidth]{figures/SparseRep_TileSize3.PNG}
  \caption{Sparse representation with tile size $n_t=3$}
  \label{Fig:SparseRep}
\end{figure}

Figure \ref{Fig:SparseRep} shows some of the details of the sparse representation.
% The tree on the left of the diagram is the actual decision tree with the nodes grouped into tiles $t_1$ and $t_2$. The tree on the right is the tree of tiles.
The arrays depicted below show how the tree is represented in memory. The first array ($\texttt{tree}$) is an array of tiles 
and has 5 elements. Each element of the array represents a single tile and has the thresholds of the nodes, the feature
indices, a tile shape ID and a pointer to the first child (shown explicitly in red). 

As a specific example, consider the tile $t_1$. The tile has four children -- $l_1$, $l_2$, $l_3$ and $t_2$ in that order (left to right). These tiles are stored contiguously in the $\texttt{tree}$ array and a pointer to the first of these, $l_1$ is stored in the tile $t_1$ (the index 1 is stored in the tile $t_1$ as shown). 

Now consider the tile $t_2$. Since all children of the tile $t_2$ are leaves, they are all moved into the $\texttt{leaves}$ array.
To store a pointer into the $\texttt{leaves}$ array, we add $\texttt{len(tree)}$ to the element index in the $\texttt{leaves}$ array.
The tile $t_2$'s child is the element at index 12 of the $\texttt{leaves}$ array. Therefore, the index $12 + 5 = 17$ is stored in 
the tile $t_2$. Any index $i$ that is greater than the length of the $\texttt{tree}$ array is regarded as an index into the
$\texttt{leaves}$ array. The index into the $\texttt{leaves}$ array is $i - \texttt{len(tree)}$.

The other aspect of the representation is that an extra hop is added for the leaves $l_1$, $l_2$ and $l_3$ in order to simplify
code generation. This enforces the invariant that all leaves are stored in the leaves array and  simplifies checking whether
we've reached a leaf. Therefore, 4 new leaves are added as children for each of the original leaves $l_1$, $l_2$ and $l_3$. 
Each of these 12 newly added leaves has the same value as its parent. These are the first 12 elements of the $\texttt{leaves}$ array.

Even though we currently have implementations of the two representations detailed in sections \ref{sec:ArrayBased} 
and \ref{sec:SparseRep}, support for other representations is not hard to add. All optimizing passes that work on 
the high level and mid level IR will continue to work as is. Programmers need only provide new lowering passes for
a few operations in the low level IR.

\subsubsection{Code Generation for Probability Based Tiling}
As probability based tiling pulls the most probable leaves of a decision tree nearest the root, it poses 
some implementation challenges. By design, the tiling process makes the tree of tiles 
imbalanced. The array based representation (section \ref{sec:ArrayBased})
cannot be used because of the memory footprint increase (a large part of the tree is empty, but would need to be allocated).
On the other hand, the sparse representation in section \ref{sec:SparseRep} adds 
an extra hop for leaves that have non-leaf siblings. But this would mean that we add extra hops for 
the most probable leaves after probability based tiling which defeats the optimization.

We address these challenges using a code generation strategy. Treebeard peels 
the tree walk and specializes the leaf checks at higher levels to avoid the extra hop. Currently, 
we determine the maximum depth of leaves needed to cover 90 percent of the inputs and peel the tree 
walk by as many iterations. For example, consider the case where leaves until depth 2 are needed to 
cover 90 percent of the training input. Then, Treebeard generates the following IR. 

\begin{lstlisting}{style=c++}
    // ...
    tree = getTree(forest, t)
    node = getRoot(tree)
    node = traverseTreeTile(tree, node, rows[i])
    if (isLeafTile(node)) {
        treePrediction = getLeafValue(tree, node)
    } else {    
        node = traverseTreeTile(tree, node, rows[i])
        if (isLeafTile(node)) {
            treePrediction = getLeafValue(tree, node)
        } else {    
            // Loop based traversal 
        }
    }
    treePrediction = getLeafValue(tree, node)
    // ...
\end{lstlisting}

The if statements check whether a node is a leaf tile and hence avoid the extra hop. 
%The memory 
%requirement is also not increased because only a small fraction of leaves are represented as full tiles.
While walk peeling is used to improve the performance of probability based tiling by specializing leaf tests,
the transformation is by itself general and can be used in different contexts. For example, it could be used 
to elide leaf checks until a depth $d$ is reached if we know all leaves are at a depth greater than $d$. 

One other issue that the code generator needs to handle is that walks of different trees in the same ensemble may 
need to be peeled to different depths. A strategy similar to what is used for uniform tiling is used to handle this.
Trees are reordered so that all trees 
with equal peeling depth are grouped together and the loops in the IR are fissed so that tree walks 
for these groups of trees can be specialized differently.
% This is very similar to the code generation strategy used for uniform tiling.


% \section{Putting it all together}
% Sections~\ref{Sec:HIR} to \ref{Sec:LIR} present various composable optimizations different combinations 
% of which may be needed for different models on different targets. Some of these optimizations like 
% tiling and tree walk interleaving are parameterized to control how aggressively they are applied. 
% In order to find the best combination of optimizations and parameters, several configurations were explored
% for each benchmark. The grid of optimizations explored is listed in Table~\ref{Tab:Optimizations}. 
% \begin{table}[htbp]
%   \small{
%   \begin{tabularx}{\linewidth}{l l}
%    \toprule
%    \textbf{Optimization} & \textbf{Configurations}\\
%    \midrule
%    \multirow{2}{*}{\texttt{Loop order}} &  One tree at a time\\
%                        &  One row at a time\\
%    \midrule
%    \texttt{Tile size} & 1, 2, 4, 8 \\
%    \midrule
%    \multirow{2}{*}{\texttt{Tiling type}} & Basic tiling\\
%    & Probability-based tiling \\
%    \midrule
%    \texttt{Tree padding and unrolling} & Yes, No \\
%    \midrule
%    \texttt{Tree walk interleaving} & 2, 4, 8 \\
%    \bottomrule
%   \end{tabularx}
%   \vskip 5pt
%   \caption{\label{Tab:Optimizations}Space of optimizations explored.}
%   }
% \end{table}
% In the current implementation, we exhaustively explore the grid of optimizations for each 
% model that is compiled. We intend to build an auto-tuner to search over the optimization 
% space more efficiently. 
