%%%%%%%%%%%%%%%%%%%%%%%%%%%%%%%%%%%%
% This is the template for submission to MICRO 2022
% The cls file is modified from 'sig-alternate.cls'
%%%%%%%%%%%%%%%%%%%%%%%%%%%%%%%%%%%%

\documentclass{sig-alternate}
\usepackage{mathptmx} % This is Times font
\usepackage{listings}
\usepackage{fancyhdr}
\usepackage{xcolor}
\usepackage[normalem]{ulem}
\usepackage[hyphens]{url}
\usepackage[sort,nocompress]{cite}
\usepackage[final]{microtype}
\usepackage[keeplastbox]{flushend}
\usepackage{subfig}
% Always include hyperref last
\usepackage[bookmarks=true,breaklinks=true,letterpaper=true,colorlinks,citecolor=blue,linkcolor=blue,urlcolor=blue]{hyperref}
\usepackage{algorithm}
%\usepackage{algorithmic}
\usepackage{algpseudocode}

\newcommand{\CommentOut}[1]{}
\newcommand\TODO[1]{\textcolor{red}{TODO #1}}

\definecolor{codegreen}{rgb}{0,0.6,0}
\definecolor{codegray}{rgb}{0.5,0.5,0.5}
\definecolor{codepurple}{rgb}{0.58,0,0.82}
\definecolor{backcolour}{rgb}{0.95,0.95,0.92}

\lstdefinestyle{c++}{language=C++,
    morekeywords={ensemble, getTree, getRoot, isLeaf, traverseTreeTile, getLeafValue, loadThresholds, loadFeatureIndices, loadTileShape, getChildNode},    
    backgroundcolor=\color{backcolour},   
    commentstyle=\color{codegreen},
    keywordstyle=\color{magenta},
    numberstyle=\tiny\color{codegray},
    stringstyle=\color{codepurple},
    basicstyle=\scriptsize,
    breakatwhitespace=false,         
    breaklines=true,                 
    captionpos=b,                    
    keepspaces=true,                 
    numbers=left,                    
    numbersep=5pt,                  
    showspaces=false,                
    showstringspaces=false,
    showtabs=false,                  
    tabsize=2
}

\lstset{style=c++}

% Ensure letter paper
\pdfpagewidth=8.5in
\pdfpageheight=11in

%%%%%%%%%%%---SETME-----%%%%%%%%%%%%%
\newcommand{\microsubmissionnumber}{XXX}
%%%%%%%%%%%%%%%%%%%%%%%%%%%%%%%%%%%%

\fancypagestyle{firstpage}{
  \fancyhf{}
  \renewcommand{\headrulewidth}{0pt}
  \fancyhead[C]{\vspace{10pt}\normalsize{MICRO 2022 Submission
      \textbf{\#\microsubmissionnumber} -- Confidential Draft -- Do NOT Distribute!!}\\\vspace{-25pt}} 
  \fancyfoot[C]{\thepage}
}

\pagenumbering{arabic}

%%%%%%%%%%%---SETME-----%%%%%%%%%%%%%
\title{Treebeard : An Optimizing Compiler for Tree Based ML Inference} 
%%%%%%%%%%%%%%%%%%%%%%%%%%%%%%%%%%%%

\begin{document}
\maketitle
\thispagestyle{firstpage}
\pagestyle{plain}



%%%%%% -- PAPER CONTENT STARTS-- %%%%%%%%
\begin{abstract}
  
  The rapid proliferation of machine learning coupled with accelerating evolution of the hardware ecosystem has led to a surge in the demand for model inference on a variety of hardware.
  %While CPUs have been the mainstay for machine learning inference, the availability of GPUs holds promise to scale to bigger and more powerful models. 
  This paper is motivated by the problems encountered when targeting inference of decision tree based models, the most popular models on tabular data, to run at peak performance on 
  diverse CPU and GPU targets. We evaluated existing solutions and found that they do not provide portable performance across different hardware targets.
  %Decision tree based models are widely used in practice due to their robustness, interpretability, and ability to handle missing data.  
  
  To address this we present the design of \Treebeard{}, a schedule guided compiler  
  for decision tree based models that searches over a large design space 
  to automatically generate high-performance inference routines.
  %We re-architect the open-source \TreebeardOLD{} infrastructure and significantly extend it to enable high-performance code generation across target processors. 
  \Treebeard{} has two core components. A scheduling language 
  that encapsulates the large optimization space for decision tree inference, 
  and techniques to efficiently explore this space.
  \TODO{Change large optimization space}.
  %We also design a set of heuristics that can find near-optimal solutions in quick time.
  Second, an optimizing multi-level compiler that can generate code based on the schedule. 
  For the latter, we re-architect the open-source \TreebeardOLD{} CPU compiler to support schedule guided compilation and
  add support for GPU code generation. GPU code generation required fundamental new optimization interfaces for 
  caching, parallel reduction, and support for multiple in-memory representations of trees.
  
  %The compiler builds on an existing compiler for CPUs, \TreebeardOLD{}, also add support for caching, parallel reduction and a plug-in mechanism to explore different in-memory representations of trees.

  We evaluate \Treebeard{} on over seven hundred diverse models and demonstrate that the best schedule varies drastically with model, batch size, and target hardware. 
  Our scheduling heuristic is able to quickly find near optimal schedules while searching over a small number (\~50) of schedules.
  In terms of performance, \Treebeard{} generated code is an order of magnitude faster than XGBoost and
  about 2-3$\times$ faster on average than RAPIDs FIL and Tahoe. While these systems only target NVIDIA GPUs, \Treebeard{} achieves competent performance on AMD GPUs as well. 
  On CPUs, \Treebeard{} achieves better scaling compared to \TreebeardOLD{}.
  For models where \TreebeardOLD{} was only able to achieve diminishing returns with an 
  increasing number of threads, \Treebeard{} is able to scale linearly with the number of threads.
  \TODO{(numbers for CPU performance?)}
\end{abstract}
\section{Introduction}
% Decision trees are among the most widely used ML models. Cite Kaggle and looking glass 
Decision tree ensembles are one of the most popular classes of machine learning models \cite{KaggleSurvey,LookingGlass}.
They are generated by machine learning techniques like gradient boosting and random forests. 
The Kaggle state of data science and machine learning survey \cite{KaggleSurvey} shows that 
these are the most widely used classes of models among data scientists. An analysis of 
several machine learning pipelines that are in production use in a real large scale web company showed that 
gradient boosting machines and random forests were the two most widely used ML algorithms in the company \cite{LookingGlass}.
Not only are decision tree based models widely used, they are also used in a diverse range of 
applications\cite{DecisionTreesOverview}. GBM models were used in the CERN large hadron collider
to classify particles based on the data collected from the collider\cite{LHCModel}. Other applications 
include search engines\cite{YahooSearch}, prediction of the financial performance of companies\cite{Finance},
medical diagnostics\cite{Med1, Med2} and recommendation and notification systems\cite{Facebook}.

% Decision tree ensemble overview
To compute the prediction of a decision tree on an input row $x$\footnote{An input row is a
vector of numbers.}, a path from the root to leaf is computed.
Each node of a decision tree has a threshold value $v$ and a feature index (or column index) $i$.
At a node, if $x_i < v$, then 
the walk moves to the left child. If not, we move to the right child. When a leaf is reached, 
the value of the leaf is returned as the prediction of the tree. 
Most decision tree based techniques combine several trees into a forest (or ensemble) to improve prediction accuracy.
To compute the prediction of the forest, the prediction of each tree is computed and these predictions are 
then combined (usually by adding them). 

% Inference on decision trees is hard 
% Pointer chasing -- cache performance, branch prediction, dependency stalls, not easy to vectorize
Given the wide spread use of decision tree ensembles, the ability to perform high throughput and low latency inference is important.
\TODO{AP Since we don't talk about latency or throughput explicitly, should we be saying this?}
However, optimizing the tree walks required for decision tree inference on modern CPUs is not straight-forward. 
Firstly, naively implemented tree walks have poor spatial and temporal locality and therefore poor cache performance \cite{FAST,MilindTreeVectorization}.
Additionally, since the decision of which node to evaluate next can only be made after evaluating the current node,
true dependencies between instructions cause several pipeline stalls. Also, using SIMD instructions to accelerate 
tree walks is extremely challenging \cite{MilindTreeVectorization}.

Several systems like XGBoost\cite{XGBoost}, LightGBM\cite{LightGBM} and Sklearn\cite{Sklearn} implement
decision tree ensemble inference. These are libraries and implement some optimizations. However, these optimizations are hand
coded and need to be manually rewritten or redesigned for every target that is supported. Also, these 
systems cannot apply specialize inference code depending on the model. A compiler based approach is 
needed to address these problems. However, existing compilers only support fixed code generation techniques\cite{Treelite, Hummingbird}.
To fill the need for an extensible compiler infrastructure for decision forest ensembles, we design and 
build \Treebeard{}\footnote{In J. R. R. Tolkein's The Lord of the Rings, Treebeard was the oldest of the Ents left in Middle-earth, an 
ancient tree-like being who was a ``shepherd of trees''.}, an optimizing compiler for decision tree ensemble inference.
Our motivations for building \Treebeard{} are as follows. 
\begin{enumerate}
  \item Existing libraries \cite{XGBoost, Treelite, LightGBM, VPred} perform specific optimizations and are hard to maintain as hardware evolves. 
  Additionally, they cannot tailor inference code to the specific model being used. Past work has identified that the optimizations required change with the model
  and architectural parameters like cache size\cite{CacheConscious1, CacheConscious2, VPred}. 
  \item The repeated effort to optimize and maintain libraries on several targets is prohibitive. The proliferation of new architectures
  further exacerbates this problem. Compilers have been successful in alleviating this problem in other domains\cite{Halide, TVM}.
  However, no such compiler infrastructure currently exists for decision tree ensembles.
  \item Currently, no system exists that allows a thorough exploration of the optimization space for decision tree ensemble inference. 
  Compilers have been successful in addressing this problem with ML models like DNNs \cite{TVM, Tiramisu, XLA}. However, compiler techniques for
  decision tree ensembles are less well studied.
\end{enumerate}

We make the following contributions.
\begin{enumerate}
  \item We design and build \Treebeard{}, an extensible compiler infrastructure for decision tree model inference. The 
  infrastructure is built to allow exploration of optimization and code generation techniques. \TODO{Is this claim too grand?}
  \item We develop a general infrastructure for the vectorization of decision tree walks based on grouping tree nodes into ``tiles''.
   This includes general support for code generation and the in-memory representation of tiled trees. The infrastructure can be 
   used to tile trees based on different cost functions.
  \item We show that trees can be tiled using different cost functions. We present two novel tiling methods that are implemented
  using the general tiling infrastructure. 
  \item We design and implement various model and loop transformations that significantly improve generated inference code performance 
  to show the power of the proposed framework.
\end{enumerate}



\section{Compiler Overview}
Figure \ref{Fig:CompilerStructure} shows the high level structure of Treebeard. The input to the compiler is a JSON file that describes the decision forest model and a set of options. The output of the compiler is a callable function that takes an array of rows and returns an array containing the model's prediction for each row. \TODO{Add a diagram, including IR stages}. The specified set of options includes information such as the batch size, the type of the input features, the type for node thresholds and the type for feature indices \TODO{There are also several optimization related inputs like tile size, type of tiling, pipeline depth etc. Should we mention those?}. Treebeard is written as a dialect in the MLIR framework \cite{MLIR}. At a high level, Treebeard first performs transformations on the trees in the model and subsequently generates and performs optimizations on a more traditional loop based IR. From an implementation perspective, Treebeard first constructs a high level representation of the decision forest inference operation and then progressively optimizes and lowers it to LLVM IR \cite{LLVM}. LLVM is then used to JIT compile the generated IR to executable code. The following sections describe each level of the IR and their lowering in more detail.

\begin{figure}
  \centering
  \includegraphics[scale = 0.3]{figures/CompilerStructure.PNG}
  \caption{Treebeard Compiler Structure}
  \label{Fig:CompilerStructure}
\end{figure}

\TODO{Should we describe the dialect's type system?}

\subsection{High Level IR}
Treebeard parses the input JSON file and generates a function with a single MLIR operation, \texttt{predictForest} that represents inference using the input model on a set of rows. The operation contains within it a tree based representation of the model that can be manipulated by optimizing transformations. Transformations on the model such as tiling, tree reordering and leaf padding are done at this level. The structure of the loop nest to walk the iteration space of trees and inputs is also decided at this level of the IR. \TODO{Should we mention that there is a scheduling language to decide this?}

\begin{lstlisting}[language=C++]
func Predict(float rows[batchSize]) {
  predictions = predictForest(rows) 
  return rows
}
\end{lstlisting}

\subsection{Mid Level IR}
The Mid Level IR makes the loop structures and tree walks more explicit. Firstly, the order in which the iteration space of trees and inputs is walked is explicitly specified in the IR through loop nests. Also, tree specific operations such as \texttt{isLeaf}, \texttt{traverseTile}, \texttt{getLeafValue} are introduced so that the traversal of trees explicitly represented. The following listing shows the IR for inference using a model with four trees on an input batch with two rows. The listed IR walks all trees for one input row before moving to the next row. One important point to note here is that details such as the data structure used for the trees are not explicitly encoded in the IR. This allows us to reuse optimization and lowering passes on this level of the IR regardless of what the final in memory representation of the model is.

\begin{lstlisting}[language=C++]
  // Constant that represents the model being compiled
  forest = ensemble(...)
  for i = 0 to 2 step 1 {
    prediction = 0
    for t = 0 to 4 step 1 {
      tree = getTree(forest, t) 
      node = getRoot(tree)
      while (isLeaf(tree, n)==false)  do {
        node = traverseTreeTile(tree, node, rows[i])
      }
      treePrediction = getLeafValue(tree, node)
      prediction = prediction + treePrediction
    }
    predictions[i] = prediction
  }
\end{lstlisting}

The IR listed above is a simplification of the actual IR. The actual IR is strongly typed and in SSA form.

\subsection{Low Level IR}
The IR is finally lowered to a form where the in memory representation of the model is made explicit. Buffers to hold the model values are inserted into the generated code and all tree operations in the mid-level IR are lowered to explicitly reference these buffers. The semantics of all operations are made explicit. For example, \texttt{traverseTreeTile} is lowered into a series of operations to load thresholds, feature indices and features, compare the features with the thresholds and compute the next node to evaluate. This IR is then lowered directly to LLVM IR and JITted.


%\section{Optimizations}
\section{high level IR Optimizations}
This section describes tiling and tree ordering, two optimizations performed at the highest level of abstraction. Recall that here the \op{predictForest} operator is abstractly represented by just a set of trees. 

%\TODO{This section needs a better name!}
\subsection{Notation}
%TODO{Notation needs to be introduced in the background section, Tiling should be the first subsection}
We represent a decision tree by $\Tree = (V, E, r)$ where $V$ is the set of nodes, $E$ the set of edges and
$r \in V$ is the root. For each node $n \in V$, we define the following.
\begin{enumerate}
    \item $threshold(n) \in \mathbb{R}$, the threshold value for $n$.
    \item $featureIndex(n) \in \mathbb{N}$, the feature index for $n$.
    \item $left(n) \in V$, the left child of $n$ or $\emptyset$ if $n$ is a leaf. If $left(n) \neq \emptyset$, then $(n, left(n)) \in E$.
    \item $right(n) \in V$, the right child of $n$ or $\emptyset$ if $n$ is a leaf. If $right(n) \neq \emptyset$, then $(n, right(n)) \in E$.
\end{enumerate}
We use $L_{\Tree} \subseteq V$ to denote the set of leaves. % subseteq because tree could have single node

\subsection{Tiling}
\label{sec:Tiling}
% Treebeard vectorizes tree walks by grouping nodes of a decision tree into \textbf{\emph{tiles}}. The nodes in a tile are evaluated concurrently using vector instructions. Once the nodes of the current tile are evaluated, a look up table is used to compute which child of the current tile to move to next.

While a decision tree is naturally represented by a binary tree, 
it is not the best representation for tree traversal as it (i) requires many memory accesses, 
(ii) has poor branching structure and (iii) cannot make use of vector instructions. 
This section proposes a tiling optimization where we group multiple nodes in a decision tree into a single tile, effectively transforming a binary tree into an $n$-ary tiled tree. This not only allows the compiler to generate vectorized code (see Section~\ref{sec:Vectorization} ) to traverse trees but also enables spatial locality improvements by grouping nodes that are likely to be accessed together. We demonstrate this with two tiling heuristics later in the section. 



\CommentOut{
Treebeard groups nodes of the decision tree into \textbf{\emph{tiles}}. Tiling provides two benefits. 
\begin{enumerate}
  \item It allows the compiler to generate vector code to traverse trees. Section \ref{sec:Vectorization} describes how Treebeard does this.
  \item It enables spatial locality improvements by grouping together nodes that are likely to be accessed together. 
\end{enumerate}
Once nodes are grouped into tiles, an $n$-ary tree whose nodes are tiles is constructed. Treebeard then generates 
optimized code to walk this tree. The listing below shows at a high level how a tiled tree is walked (This is not 
true IR, but presented for clarity). 
}

Once trees are tiled Treebeard generates tree walks with the code structure shown below.
\begin{lstlisting}[style=c++]
  ResultType Prediction_Function(...) {
    // ...
    Tile t = getRootTile(tree)
    while (!isLeaf(tree, t)) do {
      // Evaluate predicates of all nodes in the tile
      predicates = evaluateTilePredicates(t, rows[i])
      
      // Move to the correct child of the current tile
      t = getChildTile(tree, t, predicates) 
    }
    treePrediction = getLeafValue(t)
    // ...
  }  
\end{lstlisting}
The code is just an abstract representation of a tiled tree walk that enables efficient lowering of specific steps in subsequent stages. 
\op{evaluateTilePredicates} (speculatively) computes the predicates on all nodes in a tile (line 6). Then \op{getChildTile} (line 9), uses the computed predicate values to determine which child of the current tile to move to. We defer a description of how these operators are lowered to a later section but focus on tiling algorithms in this section.

\CommentOut{
To compute the prediction of the tree, the predicates of all nodes in the tile are computed simultaneously (line 6). 
Then, the computed predicate values are used to determine which child of the current tile to move to (in the 
$n$-ary tree). This section presents the details of tiling and how Treebeard's general tiling infrastructure 
can be used to develop tiling algorithms with different objectives (sections \ref{sec:UnifTiling}) and \ref{sec:ProbTiling}).
The details of how Treebeard lowers predicate evaluation and moving to the correct child to use vector instructions 
are described in section \ref{sec:Vectorization}.

\subsection{Tiles and Tree Tiling}
\label{sec:ValidTiling}
}
\subsubsection*{Conditions for Efficient Tiling}
\label{sec:ValidTiling}
While any arbitrary partitioning of the nodes of a tree could be considered for tiling we impose a few intuitive constraints.
% to only allow nodes that are likely to be accessed together to be grouped into a tile. 
% \TODO{kr : replace $n_t$ with sz?}
Given a tree $\Tree = (V, E, r)$ and a tile size $n_t$ we impose the following constraints on the generated tiles $\{ T_1, T_2, ... ,T_m \}$ .
\begin{description}
    \item[Partitioning] $T_1 \cup T_2 ... \cup T_m = V$ and $T_i \cap T_j = \emptyset$ for all $i\neq j$
    \item[Connectedness] If $u, v \in T_i$, there is a (undirected) path connecting $u$ and $v$ fully contained in $T_i$.
   \item [Leaf seperation] $\forall l \in L_{\Tree}$ : $l \in T_i \rightarrow v \notin T_i \;\; \forall v \in V \backslash \{l\}$
  \item [Maximal tiling] if there are tiles such that. $|T_i| < n_t$ then there is no $v \in V\backslash \{ T_i \cup L_{\Tree} \}$ such that $(u, v) \in E$ for some $u \in T_i$. 
\end{description}
The \textbf{partitioning} and \textbf{maximal tiling} constraints together ensure that we group nodes into as few tiles as possible. {\textbf{Connectedness}} ensures that each tile is a sub-tree, a natural grouping of nodes that are likely to be accessed together. The {\textbf{Leaf separation}} constraint ensures that leafs are not tiled along with internal nodes. Leafs in a decision tree need special handling, they are used to check for walk termination and to determine the output (prediction). This constraint ensures that tiles are homogenous, this in-turn allows us to specialize the in-memory layout of trees and also simplifies code generation. We discuss leaf handling and tree layout in section ~\ref{Sec:MemoryRep}. We refer to any tiling that satisfies the above constraints as a \emph{valid} tilling.

%%% COMMENT %%%% 
\CommentOut{
\subsection{Tiled Trees}
A tiling transformation communicates the tiling to the Treebeard infrastructure by assigning a tile ID to each node in the decision tree. Using these tile IDs, Treebeard checks the validity of the tiling and then contructs a tree whose nodes are tiles. We call this tree the \textbf{\emph{tree of tiles}}. \TODO{We need a better name for this}
Figure \ref{Fig:ValidTilingTileSize3} shows a valid tiling with tile size 3 and the tree of tiles constructed by Treebeard. Three nodes are grouped into each of the tiles $t_1$ and $t_2$ as shown. Each tile is collapsed into a single node in the tree of tiles. However, each leaf in the original tree becomes a leaf in the tree of tiles.

\begin{figure}
  \centering
  \includegraphics[width=\linewidth]{figures/TiledTree_Size3.PNG}
  \caption{Example of a valid tree tiling with tile size $n_t=3$}
  \label{Fig:ValidTilingTileSize3}
\end{figure}

Treebeard maintains the following invariants.
\begin{enumerate}
  \item All tiles in a tree are the same size $n_t$. If the tiling produces any smaller tiles, these are padded by inserting dummy nodes to make them the required size.
  \item Nodes within tiles are always ordered in level order and left to right within a level. The numbering of the nodes in the above diagram shows this node order.
  \item Children of a node are numbered from left to right (regardless of level). For example, $l_1$ is the first child of $t_1$, $l_2$ is the second and so on.
\end{enumerate}
}
%%% END COMMENT %%%%

\subsection{Basic Tiling}
\label{sec:UnifTiling}
% The justification for basic tiling -- if we assume all paths are equally likely,
% we want to minimize the depth of speculation
Algorithm~\ref{Alg:UnifTilingAlgo} shows the basic tiling algorithm that produces a valid tiling. It 
attempts to minimize the depths of all constructed tiles. 
Tiling starts at the root and constructs a tile $Tile$ by performing
a level order traversal. The call \op{LevelOrderTraversal($\Tree$, $n_t$)} picks the next $n_t$ non-leaf nodes according to the standard 
level order tree traversal algorithm. Once the current tile is constructed, 
the tiling procedure is recursively performed on subtrees rooted at each node that is a destination of an edge going out of the constructed 
tile. It is easy to see that the tiling constructed by Algorithm \ref{Alg:UnifTilingAlgo} is valid.
% \TODO{Pass a modified tree without leaves to the level order traversal call. Explicitly add leaves as seperate tiles}
% Algorithm for basic tiling
\begin{algorithm}
  \caption{Basic tree tiling}
  \label{Alg:UnifTilingAlgo}
  \small{
  \begin{algorithmic}[1]
      \Procedure{LevelOrderTraversal}{$\Tree = (V, E, r)$, $n_t$}
        \State $queue = \{ r \}$
        \State $Tile = \emptyset$
        \While{$\neg queue.empty() \wedge |Tile|<n_t$}
          \State $n = queue.dequeue()$
          \If {$n \in L_{\Tree}$}
              \State \textbf{continue}
          \EndIf
          \State $Tile = Tile \cup \{ n \}$
          \State $queue.enqueue([left(n), right(n)])$
        \EndWhile
      \EndProcedure
      \State {}
      \Procedure{TileTree}{$\Tree = (V, E, r)$, $n_t$} 
          \If {$r \in L_{\Tree}$}
              \State \textbf{return} $\{ r \}$
          \EndIf
          % \State \textcolor{codegreen}{\textit{//Level order traversal to collect $n_t$ or fewer nodes. }}
          % \State \textcolor{codegreen}{\textit{//Leaves are not included in the constructed tile. }}
          \State $Tile \leftarrow LevelOrderTraveral(\Tree, n_t)$
          \State $Tiles =  \{ Tile \}$
          \For{$(u,v) \in Out(Tile)$}
              \State $Tiles \leftarrow Tiles \cup TileTree(S_v, n_t)$
          \EndFor
          \State \textbf{return} $Tiles$
      \EndProcedure
  \end{algorithmic}
  }
\end{algorithm}

One interesting property of this tiling algorithm is that it naturally reduces 
the imbalance in trees, especially at large tile sizes. As the algorithm 
traverses down to sparser levels of the tree, it naturally groups sub-trees 
containing chains of nodes, thus balancing the trees. While it is possible to 
further enhance the algorithm to explicitly balance tiled trees, we find that 
basic tiling suffices in practice.

%%%% COMMENT %%%%% 
\CommentOut{
\subsubsection{Further Opimization and Code Generation}
We found that most leaf tiles for a given tree are at the same depth when basic tiling is used. Furthermore, we see that deeper leaves 
are more likely to be reached.
%\footnote{Intuitively, this is true because training algorithms keep splitting nodes to maximize gain and gain
%will typically be maximized by splitting a large number of inputs.}.
Based on these observations, we (optionally) pad the tree of tiles generated with basic tiling so that all leaves are at the same depth.
This transformation is performed on the high level IR after basic tiling. 
Once the trees have been padded to make all leaves equal depth, the tree walks are fully unrolled to evaluate a fixed 
number of tiles and all leaf checks are omitted.

One other complication the code generator needs to handle is the fact that different trees in the model being 
compiled potentially have different depths. In order to handle
this, Treebeard sorts the trees by their depth. This ensures that all trees with equal depth are grouped together. Once this is done, 
the loop over the trees is fissed so that each of the resulting loops only walks trees of a single depth. Consider for example a 
forest with 4 trees $T_1$, $T_2$, $T_3$, and $T_4$ in that order. Further, assume that $T_1$ and $T_4$ have depth 2 while $T_2$ and $T_3$
have depth 3. First, Treebeard reorders the trees to be in the order $T_1$, $T_4$, $T_2$, $T_3$. Then, the loop over the trees is fissed
as shown in the following listing.

% loop transformations for basic tiling (splitting) 
\begin{lstlisting}{style=c++}
  forest = ensemble(...)
  for i = 0 to batchSize step 1 {
    prediction = 0
    for t = 0 to 2 step 1 {
      tree = getTree(forest, t) 
      node = getRoot(tree)
      node = traverseTreeTile(tree, node, rows[i])
      treePrediction = getLeafValue(tree, node)
      prediction = prediction + treePrediction
    }
    for t = 2 to 4 step 1 {
      tree = getTree(forest, t) 
      node = getRoot(tree)
      node = traverseTreeTile(tree, node, rows[i])
      node = traverseTreeTile(tree, node, rows[i])
      treePrediction = getLeafValue(tree, node)
      prediction = prediction + treePrediction
    }
    predictions[i] = prediction
  }  
\end{lstlisting}

\TODO{AP: This listing is unnecesarily long. Can we maybe leave out the loop bodies and say something like "depth 2 walk"? Should 
we point to the figure in the overview section instead?}
}
%%%% COMMENT END %%%%% 

\subsection{Probability Based Tiling}
\label{sec:ProbTiling}
%\subsubsection*{Motivation}

The next algorithm we propose exploits the inherent biases among the leaves of a decision tree. In typical machine learning models some leafs (equivalently outcomes or predictions) are more likely to be reached than others. In such settings, having balanced tiled trees is not sufficient to minimize expected inference time. 

Consider for example two machine learning models \op{airline-ohe} and \op{epsilon} (also used in our evaluation). 
Consider the graphs shown in   
figures \ref{Fig:AirlineOHEStats} and \ref{Fig:EpsilonStats} that are generated from training data. Each line in these graphs corresponds to a fixed fraction of the input (say $f$). 
A point on a line at coordinate $(x, y)$ means that a fraction $y$ of trees in the model could cover a fraction $f$ of all training inputs with a fraction $x$ of 
leaves. For example, the first point on the $f=0.9$ line in figure \ref{Fig:AirlineOHEStats} says that about 52\% of trees ($y$ value) need only 1\% of their
leaves ($x$ value) to cover 90\% of the training input. 
In general, Figure \ref{Fig:AirlineOHEStats} shows that very few leaves are needed to cover a very large fraction of inputs for the benchmark \op{airline-ohe}. 
This means that a small fraction of leaves are very likely. 
We call trees with a small number of extremely likely leaves \textbf{\emph{leaf biased}}.

On the other hand, for the benchmark \op{epsilon},
figure \ref{Fig:EpsilonStats} shows that a trees need a much larger fraction of their leaves to cover a significant fraction of the training input.
This means that most trees in \op{epsilon} are not leaf biased.

\CommentOut{
\subsubsection{More Notation}
In order to formulate the probability based tiling algorithm as an optimization problem, we define the following.
\begin{enumerate}
    \item For every leaf $l \in L$, we define $p_l$ as the probability that the leaf $l$ is reached.
    \item For each node $n \in V$, we define the absolute probability $p_v$ as
    \begin{equation}
        p_v = \begin{cases}
        p_l &\text{if $l \in L$}\\
        p_{left(v)} + p_{right(v)} &\text{otherwise}
        \end{cases}
    \end{equation}
    \item For any tree $T$, $\mathcal{C}(T)$ represents the set of all valid tilings of $T$.
    \item For every $v \in V$, we define $S_v$ as the subtree rooted at $v$.
    \item For every $v \in V$, we define $L_v$ as the set of leaves of $S_v$.
    \item For a every tile $T_i$, we define $root(T_i)$ as the node $v \in T_i$ such that $v$ has no incoming edges from any other node $u \in T_i$.
    \item For a tile $T_i$, $out(T_i) \subseteq E$ is the set of edges $(u, v)$ such that $u \in T_i$ and $v \notin T_i$.
\end{enumerate}
}

\begin{figure}
    \centering
    \includegraphics[width=\linewidth]{figures/airline-ohe.stats.train.txt.png}
    \caption{Statistical profile for airline-ohe}
    \label{Fig:AirlineOHEStats}
\end{figure}
\begin{figure}
    \centering
    \includegraphics[width=\linewidth]{figures/epsilon.stats.train.txt.png}
    \caption{Statistical profile for epsilon}
    \label{Fig:EpsilonStats}
\end{figure}
\CommentOut{
We say that an input row $r_i$ is \textbf{\emph{covered}} by a subset of leaves $L' \subseteq L$ of a tree $T$, if the leaf $l$ reached by 
walking $T$ for row $r_i$ is in $L'$. We show how different models (and even different trees within the same model) behave differently 
using models for two benchmarks, airline-ohe and epsilon. Consider the graphs shown in   
figures \ref{Fig:AirlineOHEStats} and \ref{Fig:EpsilonStats}. Each line in these graphs corresponds to a fixed fraction of the input (say $f$). 
A point on a line at coordinate $(x, y)$ means that a fraction $y$ of trees in the model could cover a fraction $f$ of all training inputs with a fraction $x$ of 
leaves. For example, the first point on the $f=0.9$ line in figure \ref{Fig:AirlineOHEStats} says that about 52\% of trees ($y$ value) need only 1\% of their
leaves ($x$ value) to cover 90\% of the training input. 
In general, Figure \ref{Fig:AirlineOHEStats} shows that very few leaves are needed to cover a very large fraction of inputs for the benchmark airline-ohe. 
This means that a small fraction of leaves are very likely. On the other hand, for the benchmark epsilon,
figure \ref{Fig:EpsilonStats} shows that a trees need a much larger fraction of their leaves to cover a significant fraction of the test input.
This means that most trees in the epsilon model are not leaf biased.
One other observation we make is that most models have some leaf biased trees while the rest of the trees have equally likely leaves.
\TODO{AP Maybe define a term for trees with roughly equally likely leaves?} We design the probability based tiling algorithm to take advantage of this property 
of decision tree ensembles. 
}
\subsubsection{The Optimization Problem}

%We assume that we are given the probabilities of each leaf node of the decision tree (these can easily be computed using the training data). For every leaf $l \in L$, we %are given the probability $p_l$ that the leaf $l$ is reached. 

Observe that the latency of one tree walk is proportional to the number of tiles that need to be evaluated to reach the leaf. It is easy to see that for a leaf biased tree, basic tilling does not optimize for this objective, it considers all leafs to be equally likely. 
 
The goal of probablistic tiling is to minimize the average inference latency, or equivalently the minimize the expected number of tiles that are evaluated to compute one tree prediction. More formally, the problem is to find a \emph{valid} (as defined in Section~\ref{sec:ValidTiling}) tiling $\mathcal{T}$ such that the following objective is minimized.
\[
    \min_{\mathcal{T} \in \mathcal{C}(T)}{\sum_{l \in L_{\Tree}} p_l.depth_{\mathcal{T}}(l)}
\]
where the minimization is over all valid tilings $\mathcal{T}$ of the tree $\Tree$, $depth_{\mathcal{T}}(l)$ is the depth of the leaf $l$ given tiling ${\mathcal{T}}$. $p_l$ is the probability of of reaching leaf $l$ as observed during training.

The above optimization problem can be solved optimally using dynamic programming. 
We leave this out in the interest of space. 
Instead, we use the simple greedy algorithm listed in algorithm \ref{Alg:GreedyTilingAlgo} to construct a valid tiling given the node probabilities\footnote{Probabilites for internal nodes can be computed from probablities for leafs by summing up the probabilities of all leafs that belong to the sub-tree rooted at the internal node. Leaf probabilities are collected during training.}.
The algorithm starts at the root and greedily keeps adding the most probable legal node to the current tile until the maximum tile size is reached.
Subsequently, the tiling procedure is recursively performed on all nodes that are destinations for edges going out of the constructed tile.

% \subsubsection{Dynamic Programming Formulation}


% For any node $v \in V$, we define
% \[
%     cost(v, \mathcal{T}) = \sum_{l \in L_v} p(l | v).depth_{\mathcal{T}}(l)
% \]
% where $\mathcal{T} \in \mathcal{C}(T_v)$.

% Then, the objective function, for the tree $T_v$, can be rewritten as 
% \[
%     opt\_cost(v) = \min_{\mathcal{T} \in \mathcal{C}(T_v)}{cost(v, \mathcal{T})}
% \]

% The objective function can then be rewritten in the following recursive form.
% \[
%     opt\_cost(v) = \min_{T_0 \in TileShapes(n_t, v)}{1 + \sum_{(n_1, n_2) \in out(T_0)} p(n_1 | v)p(n_2 | n1)opt\_cost(n_2)}
% \]
% where $TileShapes(n_t, v)$ is the set of all tile shapes of size $n_t$ with root $v$. A straight forward substitution argument shows why the solution to the subproblems (tiling all sub-trees) needs to to be optimal. The objective is now in a form that can solved using
% dynamic programming. 

% \subsubsection{Greedy Algorithm}

% Intuitively, it seems like the following greedy algorithm also gives the optimal tiling. The algorithm starts at the root and greedily keeps adding the most probable node to the current tile until the maximum tile size is reached.
\begin{algorithm}
    \caption{Greedy Probability Based Tree Tiling}
    \label{Alg:GreedyTilingAlgo}
    \begin{algorithmic}
        \Procedure{TileTree}{$\Tree = (V, E, r)$, $n_t$} 
            \If {$r \in L_{\Tree}$}
                \State \textbf{return} $\{ r \}$
            \EndIf
            \State $Tile \leftarrow \{ r \}$
            \While{$|Tile| < n_t$}
                \State $e = (u,v) \in Out(Tile)$ st $p(v)$ is max and $v \notin L$
                \If{$e = \emptyset$}
                    \State \textbf{break}
                \EndIf
                \State $Tile = Tile \cup \{ v \}$
            \EndWhile
            \State $Tiles =  \{ Tile \}$
            \For{$(u,v) \in Out(Tile)$}
                \State $Tiles \leftarrow Tiles \cup TileTree(S_v, n_t)$
            \EndFor
            \State \textbf{return} $Tiles$
        \EndProcedure
    \end{algorithmic}
\end{algorithm}

% Talk about problems with increasing number of tile shapes and only performing such tiling on skewed trees
\CommentOut{
When we tried to apply algorithm \ref{Alg:GreedyTilingAlgo} on all trees in our benchmarks, we found
that even minor variations in probability caused the tiling algorithm to generate a large 
number of tile shapes. This in turn caused a loss in performance because the large size of the 
lookup table needed (section \ref{sec:LookupTable}) caused increased L1 cache misses. In order to 
alleviate this, we only perform probability based tiling on trees that are leaf biased.
}
We find probability based tiling is only beneficial for leaf biased trees\footnote{Turns out that for trees that are not leaf biased, probability based tiling produces many more tile shapes (see Section~\ref{sec:tileShapes}) which direclty impacts the cost of \op{getChildTile} making it more expensive than basic tiling.}.  Recall that   
a tree to be leaf biased if a small fraction of leaves, say $\alpha$, can cover a large fraction of training inputs, say $\beta$.
We only perform probability based tiling on trees with thresholds $\alpha=0.05$ and $\beta=0.9$ and fall back to uniform tiling otherwise. 





\subsection{A Note on Implementation}
\TODO{Kr : Not sure if this is needed}
The tiling algorithms generate a \op{TileId} attribute per tree. The \op{TileId} attribute contains a mapping from a Node to the TileId asigned to it.
This information is used when lowering to the mid level abstraction in the form of loops. A sample of the lowered MIR code is shown in Figure~\ref{Fig:Overview}.	
\subsection{Basic Tiling}
\label{sec:UnifTiling}
% The justification for basic tiling -- if we assume all paths are equally likely,
% we want to minimize the depth of speculation
Algorithm~\ref{Alg:UnifTilingAlgo} shows the basic tiling algorithm that produces a valid tiling. It 
attempts to minimize the depths of all constructed tiles. 
Tiling starts at the root and constructs a tile $Tile$ by performing
a level order traversal. The call \op{LevelOrderTraversal($\Tree$, $n_t$)} picks the next $n_t$ non-leaf nodes according to the standard 
level order tree traversal algorithm. Once the current tile is constructed, 
the tiling procedure is recursively performed on subtrees rooted at each node that is a destination of an edge going out of the constructed 
tile. It is easy to see that the tiling constructed by Algorithm \ref{Alg:UnifTilingAlgo} is valid.
% \TODO{Pass a modified tree without leaves to the level order traversal call. Explicitly add leaves as seperate tiles}
% Algorithm for basic tiling
\begin{algorithm}
  \caption{Basic tree tiling}
  \label{Alg:UnifTilingAlgo}
  \small{
  \begin{algorithmic}[1]
      \Procedure{LevelOrderTraversal}{$\Tree = (V, E, r)$, $n_t$}
        \State $queue = \{ r \}$
        \State $Tile = \emptyset$
        \While{$\neg queue.empty() \wedge |Tile|<n_t$}
          \State $n = queue.dequeue()$
          \If {$n \in L_{\Tree}$}
              \State \textbf{continue}
          \EndIf
          \State $Tile = Tile \cup \{ n \}$
          \State $queue.enqueue([left(n), right(n)])$
        \EndWhile
      \EndProcedure
      \State {}
      \Procedure{TileTree}{$\Tree = (V, E, r)$, $n_t$} 
          \If {$r \in L_{\Tree}$}
              \State \textbf{return} $\{ r \}$
          \EndIf
          % \State \textcolor{codegreen}{\textit{//Level order traversal to collect $n_t$ or fewer nodes. }}
          % \State \textcolor{codegreen}{\textit{//Leaves are not included in the constructed tile. }}
          \State $Tile \leftarrow LevelOrderTraveral(\Tree, n_t)$
          \State $Tiles =  \{ Tile \}$
          \For{$(u,v) \in Out(Tile)$}
              \State $Tiles \leftarrow Tiles \cup TileTree(S_v, n_t)$
          \EndFor
          \State \textbf{return} $Tiles$
      \EndProcedure
  \end{algorithmic}
  }
\end{algorithm}

One interesting property of this tiling algorithm is that it naturally reduces 
the imbalance in trees, especially at large tile sizes. As the algorithm 
traverses down to sparser levels of the tree, it naturally groups sub-trees 
containing chains of nodes, thus balancing the trees. While it is possible to 
further enhance the algorithm to explicitly balance tiled trees, we find that 
basic tiling suffices in practice.

%%%% COMMENT %%%%% 
\CommentOut{
\subsubsection{Further Opimization and Code Generation}
We found that most leaf tiles for a given tree are at the same depth when basic tiling is used. Furthermore, we see that deeper leaves 
are more likely to be reached.
%\footnote{Intuitively, this is true because training algorithms keep splitting nodes to maximize gain and gain
%will typically be maximized by splitting a large number of inputs.}.
Based on these observations, we (optionally) pad the tree of tiles generated with basic tiling so that all leaves are at the same depth.
This transformation is performed on the high level IR after basic tiling. 
Once the trees have been padded to make all leaves equal depth, the tree walks are fully unrolled to evaluate a fixed 
number of tiles and all leaf checks are omitted.

One other complication the code generator needs to handle is the fact that different trees in the model being 
compiled potentially have different depths. In order to handle
this, Treebeard sorts the trees by their depth. This ensures that all trees with equal depth are grouped together. Once this is done, 
the loop over the trees is fissed so that each of the resulting loops only walks trees of a single depth. Consider for example a 
forest with 4 trees $T_1$, $T_2$, $T_3$, and $T_4$ in that order. Further, assume that $T_1$ and $T_4$ have depth 2 while $T_2$ and $T_3$
have depth 3. First, Treebeard reorders the trees to be in the order $T_1$, $T_4$, $T_2$, $T_3$. Then, the loop over the trees is fissed
as shown in the following listing.

% loop transformations for basic tiling (splitting) 
\begin{lstlisting}{style=c++}
  forest = ensemble(...)
  for i = 0 to batchSize step 1 {
    prediction = 0
    for t = 0 to 2 step 1 {
      tree = getTree(forest, t) 
      node = getRoot(tree)
      node = traverseTreeTile(tree, node, rows[i])
      treePrediction = getLeafValue(tree, node)
      prediction = prediction + treePrediction
    }
    for t = 2 to 4 step 1 {
      tree = getTree(forest, t) 
      node = getRoot(tree)
      node = traverseTreeTile(tree, node, rows[i])
      node = traverseTreeTile(tree, node, rows[i])
      treePrediction = getLeafValue(tree, node)
      prediction = prediction + treePrediction
    }
    predictions[i] = prediction
  }  
\end{lstlisting}

\TODO{AP: This listing is unnecesarily long. Can we maybe leave out the loop bodies and say something like "depth 2 walk"? Should 
we point to the figure in the overview section instead?}
}
%%%% COMMENT END %%%%% 

\subsection{Probability Based Tiling}
\label{sec:ProbTiling}
%\subsubsection*{Motivation}

The next algorithm we propose exploits the inherent biases among the leaves of a decision tree. In typical machine learning models some leafs (equivalently outcomes or predictions) are more likely to be reached than others. In such settings, having balanced tiled trees is not sufficient to minimize expected inference time. 

Consider for example two machine learning models \op{airline-ohe} and \op{epsilon} (also used in our evaluation). 
Consider the graphs shown in   
figures \ref{Fig:AirlineOHEStats} and \ref{Fig:EpsilonStats} that are generated from training data. Each line in these graphs corresponds to a fixed fraction of the input (say $f$). 
A point on a line at coordinate $(x, y)$ means that a fraction $y$ of trees in the model could cover a fraction $f$ of all training inputs with a fraction $x$ of 
leaves. For example, the first point on the $f=0.9$ line in figure \ref{Fig:AirlineOHEStats} says that about 52\% of trees ($y$ value) need only 1\% of their
leaves ($x$ value) to cover 90\% of the training input. 
In general, Figure \ref{Fig:AirlineOHEStats} shows that very few leaves are needed to cover a very large fraction of inputs for the benchmark \op{airline-ohe}. 
This means that a small fraction of leaves are very likely. 
We call trees with a small number of extremely likely leaves \textbf{\emph{leaf biased}}.

On the other hand, for the benchmark \op{epsilon},
figure \ref{Fig:EpsilonStats} shows that a trees need a much larger fraction of their leaves to cover a significant fraction of the training input.
This means that most trees in \op{epsilon} are not leaf biased.

\CommentOut{
\subsubsection{More Notation}
In order to formulate the probability based tiling algorithm as an optimization problem, we define the following.
\begin{enumerate}
    \item For every leaf $l \in L$, we define $p_l$ as the probability that the leaf $l$ is reached.
    \item For each node $n \in V$, we define the absolute probability $p_v$ as
    \begin{equation}
        p_v = \begin{cases}
        p_l &\text{if $l \in L$}\\
        p_{left(v)} + p_{right(v)} &\text{otherwise}
        \end{cases}
    \end{equation}
    \item For any tree $T$, $\mathcal{C}(T)$ represents the set of all valid tilings of $T$.
    \item For every $v \in V$, we define $S_v$ as the subtree rooted at $v$.
    \item For every $v \in V$, we define $L_v$ as the set of leaves of $S_v$.
    \item For a every tile $T_i$, we define $root(T_i)$ as the node $v \in T_i$ such that $v$ has no incoming edges from any other node $u \in T_i$.
    \item For a tile $T_i$, $out(T_i) \subseteq E$ is the set of edges $(u, v)$ such that $u \in T_i$ and $v \notin T_i$.
\end{enumerate}
}

\begin{figure}
    \centering
    \includegraphics[width=\linewidth]{figures/airline-ohe.stats.train.txt.png}
    \caption{Statistical profile for airline-ohe}
    \label{Fig:AirlineOHEStats}
\end{figure}
\begin{figure}
    \centering
    \includegraphics[width=\linewidth]{figures/epsilon.stats.train.txt.png}
    \caption{Statistical profile for epsilon}
    \label{Fig:EpsilonStats}
\end{figure}
\CommentOut{
We say that an input row $r_i$ is \textbf{\emph{covered}} by a subset of leaves $L' \subseteq L$ of a tree $T$, if the leaf $l$ reached by 
walking $T$ for row $r_i$ is in $L'$. We show how different models (and even different trees within the same model) behave differently 
using models for two benchmarks, airline-ohe and epsilon. Consider the graphs shown in   
figures \ref{Fig:AirlineOHEStats} and \ref{Fig:EpsilonStats}. Each line in these graphs corresponds to a fixed fraction of the input (say $f$). 
A point on a line at coordinate $(x, y)$ means that a fraction $y$ of trees in the model could cover a fraction $f$ of all training inputs with a fraction $x$ of 
leaves. For example, the first point on the $f=0.9$ line in figure \ref{Fig:AirlineOHEStats} says that about 52\% of trees ($y$ value) need only 1\% of their
leaves ($x$ value) to cover 90\% of the training input. 
In general, Figure \ref{Fig:AirlineOHEStats} shows that very few leaves are needed to cover a very large fraction of inputs for the benchmark airline-ohe. 
This means that a small fraction of leaves are very likely. On the other hand, for the benchmark epsilon,
figure \ref{Fig:EpsilonStats} shows that a trees need a much larger fraction of their leaves to cover a significant fraction of the test input.
This means that most trees in the epsilon model are not leaf biased.
One other observation we make is that most models have some leaf biased trees while the rest of the trees have equally likely leaves.
\TODO{AP Maybe define a term for trees with roughly equally likely leaves?} We design the probability based tiling algorithm to take advantage of this property 
of decision tree ensembles. 
}
\subsubsection{The Optimization Problem}

%We assume that we are given the probabilities of each leaf node of the decision tree (these can easily be computed using the training data). For every leaf $l \in L$, we %are given the probability $p_l$ that the leaf $l$ is reached. 

Observe that the latency of one tree walk is proportional to the number of tiles that need to be evaluated to reach the leaf. It is easy to see that for a leaf biased tree, basic tilling does not optimize for this objective, it considers all leafs to be equally likely. 
 
The goal of probablistic tiling is to minimize the average inference latency, or equivalently the minimize the expected number of tiles that are evaluated to compute one tree prediction. More formally, the problem is to find a \emph{valid} (as defined in Section~\ref{sec:ValidTiling}) tiling $\mathcal{T}$ such that the following objective is minimized.
\[
    \min_{\mathcal{T} \in \mathcal{C}(T)}{\sum_{l \in L_{\Tree}} p_l.depth_{\mathcal{T}}(l)}
\]
where the minimization is over all valid tilings $\mathcal{T}$ of the tree $\Tree$, $depth_{\mathcal{T}}(l)$ is the depth of the leaf $l$ given tiling ${\mathcal{T}}$. $p_l$ is the probability of of reaching leaf $l$ as observed during training.

The above optimization problem can be solved optimally using dynamic programming. 
We leave this out in the interest of space. 
Instead, we use the simple greedy algorithm listed in algorithm \ref{Alg:GreedyTilingAlgo} to construct a valid tiling given the node probabilities\footnote{Probabilites for internal nodes can be computed from probablities for leafs by summing up the probabilities of all leafs that belong to the sub-tree rooted at the internal node. Leaf probabilities are collected during training.}.
The algorithm starts at the root and greedily keeps adding the most probable legal node to the current tile until the maximum tile size is reached.
Subsequently, the tiling procedure is recursively performed on all nodes that are destinations for edges going out of the constructed tile.

% \subsubsection{Dynamic Programming Formulation}


% For any node $v \in V$, we define
% \[
%     cost(v, \mathcal{T}) = \sum_{l \in L_v} p(l | v).depth_{\mathcal{T}}(l)
% \]
% where $\mathcal{T} \in \mathcal{C}(T_v)$.

% Then, the objective function, for the tree $T_v$, can be rewritten as 
% \[
%     opt\_cost(v) = \min_{\mathcal{T} \in \mathcal{C}(T_v)}{cost(v, \mathcal{T})}
% \]

% The objective function can then be rewritten in the following recursive form.
% \[
%     opt\_cost(v) = \min_{T_0 \in TileShapes(n_t, v)}{1 + \sum_{(n_1, n_2) \in out(T_0)} p(n_1 | v)p(n_2 | n1)opt\_cost(n_2)}
% \]
% where $TileShapes(n_t, v)$ is the set of all tile shapes of size $n_t$ with root $v$. A straight forward substitution argument shows why the solution to the subproblems (tiling all sub-trees) needs to to be optimal. The objective is now in a form that can solved using
% dynamic programming. 

% \subsubsection{Greedy Algorithm}

% Intuitively, it seems like the following greedy algorithm also gives the optimal tiling. The algorithm starts at the root and greedily keeps adding the most probable node to the current tile until the maximum tile size is reached.
\begin{algorithm}
    \caption{Greedy Probability Based Tree Tiling}
    \label{Alg:GreedyTilingAlgo}
    \begin{algorithmic}
        \Procedure{TileTree}{$\Tree = (V, E, r)$, $n_t$} 
            \If {$r \in L_{\Tree}$}
                \State \textbf{return} $\{ r \}$
            \EndIf
            \State $Tile \leftarrow \{ r \}$
            \While{$|Tile| < n_t$}
                \State $e = (u,v) \in Out(Tile)$ st $p(v)$ is max and $v \notin L$
                \If{$e = \emptyset$}
                    \State \textbf{break}
                \EndIf
                \State $Tile = Tile \cup \{ v \}$
            \EndWhile
            \State $Tiles =  \{ Tile \}$
            \For{$(u,v) \in Out(Tile)$}
                \State $Tiles \leftarrow Tiles \cup TileTree(S_v, n_t)$
            \EndFor
            \State \textbf{return} $Tiles$
        \EndProcedure
    \end{algorithmic}
\end{algorithm}

% Talk about problems with increasing number of tile shapes and only performing such tiling on skewed trees
\CommentOut{
When we tried to apply algorithm \ref{Alg:GreedyTilingAlgo} on all trees in our benchmarks, we found
that even minor variations in probability caused the tiling algorithm to generate a large 
number of tile shapes. This in turn caused a loss in performance because the large size of the 
lookup table needed (section \ref{sec:LookupTable}) caused increased L1 cache misses. In order to 
alleviate this, we only perform probability based tiling on trees that are leaf biased.
}
We find probability based tiling is only beneficial for leaf biased trees\footnote{Turns out that for trees that are not leaf biased, probability based tiling produces many more tile shapes (see Section~\ref{sec:tileShapes}) which direclty impacts the cost of \op{getChildTile} making it more expensive than basic tiling.}.  Recall that   
a tree to be leaf biased if a small fraction of leaves, say $\alpha$, can cover a large fraction of training inputs, say $\beta$.
We only perform probability based tiling on trees with thresholds $\alpha=0.05$ and $\beta=0.9$ and fall back to uniform tiling otherwise. 




\subsection{Tree ordering}
\label{sec:treeorder}	
%Another objective at the HIR is to reduce the number of trees that need to be specialized by the lower levels.   
\TODO{kr : Not sure if i got this para right}
Specializing the code for each tree in a model comes at a cost. First the code generator needs to generate different code for different trees potentially increasing the size of the generated code. Second some cross tree optimizations (applied at the lower levels of abstraction) like tree walk interleaving require that the multiple trees share identical code. 

In order to handle
this, \Treebeard{} pads trees with dummy nodes to make them balanced and then sorts the trees by their depth, so that trees of same depth can share code. Padding is only done for almost balanced trees (as generated by basic tiling), this is ensured by only adding up to a fixed fraction of dummy nodes.  
\TODO{Kr : Can we add a threshold say 10\%?}
%This ensures that all trees with equal depth are grouped together.
 Once this is done, 
the loop over the trees is fissed so that each of the resulting loops only walks trees of a single depth. Consider for example a 
forest with 4 trees $T_1$, $T_2$, $T_3$, and $T_4$ in that order. Further, assume that $T_1$ and $T_4$ have depth 2 while $T_2$ and $T_3$
have depth 3. First, Treebeard reorders the trees to be in the order $T_1$, $T_4$, $T_2$, $T_3$. Then, the loop over the trees is fissed
as shown in the following listing.

% loop transformations for basic tiling (splitting) 
\begin{lstlisting}{style=c++}
  forest = ensemble(...)
  for i = 0 to batchSize step 1 {
    prediction = 0
    for t = 0 to 2 step 1 {
      tree = getTree(forest, t) 
      node = getRoot(tree)
      node = traverseTreeTile(tree, node, rows[i])
      treePrediction = getLeafValue(tree, node)
      prediction = prediction + treePrediction
    }
    for t = 2 to 4 step 1 {
      tree = getTree(forest, t) 
      node = getRoot(tree)
      node = traverseTreeTile(tree, node, rows[i])
      node = traverseTreeTile(tree, node, rows[i])
      treePrediction = getLeafValue(tree, node)
      prediction = prediction + treePrediction
    }
    predictions[i] = prediction
  }  
\end{lstlisting}


\section{Loop optimizations at mid level}
\subsection{Tree Walk Interleaving}
While tiling and subsequent vectorization gives significant performance gains, profiling 
the generated code showed that true dependencies between instructions were still causing
a significant number of stall cycles. In order to fill these stall cycles, Treebeard does 
an unroll and jam on the inner most loops of the loop nest. This transformation has 
the effect of walking multiple tree and input row pairs in an interleaved fashion. 
The dependency stalls can be hidden by scheduling instructions from independent tree walks 
in the stall cycles. 

This optimization is performed across both the mid-level IR and low-level IR. Loops on which 
to perform unroll and jam are identified in the mid-level IR. This information is communicated to the 
lowering passes by annotating these tree walk operations as descibred in section \ref{sec:Overview}. The lowering 
passes that transform the mid-level IR to low-level IR interleave operations across independent
tree walks.

The following listing shows the mid-level IR when the inner loop over the input rows is unrolled 
by a factor of 2 and the two resulting tree walks are jammed together.

\begin{lstlisting}[style=c++]
  forest = ensemble(...)
  for t = 0 to numTrees step 1 {
    for i = 0 to batchSize step 2 {
      tree = getTree(forest, t)
      prediction1, prediction2 = InterleavedWalk((tree, rows[i]), (tree, rows[i+1]))
    }
  }
\end{lstlisting}

When lowered to low-level IR, the operations to traverse each of the tree, input row pairs 
(the arguments to the \texttt{InterleavedWalk}) are interleaved. One step of the interleaved 
walk is listed below. 
\begin{lstlisting}[style=c++]
  // ... 
  n1 = n2 = getRoot(tree)
  // ...
  threshold1 = loadThresholds(n1)
  threshold2 = loadThresholds(n2)
  featureIndex1 = loadFeatureIndices(n1)
  featureIndex2 = loadFeatureIndices(n2)
  feature1 = rows[i][featureIndex1]
  feature2 = rows[i][featureIndex2]
  pred1 = feature1 < threshold1
  pred2 = feature2 < threshold2
  n1 = getChildNode(n1, pred1)
  n2 = getChildNode(n2, pred2)
  // ...
\end{lstlisting}
These transformations are fairly general and are not aware of the in memory representation of the model. Therefore, they 
are reusable across different in memory representations - the ones that are currently built into Treebeard or ones that 
maybe added in the future.
\TODO{AP I feel the way this section is currently written makes the optimization seem extremely trivial. Is there a different 
way to present it?}
\subsection{Loop peeling and walk unrolling}
\Treebeard{} splits the loop that performs a tree walk into two parts. As it is aware of the depth of the first leaf in a tree, it peels and introduces a \op{prologue} loop that walks down the tree a constant number of steps (up to first leaf) and then performs the rest of the tree walk in a separate loop. Further \Treebeard{} unrolls the prologue completely, avoiding all traversal induced branching in it.

Note that for trees where \Treebeard{} has already padded and balanced the tree (Section~\ref{sec:treeorder}), walk unrolling completely avoids all traversal induced spills.
\TODO{kr: Add example?}
\input{parallel}
\section{Low level optimizations}

\subsection{Vectorization}
\label{sec:Vectorization}
Vectorization performed by Treebeard is enabled by the tiling transformations described in section \ref{sec:Tiling}. 
When the low level IR is translated to LLVM IR, Treebeard generates LLVM instructions that operate on the threholds and feature indices 
of nodes within a tile in a vector fashion. Therefore, thresholds and feature indices are loaded using vector loads and predicates are 
evaluated using vector comparisons. These vector LLVM IR instructions are then converted to vector instructions in the target ISA by 
the LLVM JIT.

The below listing shows some of the details of a vectorized tree walk. 
\begin{lstlisting}[style=c++]
  // A lookup table that determines the child index of
  // the next tile given the tile shape and the outcome
  // of the vector comparison on the current tile
  int16_t LUT[NUM_TILE_SHAPES, pow(2, TileSize)]
  
  ResultType Prediction_Function(...) {
    // ...
    Node n = getRoot(tree)
    while (isLeaf(tree, n)==false) do {
      thresholds = loadThresholds(tree, n)
      featureIndices = loadFeatureIndices(tree, n)
      // Gather required feature from the current row
      features = rows[i][featureIndices]
      // Vector comparison of features and thresholds
      comparison = features < thresholds
      
      // Pack bits in comparison vector into an integer
      comparisonIndex = combineBitsIntoInt(comparison)
      
      // Get child index of tile we need to move to
      tileShape = loadTileShape(tree, n)
      childIndex = LUT[tileShapeID, comparisonIndex]
      
      // Move to the correct child of the current node
      n = getChildNode(tree, n, childIndex) 
    }
    ThresholdType prediction = getLeafValue(n)
    // ...
  }  
\end{lstlisting}
To evaluate the current tile, the vector of thresholds is first loaded (\texttt{loadThresholds}). This vector contains the thresholds of all nodes in the tile. Then, the features required for comparison are gathered into a vector (lines 11 and 13). The feature vector is compared to the threshold vector and the child tile to move to is determined (lines 15 to 25). More details about tile shapes and the look up table are presented in subsequent sections.

\subsubsection{Tile Shapes}
Informally, the \textbf{\emph{tile shape}} is the shape of the region that encloses all nodes in a tile in a diagram of the decision tree. More formally, for a tile size $n_t$, each unique legal binary tree containing $n_t$ nodes (nodes being indistinguishable) corresponds to a tile shape.

Figure \ref{Fig:TileSize3Shapes} enumerates all tile shapes with a tile size of 3. There are a total of 5 tile shapes with size 3. The number of tile shapes with a tile size $n_t$, denoted by $NTS(n_t)$ is given by the following equation. 

\begin{equation}
  NTS(n) = \sum_{k=0}^{n-1} NTS(k) \times NTS(n-k-1)
\end{equation}

where $NTS(0) = NTS(1) = 1$.

\begin{figure}
  \centering
  \includegraphics[width=\linewidth]{figures/TileShapes_Size3.PNG}
  \caption{All possible tile shapes with tile size $n_t=3$}
  \label{Fig:TileSize3Shapes}
\end{figure}

\subsubsection{Tile Shapes and Decision Tree Inference}
\label{Sec:TileShapesAndDecisionTreeInference}
Treebeard uses vector instructions to accelerate decision tree walks. Vector instructions are used to evaluate the predicates of all the nodes in a tile simultaneously. However, once the predicates of all the nodes in the tile are evaluated, computing the next tile to move to, given the outcome of the comparison depends on the tile shape of the current tile. To illustrate this problem, consider the case of the tiles of size 3 shown in figure \ref{Fig:TileTraversalTileSize3}. 
\begin{figure}
  \centering
  \includegraphics[width=\linewidth]{figures/TileTraversal_Size3.PNG}
  \caption{Example tile traversals with tile size $n_t=3$}
  \label{Fig:TileTraversalTileSize3}
\end{figure}
The diagram shows 3 of the 5 possible tile shapes for a tile size of 3. The nodes drawn in black are members of the tile $t_1$. The nodes in blue are the entry nodes of the children tiles of $t_1$. \TODO{Define entry nodes}

% To traverse a tile on an input row, first, the predicate of each node in the tile is computed. Subsequently, we need to determine which of the child tiles to move to next. Note that a true predicte (bit value 1) on a node implies a move to the left child and a false predicate (bit value 0) implies a move to the right child.

In the diagram, the bit strings (written in red) show which child we need to move to given the outcomes of the comparison. The bits represent the comparison outcomes of nodes and are in the order of the nodes in the tile -- marked 0, 1 and 2 in the diagram, i.e., the MSB is the predicate outcome of node 0 and the LSB the predicate outcome of node 2. For example, for the first tile shape, if the predicates of all nodes are true (i.e. the comparison outcome is 111), the next node to evaluate is $a$. 
% However, if the predicate of node 1 is false, then we need to move to $d$ regardless of the outcomes of nodes 2 and 3.
It is easy to see from the diagram that, depending on the tile shape, the same predicate outcomes can mean moving to different children. For example, for the outcome "011", the next tile is the 4th child (node $d$) for the first two tile shapes while it is the 3rd child for the other tile shape (node $c$).

\subsubsection{Lookup Table}
\label{sec:LookupTable}
A lookup table (LUT) is used to solve the problem described in section \ref{Sec:TileShapesAndDecisionTreeInference}, i.e. given the outcome of the comparisons of all nodes in a tile, determine the child tile we should evaluate next. The LUT is indexed by the tile shape and the comparison outcome. Formally, the LUT is a map.
\[
LUT : (TileShape, < Boolean \times n_t >) \rightarrow [0, n_t] \subset \mathbb{N}
\]

where $n_t$ is the tile size, $< Boolean \times n_t >$ is a vector of $n_t$ booleans. The value returned by the LUT is the index of the child of the current tile that should be evaluated next. For example, if we are evaluating the first tile $t$ in figure \ref{Fig:TileTraversalTileSize3}, and the result of the comparison is 110, then $LUT(TileShape(t), 110)=1$ since the tile we need to evaluate next is the tile with node $b$, which is the second child of the current tile.

In order to realize this LUT in generated code, Treebeard associates a non-negative integer ID with every unique tile shape of the given tile size. The result of the comparison, a vector of booleans, can be interpreted as a 64-bit integer. Therefore, the LUT can be implemented as a 2 dimensional array.
% \begin{lstlisting}{style=c++}
%   int16_t LUT[NTS(n_t), pow(2, n_t)]  
% \end{lstlisting}
Treebeard computes the values in the LUT statically as the tile size is a compile time constant.
\TODO{AP What comes after subsubsection?}

\subsection{In Memory Representation of Tiled Trees}
Treebeard currently has two in memory representations for tiled trees - an array based representation and a sparse representation. Both representations use an array of structs to represent all tiles of the model. 

\subsubsection{Array Based Representation}
\label{sec:ArrayBased}
Each tree in the model is represented as an array of tiles using the standard representation of trees as arrays. The root node is at index 0 and for a node at index $n$ in the array, the index of its $i^{th}$ child is given by $(n_t + 1) n + (i + 1)$ (nodes in the tree of tiles have $n_t + 1$ children). A tile is represented by an object of the following struct.
\begin{lstlisting}{style=c++}
  struct Tile {
    // A vector of TileSize elements
    <ThresholdType x TileSize> thresholds; 
    <FeatureIndexType x TileSize> featureIndices;
    // Integer that identifies the tile shape
    TileShapeIDType tileShapeID; 
  };  
\end{lstlisting}
\TODO{AP Is this level of detail really needed? Also, the vector type notation needs to be introduced somewhere.}
Even though this representation is simple and efficient for small models, the memory required for bigger models is very large. 
%The memory footprint is up to 20X that of the scalar representation.
This memory bloat causes performance problems because the span of the L1 TLB is not sufficient to efficiently translate 
addresses for the whole model. Storing leaves as full tiles (even though leaves just have to represent one value) and the
empty space introduced due to the array based representation of trees that are not complete account for most
of the increase.
%The sparse representation described next tries to address these issues. 

\subsubsection{Sparse Representation}
\label{sec:SparseRep}
The sparse representation tries to address the large memory footprint of the array based representation by doing the following.

\begin{itemize}
  \item We add a child pointer to each tile to eliminate the wasted space in the array representation. This points to the first child of the tile. All children of a tile are stored contiguously.
  \item Leaves are stored as a separate array of scalar values. Across all our benchmarks, after tiling a large fraction of 
  leaves are such that all their siblings are also leaves. Such leaves are directly moved into the leaves array. For leaves
  for which this property does not hold, an extra ``hop'' is added by making the original leaf tile a normal tile. All its
  children are made leaves with the same value as the original leaf.
\end{itemize}

\begin{figure}
  \centering
  \includegraphics[width=\linewidth]{figures/SparseRep_TileSize3.PNG}
  \caption{Sparse representation with tile size $n_t=3$}
  \label{Fig:SparseRep}
\end{figure}

Figure \ref{Fig:SparseRep} shows some of the details of the sparse representation.
% The tree on the left of the diagram is the actual decision tree with the nodes grouped into tiles $t_1$ and $t_2$. The tree on the right is the tree of tiles.
The arrays depicted below show how the tree is represented in memory. The first array ($\texttt{tree}$) is an array of tiles 
and has 5 elements. Each element of the array represents a single tile and has the thresholds of the nodes, the feature
indices, a tile shape ID and a pointer to the first child (shown explicitly in red). 

As a specific example, consider the tile $t_1$. The tile has four children -- $l_1$, $l_2$, $l_3$ and $t_2$ in that order (left to right). These tiles are stored contiguously in the $\texttt{tree}$ array and a pointer to the first of these, $l_1$ is stored in the tile $t_1$ (the index 1 is stored in the tile $t_1$ as shown). 

Now consider the tile $t_2$. Since all children of the tile $t_2$ are leaves, they are all moved into the $\texttt{leaves}$ array.
To store a pointer into the $\texttt{leaves}$ array, we add $\texttt{len(tree)}$ to the element index in the $\texttt{leaves}$ array.
The tile $t_2$'s child is the element at index 12 of the $\texttt{leaves}$ array. Therefore, the index $12 + 5 = 17$ is stored in 
the tile $t_2$. Any index $i$ that is greater than the length of the $\texttt{tree}$ array is regarded as an index into the
$\texttt{leaves}$ array. The index into the $\texttt{leaves}$ array is $i - \texttt{len(tree)}$.

The other aspect of the representation is that an extra hop is added for the leaves $l_1$, $l_2$ and $l_3$ in order to simplify
code generation. This enforces the invariant that all leaves are stored in the leaves array and  simplifies checking whether
we've reached a leaf. Therefore, 4 new leaves are added as children for each of the original leaves $l_1$, $l_2$ and $l_3$. 
Each of these 12 newly added leaves has the same value as its parent. These are the first 12 elements of the $\texttt{leaves}$ array.

Even though we currently have implementations of the two representations detailed in sections \ref{sec:ArrayBased} 
and \ref{sec:SparseRep}, support for other representations is not hard to add. All optimizing passes that work on 
the high level and mid level IR will continue to work as is. Programmers need only provide new lowering passes for
a few operations in the low level IR.

\subsubsection{Code Generation for Probability Based Tiling}
As probability based tiling pulls the most probable leaves of a decision tree nearest the root, it poses 
some implementation challenges. By design, the tiling process makes the tree of tiles 
imbalanced. The array based representation (section \ref{sec:ArrayBased})
cannot be used because of the memory footprint increase (a large part of the tree is empty, but would need to be allocated).
On the other hand, the sparse representation in section \ref{sec:SparseRep} adds 
an extra hop for leaves that have non-leaf siblings. But this would mean that we add extra hops for 
the most probable leaves after probability based tiling which defeats the optimization.

We address these challenges using a code generation strategy. Treebeard peels 
the tree walk and specializes the leaf checks at higher levels to avoid the extra hop. Currently, 
we determine the maximum depth of leaves needed to cover 90 percent of the inputs and peel the tree 
walk by as many iterations. For example, consider the case where leaves until depth 2 are needed to 
cover 90 percent of the training input. Then, Treebeard generates the following IR. 

\begin{lstlisting}{style=c++}
    // ...
    tree = getTree(forest, t)
    node = getRoot(tree)
    node = traverseTreeTile(tree, node, rows[i])
    if (isLeafTile(node)) {
        treePrediction = getLeafValue(tree, node)
    } else {    
        node = traverseTreeTile(tree, node, rows[i])
        if (isLeafTile(node)) {
            treePrediction = getLeafValue(tree, node)
        } else {    
            // Loop based traversal 
        }
    }
    treePrediction = getLeafValue(tree, node)
    // ...
\end{lstlisting}

The if statements check whether a node is a leaf tile and hence avoid the extra hop. 
%The memory 
%requirement is also not increased because only a small fraction of leaves are represented as full tiles.
While walk peeling is used to improve the performance of probability based tiling by specializing leaf tests,
the transformation is by itself general and can be used in different contexts. For example, it could be used 
to elide leaf checks until a depth $d$ is reached if we know all leaves are at a depth greater than $d$. 

One other issue that the code generator needs to handle is that walks of different trees in the same ensemble may 
need to be peeled to different depths. A strategy similar to what is used for uniform tiling is used to handle this.
Trees are reordered so that all trees 
with equal peeling depth are grouped together and the loops in the IR are fissed so that tree walks 
for these groups of trees can be specialized differently.
% This is very similar to the code generation strategy used for uniform tiling.




\CommentOut{
\begin{abstract}

  This document is intended to serve as a sample for submissions to the 55\textsuperscript{th} IEEE/ACM International Symposium on Microarchitecture\textsuperscript{\textregistered} (MICRO 2022). We provide some guidelines that authors should follow when submitting papers to the conference.  This format is derived from the ACM sig-alternate.cls file, and is used with an objective of keeping the submission version similar to the camera-ready version.

\end{abstract}

\section{Introduction}

This document provides instructions for submitting papers to the 55\textsuperscript{th} IEEE/ACM International Symposium on Microarchitecture\textsuperscript{\textregistered} (MICRO 2022).  In an effort to respect the efforts of reviewers and in the interest of fairness to all prospective authors, we request that all submissions to MICRO 2022 follow the formatting and submission rules detailed below. Submissions that violate these instructions may not be reviewed, at the discretion of the program chairs, in order to maintain a review process that is fair to all potential authors. 

This document is itself formatted using the MICRO 2022 submission format. The content of this document mirrors that of the submission instructions that appear on the conference website. All questions regarding paper formatting and submission should be directed to the program chairs.

\subsection{Format Highlights}
\begin{itemize}
\item Paper must be submitted in printable PDF format.
\item Text must be in a minimum 10pt font, see Table~\ref{table:formatting}.
\item Papers must be at most 11 pages, not including references.
\item No page limit for references.
\item Each reference must specify {\em all} authors (no {\em et al.}).
\item Author anonymity must be fully preserved, including in any referenced artifacts (e.g., GitHub repository).
\end{itemize}

\subsection{Paper Evaluation Objectives} 
The committee will make every effort to judge each submitted paper on its own merits. There will be no target acceptance rate. We expect to accept a wide range of papers with appropriate expectations for evaluation---while papers that build on significant past work with strong evaluations are valuable, papers that open new areas with less rigorous evaluation are equally welcome and especially encouraged.

\section{Paper Preparation Instructions}

\subsection{Paper Formatting}

Papers must be submitted in printable PDF format and should contain a {\em maximum of 11 pages} of single-spaced two-column text, {\bf not including references}.  You may include any number of pages for references, but see below for more instructions.  If you are using \LaTeX~\cite{lamport94} to typeset your paper, then we suggest that you use the template here: \href{https://www.microarch.org/micro55/submit/micro55-latex-template.zip}{\LaTeX~Template}. This document was prepared with that template. Note that the template and sample paper may render slightly differently on different \LaTeX~engines, due to typesetting changes between versions. If you use a different software package to typeset your paper, then please adhere to the guidelines given in Table~\ref{table:formatting}. 

\begin{scriptsize}
\begin{table}[h!]
  \centering
  \begin{tabular}{|l|l|}
    \hline
    \textbf{Field} & \textbf{Value}\\
    \hline
    \hline
    File format & PDF \\
    \hline
    Page limit & 11 pages, {\bf not including}\\
               & {\bf references}\\
    \hline
    Paper size & US Letter 8.5in $\times$ 11in\\
    \hline
    Top margin & 1in\\
    \hline
    Bottom margin & 1in\\
    \hline
    Left margin & 0.75in\\
    \hline
    Right margin & 0.75in\\
    \hline
    Body & 2-column, single-spaced\\
    \hline
    Space between columns & 0.25in\\
    \hline
    Line spacing (leading) & 11pt \\
    \hline
    Body font & 10pt, Times\\
    \hline
    Abstract font & 10pt, Times\\
    \hline
    Section heading font & 12pt, bold\\
    \hline
    Subsection heading font & 10pt, bold\\
    \hline
    Caption font & 9pt (minimum), bold\\
    \hline
    References & 8pt, no page limit, list \\
               & all authors' names\\
    \hline
  \end{tabular}
  \caption{Formatting guidelines for submission.}
  \label{table:formatting}
\end{table}
\end{scriptsize}

{\em Please ensure that you include page numbers with your submission}. This makes it easier for the reviewers to refer to different parts of your paper when they provide comments. Please ensure that your submission has a banner at the top of the title page, similar to this document, which contains the submission number and the notice of confidentiality.  If using the template, just replace XXX with your submission number.

\subsection{Content}

Reviewing will be {\em double blind} (no author list); therefore, please do not include any author names on any submitted documents except in the space provided on the submission form.  You must also ensure that the metadata included in the PDF does not give away the authors. You must fully anonymize any links to artifacts (e.g., GitHub repository) and remove any links to artifacts that cannot be fully anonymized. {\bf Papers that violate the anonymization policy may be rejected without review} at the discretion of the program chairs.


If you are improving upon your prior work, refer to your prior work in the third person and include a full citation for the work in the bibliography.  For example, if you are building on {\em your own} prior work in the papers \cite{nicepaper1,nicepaper2,nicepaper3}, you would say something like: "While the authors of \cite{nicepaper1,nicepaper2,nicepaper3} did X, Y, and Z, this paper additionally does W, and is therefore much better."  Do NOT omit or anonymize references for blind review.  There is one exception to this for your own prior work that appeared in IEEE CAL, arXiv, workshops without archived proceedings, etc.\, as discussed later in this document.

\noindent\textbf{Figures and Tables:} Ensure that the figures and tables are legible.  Please also ensure that you refer to your figures in the main text.  Many reviewers print the papers in gray-scale. Therefore, if you use colors for your figures, ensure that the different colors are highly distinguishable in gray-scale.

\noindent\textbf{References:}  There is no length limit for references. {\em Each reference must explicitly list all authors of the paper.  Papers not meeting this requirement will be rejected.} Authors of NSF proposals should be familiar with this requirement. Knowing all authors of related work will help find the best reviewers. Since there is no length limit for the number of pages used for references, there is no need to save space here.

\section{Paper Submission Instructions}

\subsection{Guidelines for Determining Authorship}
IEEE guidelines dictate that authorship should be based on a {\em substantial intellectual contribution}. It is assumed that all authors have had a significant role in the creation of an article that bears their names. In particular, the authorship credit must be reserved only for individuals who have met each of the following conditions:

\begin{enumerate}
\item Made a significant intellectual contribution to the theoretical development, system or experimental design, prototype development, and/or the analysis and interpretation of data associated with the work contained in the article;

\item Contributed to drafting the article or reviewing and/or revising it for intellectual content; and

\item Approved the final version of the article as accepted for publication, including references.
\end{enumerate}

A detailed description of the IEEE authorship guidelines and responsibilities is available \href{https://www.ieee.org/publications_standards/publications/rights/Section821.html}{here}. Per these guidelines, it is not acceptable to award {\em honorary } authorship or {\em gift} authorship. Please keep these guidelines in mind while determining the author list of your paper.

\subsection{Declaring Authors}
Declare all the authors of the paper upfront. Addition/removal of authors once the paper is accepted will have to be approved by the program chairs, since it potentially undermines the goal of eliminating conflicts for reviewer assignment.

\subsection{Areas and Topics}
Authors should indicate these areas on the submission form as well as specific topics covered by the paper for optimal reviewer match. If you are unsure whether your paper falls within the scope of MICRO, please check with the program chairs -- MICRO is a broad, multidisciplinary conference and encourages new topics.

\subsection{Revision of Previously-Reviewed\\ Manuscript}
If the manuscript has been previously reviewed and rejected and is now being submitted to MICRO, the authors have an option of providing a letter explaining how the paper has been revised for this current submission. We expect this revision information to improve both the submission and the review process. Authors choosing to provide such a letter have control over who has access to it by specifying one of the following options:

\begin{enumerate}
\item Shared with all reviewers of the paper 
\item Shared with reviewers who declare that they reviewed a prior version and who request the revision information
\item Not shared with any PC member but available to the program chairs
\end{enumerate}

We encourage you to keep this letter concise and optionally append additional information, such as a version of the paper that highlights the differences or any other material of your choice.

\subsection{Declaring Conflicts of Interest}
Authors must register all their conflicts for their paper submission. Conflicts are needed to ensure appropriate assignment of reviewers. {\bf If a paper is found to have an undeclared conflict that causes a problem OR if a paper is found to declare false conflicts in order to abuse or ``game'' the review system, the paper may be rejected without review.} We use the following conflict of interest guidelines for determining the conflict period for MICRO 2022.  Please declare a conflict of interest (COI) with the following people for any author of your paper:

\begin{enumerate}
\item Your Ph.D. advisor(s), post-doctoral advisor(s), Ph.D. students,
      and post-doctoral advisees, forever.
\item Family relations by blood or marriage, or their equivalent,
      forever (if they might be potential reviewers).
\item People with whom you have collaborated in the last FOUR years, including:
  \begin{itemize}
  \item co-authors of accepted/rejected/pending papers
  \item co-PIs on accepted/rejected/pending grant proposals
  \end{itemize}
\item Ongoing collaboration that has not yet resulted in a paper or proposal submission. Justification may be queried.
\item When there is a direct funding relationship between an author and the potential reviewer (e.g., the reviewer is a sponsor of an author's research on behalf of his/her company or vice versa) in the last FOUR years.
\item People (including students) who shared your primary institution(s) in the last FOUR years.
\item Other relationships, such as close personal friendship, that you think might tend
to affect your judgment or be seen as doing so by a reasonable person familiar
with the relationship.
\end{enumerate}

We would also like to emphasize that the following scenarios do {\em not} constitute a conflict:
\begin{enumerate}
\item Authors of previously-published, closely related work on that basis alone.
\item ``Service'' collaborations such as co-authoring a report for a professional organization, serving on a program committee, or co-presenting tutorials.
\item Co-authoring a paper that is a compendium of various projects with no true collaboration among the projects.
\item People who work on topics similar to or related to those in your papers.
\item Collaborators on large funded projects where there is no close collaboration and no joint benefit in the paper being accepted.
\end{enumerate}

We hope to draw most reviewers from the program committee, but others
from the community may also write reviews. {\bf Please declare all your conflicts (not just restricted to the PC).} When in doubt, please contact the program chairs.

%Please note that all paper submissions require all authors to electronically sign a statement confirming their best effort to accurately identify potential reviewers with a conflict of interest, and importantly also {\bf assuring that each author will make no explicit attempt to directly or indirectly influence any reviewer opinion or decision about the submitted paper}. Importantly, we do not consider technical discussion of a paper's content or any other sharing of content from the paper to violate the above policy. 


\subsection{Concurrent Submissions and Workshops}

By submitting a manuscript to MICRO 2022, the authors guarantee that the manuscript has not been previously published or accepted for publication in a substantially similar form in any conference, journal, or the archived proceedings of a workshop (e.g., in the ACM/IEEE digital library) -- see exceptions below. The authors also guarantee that no paper that contains significant overlap with the contributions of the submitted paper will be under review for any other conference or journal or an archived proceedings of a workshop during the MICRO 2022 review period. Violation of any of these conditions will lead to rejection.

The only exceptions to the above rules are for the authors' own papers in (1) workshops without archived proceedings such as in the ACM/IEEE digital library (or where the authors chose not to have their paper appear in the archived proceedings), or (2) venues such as IEEE CAL or arXiv where there is an explicit policy that such publication does not preclude longer conference submissions.  In all such cases, the submitted manuscript may ignore the above work to preserve author anonymity. This information must, however, be provided on the submission form -- the program chairs will make this information available to reviewers if it becomes necessary to ensure a fair review.  As always, if you are in doubt, it is best to contact the program chairs.


Finally, the ACM/IEEE Plagiarism Policy (\href{http://www.acm.org/publications/policies/plagiarism_policy}{here} and \href{https://www.ieee.org/publications_standards/publications/rights/plagiarism.html}{here}) covers a range of ethical issues concerning the misrepresentation of other works or one's own work.


\section{Ethics}

\begin{enumerate}
\item Authors must abide by the ACM code of ethics and the IEEE code of ethics
\item Authors must not contact reviewers or PC members about any submission, including their own. This includes attempting to sway a reviewer, requesting information about any aspect of the reviewing process, and/or asking about the outcome of a submission. Similarly, authors are not allowed to ask another party to contact the reviewers on their behalf.
\item Authors must not disclose the content of reviews for their paper publicly (e.g., on social media)  before the results are announced. 
\item Authors must report any allegations of submission or reviewing misconduct to the program chairs. The only exception is if the complaint is about the program chairs; in this case, the Steering Committee should be contacted. 
\end{enumerate}


\section*{ACKNOWLEDGMENTS}
This document is derived from previous conferences, in particular MICRO 2013, ASPLOS 2015, MICRO 2015, MICRO 2016, MICRO 2017, MICRO
2018, MICRO 2019, MICRO 2020 and MICRO 2021, as well as SIGARCH/TCCA's Recommended Best Practices for the Conference Reviewing Process.
}

%%%%%%% -- PAPER CONTENT ENDS -- %%%%%%%%


%%%%%%%%% -- BIB STYLE AND FILE -- %%%%%%%%
% \bibliographystyle{IEEEtranS}
% \bibliography{refs}
%%%%%%%%%%%%%%%%%%%%%%%%%%%%%%%%%%%%

\end{document}
