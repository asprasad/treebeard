\section{Related Work}
\label{Sec:Related}
% While several optimization strategies for decision tree based models have been 
% studied in the literature, to the best of our knowledge, no systems that are 
% capable of exploring the full optimization space exist.
This section discusses prior work related to \Treebeard{}.

\emph{Decision Tree Inference Systems:} 
Tahoe~\cite{Tahoe} is a library-based system that picks 
between four predefined strategies to implement decision tree inference on GPUs.
% Also, \Treebeard{} generates code that is specific to 
% a particular model, specializing both the parallelism (by deciding the thread block 
% structure on a per model basis) and the kernel code itself by 
% performing optimizations like tree walk unrolling and interleaving.
% In contrast, Tahoe 
% uses a library-based approach, and cannot generate code tailored to a model 
% like \Treebeard{} does.
RAPIDS FIL\cite{FIL}, the most widely used GPU library for decision tree inference
% implements some heuristics to pick a good configuration for every model, 
% but these techniques are limited and the library uses a single strategy 
% and in-memory representation for all models. 
and XGBoost's\cite{XGBoost} GPU library\cite{XGBGPU} 
use a single strategy and in-memory representation.
In comparison, \Treebeard{} explores a much larger set of implementation options and 
in-memory representations for models and picks the best.
% In contrast, \Treebeard{} is a compiler that can explore a much larger
% optimization space and can generate code that is tailored to a specific model.
As shown in Section \ref{sec:results}, \Treebeard{} outperforms these systems
by a significant margin.
% Finally, all of Tahoe's optimizations are designed to address GPU specific problems. 
% For example, it uses an elaborate and opaque heuristic based on locality sensitive hashing 
% to coalesce accesses to the GPU global memory. We believe that the tree tiling
% infrastructure described in this paper will allow us to deterministically 
% coalesce accesses to GPU global memory.
%As described previously, Hummingbird\cite{Hummingbird} uses tensor operations to perform decision tree inference. 
% While compiling decision tree inference to GPUs presents a distinct set of challenges,
% we believe, we can achieve this while reusing much of \Treebeard{}'s current infrastructure.
% Specifically, we expect that most of the HIR and MIR
% optimizations described in this paper will carry over directly while LIR 
% optimizations and in-memory representations will need to be retargeted.
% This can be the subject of a separate future work.

Some compilers for decision tree ensembles have been proposed in the 
literature \cite{Treelite, Treebeard, Hummingbird}. \TreebeardOLD{} and Treelite
exclusively target CPUs and all their optimizations are designed purely for 
performance on CPUs. 
Treelite\cite{Treelite} is a model compiler that only  
generates \op{if-else} code for each tree in the model. 
\TreebeardOLD{} is the work most closely related to \Treebeard{}. While we 
build on top of \TreebeardOLD{}, \Treebeard{} is a significant enhancement 
over \TreebeardOLD{}. 
Hummingbird\cite{Hummingbird} is a compiler that compiles traditional ML models
to tensor operations, thereby enabling them to be run using tensor-based frameworks like
PyTorch\cite{NEURIPS2019_9015} on CPUs and GPUs. As reported in their paper, Hummingbird's performance
on decision tree models is comparable to that of RAPIDS on GPUs. On CPU, \TreebeardOLD{} 
is faster~\cite{Treebeard}.
% While it can target both CPUs
% and GPUs, its performance has been shown to be lower than that of other frameworks~\cite{Treebeard}.

On CPUs, XGBoost\cite{XGBoost}, LightGBM\cite{LightGBM} and
scikit-learn\cite{Sklearn} are extremely popular.
A recent paper~\cite{PACTVanLunteren} describes an adaptive mechanism 
to pick one of a few predefined parallelization and vectorization strategies.
Other systems that hide dependency stalls by interleaving tree walks\cite{VPred},
implement optimized algorithms for tree inference\cite{QuickScorer, QuickScorer1}
and improve cache performance of decision tree ensembles on CPUs\cite{CacheConscious1, CacheConscious2}
have been proposed.
Some systems have been proposed to parallelize decision tree training 
on CPUs and GPUs\cite{Jansson2014gpuRFAG, Nasridinov2013DecisionTC}.
However, none of these systems provide portable performance across different target machines.
While the specifics may vary, \Treebeard{} supports retargetable compiler optimizations that 
achieve similar results. % as techniques in these systems.

\emph{Other Systems and Techniques:} Ren et. al.~\cite{PortableVM} design an
intermediate language and a virtual machine to enable vector execution of decision tree
inference. However, this virtual machine is implemented by 
hand on different target processors.
%This is clearly more expensive than \Treebeard{}'s approach.
% Additionally, even though they perform layout optimizations, their system does
% not perform any model specific optimizations.
Jo et. al.\cite{MilindTreeVectorization} describe code transformations and runtime 
techniques that vectorize tree-based applications but these 
optimizations are not specific to decision trees.
% Both these works vectorize tree walks by performing different tree walks on each
% vector lane. The main issue with this approach is the divergence of tree walks.
% Another issue is that memory accesses are \op{gather}'s rather than vector
% loads. \Treebeard{}'s approach to vectorization solves both these issues. Also,
% these approaches are not precluded by \Treebeard{}'s tiling based vectorization.
% Multiple tiled tree walks can be combined into a single vectorized walk. We
% leave an exploration of this to future work.  FAST\cite{FAST} is a system that
% accelerates tree structured index search on CPUs and GPUs. FAST defines a layout
% for the index tree that enables vectorization of the tree walk. FAST uses a tree
% tiling approach to vectorize tree walks. However, FAST only uses a single
% triangular tile shape. \Treebeard{}'s basic tiling algorithm is a generalization
% of the tiling used in FAST. If given a perfectly balanced tree, the basic tiling
% algorithm would return exactly the tiling used by FAST.  Also, the tree walks in
% FAST are hand coded using intrinsics on the CPU and CUDA on the GPU and
% therefore need repeated effort to implement on each target. 
Inspector-executor systems \cite{TaoOfParallelism,HybridCPUGPU}  
parallelize tree walks but are not a good fit 
for decision tree inference as the individual node predicates are 
simple and the overhead of an inspector-executor system would be prohibitive.

\emph{Code Generation Systems for Other Domains:}
% Several optimizing compilers and code generation techniques have been developed 
% for other domains. 
TVM\cite{TVM}, Tiramisu\cite{Tiramisu}, and Tensor 
Comprehensions\cite{TensorComprehensions} are optimizing compilers 
for DNNs that can target a variety of processors. Similarly, 
Halide\cite{Halide} is a DSL and compiler primarily designed for 
image processing applications. The concept of separating the computation 
from the schedule was effectively utilized by Halide and has since been adopted 
by several other systems~\cite{TVM,Tiramisu,GraphIt}. 
% However, to 
% the best of our knowledge, \Treebeard{} is the first system to design
% a scheduling language for decision tree inference optimization
% and to build a system capable of state-of-the-art performance 
% across different processors.
% {\sc {Cortex}} transforms 
% recursive computations in DNNs to loop based computations and optimizes them.
% However, the model properties {\sc {Cortex}} assumes to optimize
% generated code do not hold in the context of decision tree inference. For example, 
% {\sc {Cortex}} assumes that control flow depends purely on data structure 
% connectivity. This is not the case with decision trees. Control flow is determined 
% by the input value and not by the structure of the tree. This makes the 
% problems addressed by \Treebeard{} very different from the ones handled
% by {\sc {Cortex}}.
Libraries that compose or generate optimized implementations  
for BLAS\cite{BLIS, atlas_sc98, CUTLASS} and signal processing\cite{FFTW, SPIRAL}
have also been developed.
However, \Treebeard{} is the first system that provides state-of-the-art performance
across targets by implementing a scheduling language for decision tree inference.
% BLIS\cite{BLIS} and ATLAS\cite{atlas_sc98}
% are systems to instantiate high performance BLAS routines on multiple 
% target architectures. CUTLASS\cite{CUTLASS} provides building blocks 
% in the form of C++ templates to quickly instantiate high performance 
% BLAS functions on different GPU architectures. SPIRAL\cite{SPIRAL}
% is a domain specific compiler for signal processing applications
% that instantiates high performance routines for linear transformations 
% like FFTs and FIR filters. FFTW\cite{FFTW} is a fast fourier transform 
% compiler that generates high performance FFT routines by customizing 
% the routine based on FFT size and the target machine. 
% However, no such frameworks for decision tree ensembles exist currently.
% \Treebeard{} is a first step in this direction and unlocks several future 
% optimization opportunities.

\emph{Reductions:} CUB\cite{CUB} and Thrust\cite{Thrust} are libraries 
that implement reductions on GPUs. 
However, it is not possible to fuse these functions with other computations  
as required in \Treebeard{}. Reddy et. al.~\cite{ChandanReduction}
describe language constructs in {\sc {Pencil}}~\cite{Pencil}
to express reductions and optimize them using the polyhedral 
framework. 
% It is not clear how these techniques can be fused with 
% other computations in arbitrary loop nests as required in \Treebeard{}.
%Additionally, 
Their system does not express the hierarchical nature of 
reductions and also only targets GPUs. Suriana et. al.~\cite{HalideReductions}
extend Halide to add support for reductions in the Halide 
scheduling language and to synthesize reduction operators. De Gonzalo 
et. al.~\cite{TangramReduction} describe a system based on Tangram 
that composes several partial reduction implementations into different 
reduction implementations for GPUs and then searches through these
alternate implementations to find the best ones. In summary, none of 
these systems provide abstractions and a general framework to generate 
and optimize reductions across different target processors as \Treebeard{} does.