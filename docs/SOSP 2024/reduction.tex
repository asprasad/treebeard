\section{Reductions : Representation, Optimization and Lowering}
\label{sec:reduction}
\Treebeard{} needs to sum up individual tree predictions to compute 
the prediction of the model while performing inference. However,
generating fused reductions within arbitrary loop nests specified 
using \Treebeard{}'s scheduling language is non-trivial. We found 
that existing reduction support in MLIR is insufficient to code
generate and optimize these reductions. MLIR only supports reductions
of value types and does not provide ways to lower reductions to GPUs. 
To address this gap, we design an MLIR dialect that allows us to
specify accumulating values into an element of a multi-dimensional array
and can be lowered to CPU or GPU. 

The main abstraction we introduce is the \op{reduce} op. It models 
atomically accumulating values into an element of a 
multi-dimensional array (represented by an MLIR \op{memref}).
The following example shows how the \op{reduce} op can be used to 
sum up the elements of an array in parallel.

\begin{lstlisting}[style=c++]
  float arr[10], result[1]
  par.for i0 = 0  to 10 step 5 {
    for i1 = 0 to 5 
      reduce(result[0], arr[i0 + i1]) <"+", 0.0>
  }
\end{lstlisting}

The semantics of the \op{reduce} op is exactly the semantics of 
an atomic accumulation, i.e. it guarantees that all accumulations 
are correctly performed even in the presence of parallel loops. 
The \op{reduce} op is defined for all associative and commutative
reduction operations with a well-defined initial value. The 
reduction operator and the initial value are attributes applied
on the \op{reduce} op. 

Having modeled the reductions with an abstract operation, the 
aim now is to lower this to a correct and optimized 
implementation on both CPU and GPU. In order to do this, we 
first determine if any parallel loop iterations can accumulate 
into the same array element. We call such loops 
\emph{\textbf{reduction loops}}. If such loops exist, we 
\textbf{\emph{privatize}} the array for each iteration of the
loop. We call this process \textbf{\emph{legalization}}.
Subsequently, each privatized dimension 
can be reduced at the end of the reduction loop it was inserted for. 
\TODO{We cannot do better than this in terms of memory usage} \TODO{Need a proof}.


In our example above, parallel iterations of the \op{i0} loop accumulate 
into the same element of the \op{result} array.
We would therefore privatize the \op{result} array for each iteration of 
the \op{i0} loop as follows. 
\begin{lstlisting}[style=c++]
  float arr[10], result[1]
  float resultPriv[2][1]
  par.for i0 = 0  to 10 step 5 {
    for i1 = 0 to 5 
      reduce(resultPriv[i0/5][0], arr[i0 + i1]) <"+", 0.0>
  }
  result = reduce_dimension(resultPriv, 0)
\end{lstlisting}

The op \op{reduce\_dimension} reduces values across the specified
dimension of an n-dimensional array. In the above example, 
the \op{reduce\_dimension} op is reducing across all elements 
of the first dimension (dimension 0). Therefore, in this case, it 
produces a result memref with a single element (the first dimension
with size 2 is collapsed). 

% \begin{definition}
%   A parallel loop surrounding one or more \op{reduce} ops is 
%   a \textbf{conflicting loop} for a target multi-dimensional array if this 
%   loop has a non-zero dependence distance for the dependence between
%   any of the contained \op{reduce} ops.
% \end{definition} 

% \begin{definition}
%   \op{\textbf{reduce\_dimension(targetMemref, memref, dim, [indices], [rangeStart], [rangeEnd])}}:
%    Computes the reduction over the dimension specified by \op{dimension} and stores the 
%    result in \op{targetMemref}. \op{[indices]} must be a vector of \op{dim} elements
%    (or empty if the dimension being reduced is the first dimension). \op{[rangeStart]} 
%    and \op{[rangeEnd]} represent the range of indices following the reduction dimension and 
%    must have the same number of elements. If both are \op{null} (not passed), 
%    all elements of these dimensions are reduced. The computation performed by the op is as follows.

%   $targetMemref[\vec{\boldsymbol{indices}}, \vec{\boldsymbol{k}}] = \sum_{i=0}^{shape[dim]} memref[\vec{\boldsymbol{indices}}, i, \vec{\boldsymbol{k}}]\quad   \forall \vec{\boldsymbol{k}} \in \left[[rangeStart_0, rangeEnd_0), ... , [rangeStart_n, rangeEnd_n)\right]$
% \end{definition}

% Consider the following code with nested parallel loops. 
% (A situation where trees are split across both
% threads and thread blocks could result in such generated 
% code in \Treebeard{}.)
% \begin{lstlisting}[style=c++]
%   builtin.func @ReduceVector(%arr: memref<num_elemsxf64>, %result: memref<1xf64>) -> void {
%     par.for i0 = range(0 : num_elems/2 : num_elems) {
%       par.for i1 = range(0 : num_elems/4 : num_elems/2) {
%         for i2 = range(0 : num_elems/4) 
%           reduce(%result, 0, arr[i0 + i1 + i2]) <"+", 0.0>
%       }
%     }
%   }
% \end{lstlisting}

% Here, the \op{i0} and \op{i1} loops are conflicting loops wrt 
% the \op{result} memref. We \textbf{legalize} the reduction 
% by privatizing the \op{result} array wrt the \op{i0} and \op{i1} loops.
% However, there are now two privatized dimensions and therefore, 
% two dimensions need to be reduced to compute the final result.
% This multi-stage reduction is what enables us to model 
% hierarchical reductions.

% The following code shows how the reduction above is legalized. 
% We introduce a new op, \op{reduce\_dimension\_inplace} which 
% reduces a dimension of the input memref and stores results 
% in the same array. This helps saves memory by removing the 
% need to create multiple intermediate arrays to store results.
% Only the final dimension reduction uses the \op{reduce\_dimension} op.

% \begin{lstlisting}[style=c++]
%   builtin.func @ReduceVector(%arr: memref<num_elemsxf64>, %result: memref<1xf64>) -> void {
%     results_1 = memref<2x2x1xf64>
%     par.for i0 = range(0 : num_elems/2 : num_elems) {
%       par.for i1 = range(0 : num_elems/4 : num_elems/2) {
%         for i2 = range(0 : num_elems/4) 
%           index0 = i0/(num_elems/2)
%           index1 = i1/(num_elems/4)
%           reduce(%result_1[index0, index1, 0], arr[i0 + i1 + i2]) <"+", 0.0>
%       }
%       // result_1[i0/(num_elems/2), 0] = sum(result_1[i0/(num_elems/2), :])
%       reduce_dimension_inplace(%result_1, 1, i0/(num_elems/2)) 
%     }
%     // result = sum(result[:, 0])
%     %result = reduce_dimension(%result_1, 0)
%   }
% \end{lstlisting}

To reduce the amount of memory used by arrays introduced for reduction,
we introduce the \op{reduce\_dimension\_inplace} operation. It is similar to the 
\op{reduce\_dimension} op except that it updates the input array inplace
rather than writing results to a target array. It writes results to the 
zeroth index of the dimension being reduced.

% \begin{definition}
%  \textbf{\op{reduce\_dimension\_inplace(memref, dim, [indices], [rangeStart], [rangeEnd])}}:
%   Computes the reduction over the dimension specified by \op{dimension} and stores the 
%   result at index 0 of that dimension. \op{[indices]} must be a vector of \op{dim} elements
%    (or empty if the dimension being reduced is the first dimension). \op{[rangeStart]} 
%    and \op{[rangeEnd]} must have the same number of elements. If both are \op{null} (not passed), 
%    all elements of the corresponding dimension are reduced. 
  
%   The computation performed by the op is defined by the following equation.
  
%   $memref[\vec{\boldsymbol{indices}}, 0, \vec{\boldsymbol{k}}] = \sum_{i=0}^{shape[dim]} memref[\vec{\boldsymbol{indices}}, i, \vec{\boldsymbol{k}}]\quad   \forall \vec{\boldsymbol{k}} \in \left[[rangeStart_0, rangeEnd_0), ... , [rangeStart_n, rangeEnd_n)\right]$  
% \end{definition}

\subsection{Lowering Reduction Operations}
We implement lowering of the operations defined above to both the CPU and GPU.
Since the lowering pipeline from MIR to LIR are different for CPU and GPU 
compilation, we implement lowering and optimization of our reduction dialect to CPUs and
GPUs simply using different MLIR rewrite patterns. In this section, we briefly describe 
how these operations are lowered to the CPU and GPU. 

\subsubsection{Lowering to CPU}
The lowering of the reduction operations to CPU is fairly straightforward. We lower 
\op{reduce\_dimension\_inplace} and 
\op{reduce\_dimension} to a simple loop nest that goes over the specified
subset of the input array, performs the reduction and writes 
the result into the appropriate location of the target array. 
If the schedule specifies that the reduction is to be vectorized,
then as many elements as specified by the vector width are read 
from the input array as a vector, accumulated as a vector, and 
finally written back to the target array.

% In general, this works 
% well because reductions are typically being performed on dimensions
% other than the inner-most dimension and therefore, this strategy
% loads successive elements from memory maximizing memory bandwidth 
% utilization. 

% \TODO{explain atomic reduction}

\subsubsection{Lowering to GPU}
The same abstractions can be lowered to efficient GPU implementations
and therefore, simplify higher-level code generation. The lowering for 
the inplace and non-inplace operations are essentially the same, except 
for the target array and we do not distinguish between them except 
for finally storing the result. 

The lowering of the \op{reduce\_dimension\_*} ops can either exploit
parallelism across the independent reductions or 
the inherent parallelism in the reduction by performing a divide and conquer 
reduction. If there are enough independent reductions to keep all threads
in a thread block busy, then the lowering pass can generate code that performs one (or 
multiple) reductions in each thread. If, however, there are not 
enough independent reductions, then the lowering pass generates a tree 
style reduction where multiple threads cooperate to perform a single reduction
using inter-thread shuffles.

Another feature specific to GPU reductions is the use of shared memory. 
If the schedule specifies that the reduction needs to be performed 
using shared memory, the privatized buffer is allocated in shared memory. 
The compiler only allocates as much shared memory 
as needed to hold values processed by a single thread-block.
%  and 
% index offsets are appropriately rewritten to handle the differences between 
% the indexing of the target memref and the shared memory array.
Our abstractions allow our lowering passes to be written completely 
independent of whether we use shared memory and therefore allow 
us to enable or disable shared memory use independently from the other 
parts of the compiler. 

\subsection{Use in \Treebeard{}}
We now show how \Treebeard{} uses the reduction dialect to generate 
code for decision tree inference using an example. 
In our example, \op{N\_t} is the number of trees 
and \op{batch\_size} is the batch size. The schedule tiles both the 
batch and tree loops and parallelizes the outer batch 
and tree loops. The schedule with which code is generated is as follows.
% In the context of \Treebeard{} we note that, the set of reduction loops is exactly the set 
% of surrounding parallel loops that are iterating over trees. The 
% results can be privatized for each loop iteratively 
% and reductions along each privatized dimension can be inserted  
% immediately following the loop the dimension was inserted due to.

\begin{lstlisting}[style=c++]
tile(batch, i0, i1, batch_size/2);
tile(tree, t0, t1, N_t/2);
reorder({i0, t0, t1, i1});
parallel(t0);
parallel(i0);
\end{lstlisting}

The MIR generated by \Treebeard{} for the above schedule is as follows. 
\begin{lstlisting}[style=c++]
  float result[batch_size]
  model = ensemble(...) 
  par.for i0 = 0 to batch_size step batch_size/2 {
    par.for t0 = 0 to N_t step N_t/2 {
      for t1 = 0 to N_t/2 {
        for i1 = 0 to batch_size/2 {
          t = getTree(model, t0 + t1) 
          p = walkDecisionTree(t, rows[i0+i1])
          reduce(result[i0+i1], p)
        }
      }
    }
  }
\end{lstlisting}

\Treebeard{} determines that the \op{t0} loop is a reduction 
loop w.r.t the \op{result} array and therefore legalizes 
the reduction by inserting a privatized array 
\op{partResults}. The privatized dimension of this array 
is reduced at the end of the \op{t0} loop.

\begin{lstlisting}[style=c++]
  float result[batch_size], partResults[2][batch_size]
  model = ensemble(...) 
  par.for i0 = 0 to batch_size step batch_size/2 {
    par.for t0 = 0 to N_t step N_t/2 {
      for t1 = 0 to N_t/2 {
        for i1 = 0 to batch_size/2 {
          t = getTree(model, t0 + t1) 
          p = walkDecisionTree(t, rows[i0+i1])
          reduce(result[i0+i1], p)
        }
      }
    }
    results[i0:i0+batch_size/2] = reduce_dimension(partResults[:, i0:i0+batch_size/2], 0)
  }
\end{lstlisting}

While legalizing the reduction, the compiler determines that the 
\op{reduce\_dimension} operation can only compute a subset of the final 
result (the subset that is computed within the current parallel iteration of the 
\op{i0} loop). 
% Once this process is complete, the \op{reduce} ops in 
% the result IR can be lowered to a simple ``read-accumulate-write''
% sequence of instructions

Finally, we note that in our experiments, we found that our 
current implementation of lowering the reduction operations 
was sufficient and reduction is not the bottleneck in our 
generated code. However, we believe this approach to enabling 
higher level code generators to easily generate reductions 
through simple abstractions and then having the compiler 
automatically lower them to efficient implementation is an
important area for future work with applicability in several 
domains. 

%\TODO{Should we mention how we handle multi-class models?}