%%
%% This is file `sample-sigplan.tex',
%% generated with the docstrip utility.
%%
%% The original source files were:
%%
%% samples.dtx  (with options: `sigplan')
%% 
%% IMPORTANT NOTICE:
%% 
%% For the copyright see the source file.
%% 
%% Any modified versions of this file must be renamed
%% with new filenames distinct from sample-sigplan.tex.
%% 
%% For distribution of the original source see the terms
%% for copying and modification in the file samples.dtx.
%% 
%% This generated file may be distributed as long as the
%% original source files, as listed above, are part of the
%% same distribution. (The sources need not necessarily be
%% in the same archive or directory.)
%%
%% Commands for TeXCount
%TC:macro \cite [option:text,text]
%TC:macro \citep [option:text,text]
%TC:macro \citet [option:text,text]
%TC:envir table 0 1
%TC:envir table* 0 1
%TC:envir tabular [ignore] word
%TC:envir displaymath 0 word
%TC:envir math 0 word
%TC:envir comment 0 0
%%
%%
%% The first command in your LaTeX source must be the \documentclass command.
% \documentclass[sigplan,screen]{acmart}

\documentclass[sigplan,10pt]{acmart}
\renewcommand\footnotetextcopyrightpermission[1]{}

\usepackage{listings}
\usepackage{tikz}
\usepackage{amsthm}
\usepackage{graphicx}
\usepackage{subcaption}
\usepackage{algorithm}
%\usepackage{algorithmic}
\usepackage{algpseudocode}
\usepackage{tabularx,booktabs}
\usepackage{multirow}

\newcommand{\CommentOut}[1]{}
\newcommand\TODO[1]{\textcolor{red}{TODO #1}}
\newcommand\COMMENT[1]{\textcolor{green}{COMMENT #1}}
\newcommand{\Treebeard}{{\sc {SilvanForge}}}
\newcommand{\TreebeardOLD}{{\sc {Treebeard}}}
\newcommand{\op}[1]{{\texttt {#1}}}

\definecolor{codegreen}{rgb}{0,0.6,0}
\definecolor{codegray}{rgb}{0.5,0.5,0.5}
\definecolor{codepurple}{rgb}{0.58,0,0.82}
% \definecolor{backcolour}{rgb}{0.95,0.95,0.92}
\definecolor{backcolour}{HTML}{F2F7FC}
\definecolor{circlegreen}{rgb}{0.44,0.78,0.28}
\definecolor{circleblue}{rgb}{0.27,0.45,0.77}
\definecolor{circleorange}{HTML}{F4B183}
\definecolor{MidnightBlue}{HTML}{006795}
\definecolor{OliveGreen}{HTML}{3C8031}
\definecolor{OrangeCircleOutline}{HTML}{C55A11}

\newcommand*\circled[1]{\tikz[baseline=(char.base)]{
            \node[shape=circle, draw=OliveGreen, inner sep=1pt, fill=circlegreen, text=white] (char) {#1};}}
\newcommand*\bluecircled[1]{\tikz[baseline=(char.base)]{
            \node[shape=circle, draw=MidnightBlue, inner sep=1pt, fill=circleblue, text=white] (char) {#1};}}
\newcommand*\orangecircled[1]{\tikz[baseline=(char.base)]{
              \node[shape=circle, draw=OrangeCircleOutline, inner sep=1pt, fill=circleorange, text=white] (char) {#1};}}

\lstdefinestyle{c++}{language=C++,
    morekeywords={ensemble, getTree, getRoot, isLeaf, isLeafTile, traverseTreeTile, getLeafValue, loadThresholds,
                  loadFeatureIndices, loadTileShape, getChildTile, parallel, par, InterleavedWalk, evaluateTilePredicates,
                  getChildTile, WalkDecisionTree, walkDecisionTree, moveToChildTile, to, step, tile, split, unroll,
                  reorder, specialize, unrollDepth, unrollWalk, reduce, gpuDimension},
    backgroundcolor=\color{backcolour},
    commentstyle=\color{codegreen},
    keywordstyle=\color{blue},
    numberstyle=\tiny\color{codegray},
    stringstyle=\color{codepurple},
    basicstyle=\scriptsize,
    breakatwhitespace=false,
    breaklines=true,
    captionpos=b,
    keepspaces=true,
    numbers=left,
    numbersep=5pt,
    showspaces=false,
    showstringspaces=false,
    showtabs=false,
    tabsize=2
}

\lstset{style=c++}


%% NOTE that a single column version is required for 
%% submission and peer review. This can be done by changing
%% the \doucmentclass[...]{acmart} in this template to 
%% \documentclass[manuscript,screen,review]{acmart}
%% 
%% To ensure 100% compatibility, please check the white list of
%% approved LaTeX packages to be used with the Master Article Template at
%% https://www.acm.org/publications/taps/whitelist-of-latex-packages 
%% before creating your document. The white list page provides 
%% information on how to submit additional LaTeX packages for 
%% review and adoption.
%% Fonts used in the template cannot be substituted; margin 
%% adjustments are not allowed.
%%
%% \BibTeX command to typeset BibTeX logo in the docs
\AtBeginDocument{%
  \providecommand\BibTeX{{%
    \normalfont B\kern-0.5em{\scshape i\kern-0.25em b}\kern-0.8em\TeX}}}

%% Rights management information.  This information is sent to you
%% when you complete the rights form.  These commands have SAMPLE
%% values in them; it is your responsibility as an author to replace
%% the commands and values with those provided to you when you
%% complete the rights form.
\setcopyright{acmlicensed}
\copyrightyear{2018}
\acmYear{2018}
\acmDOI{XXXXXXX.XXXXXXX}

%% These commands are for a PROCEEDINGS abstract or paper.
\acmConference[Conference acronym 'XX]{Make sure to enter the correct
  conference title from your rights confirmation emai}{June 03--05,
  2018}{Woodstock, NY}
%
%  Uncomment \acmBooktitle if th title of the proceedings is different
%  from ``Proceedings of ...''!
%
%\acmBooktitle{Woodstock '18: ACM Symposium on Neural Gaze Detection,
%  June 03--05, 2018, Woodstock, NY} 
\acmISBN{978-1-4503-XXXX-X/18/06}


%%
%% Submission ID.
%% Use this when submitting an article to a sponsored event. You'll
%% receive a unique submission ID from the organizers
%% of the event, and this ID should be used as the parameter to this command.
%%\acmSubmissionID{123-A56-BU3}

%%
%% For managing citations, it is recommended to use bibliography
%% files in BibTeX format.
%%
%% You can then either use BibTeX with the ACM-Reference-Format style,
%% or BibLaTeX with the acmnumeric or acmauthoryear sytles, that include
%% support for advanced citation of software artefact from the
%% biblatex-software package, also separately available on CTAN.
%%
%% Look at the sample-*-biblatex.tex files for templates showcasing
%% the biblatex styles.
%%

%%
%% The majority of ACM publications use numbered citations and
%% references.  The command \citestyle{authoryear} switches to the
%% "author year" style.
%%
%% If you are preparing content for an event
%% sponsored by ACM SIGGRAPH, you must use the "author year" style of
%% citations and references.
%% Uncommenting
%% the next command will enable that style.
%%\citestyle{acmauthoryear}

%%
%% end of the preamble, start of the body of the document source.
\begin{document}

%%
%% The "title" command has an optional parameter,
%% allowing the author to define a "short title" to be used in page headers.
\title{{\textbf{\Treebeard{}}} : A Schedule Guided Retargetable Compiler for Decision Tree Inference}

%%
%% The "author" command and its associated commands are used to define
%% the authors and their affiliations.
%% Of note is the shared affiliation of the first two authors, and the
%% "authornote" and "authornotemark" commands
%% used to denote shared contribution to the research.

% \author{Ben Trovato}
% \authornote{Both authors contributed equally to this research.}
% \email{trovato@corporation.com}
% \orcid{1234-5678-9012}
% \author{G.K.M. Tobin}
% \authornotemark[1]
% \email{webmaster@marysville-ohio.com}
% \affiliation{%
%   \institution{Institute for Clarity in Documentation}
%   \streetaddress{P.O. Box 1212}
%   \city{Dublin}
%   \state{Ohio}
%   \country{USA}
%   \postcode{43017-6221}
% }

% \author{Lars Th{\o}rv{\"a}ld}
% \affiliation{%
%   \institution{The Th{\o}rv{\"a}ld Group}
%   \streetaddress{1 Th{\o}rv{\"a}ld Circle}
%   \city{Hekla}
%   \country{Iceland}}
% \email{larst@affiliation.org}

% \author{Valerie B\'eranger}
% \affiliation{%
%   \institution{Inria Paris-Rocquencourt}
%   \city{Rocquencourt}
%   \country{France}
% }

% \author{Aparna Patel}
% \affiliation{%
%  \institution{Rajiv Gandhi University}
%  \streetaddress{Rono-Hills}
%  \city{Doimukh}
%  \state{Arunachal Pradesh}
%  \country{India}}

% \author{Huifen Chan}
% \affiliation{%
%   \institution{Tsinghua University}
%   \streetaddress{30 Shuangqing Rd}
%   \city{Haidian Qu}
%   \state{Beijing Shi}
%   \country{China}}

% \author{Charles Palmer}
% \affiliation{%
%   \institution{Palmer Research Laboratories}
%   \streetaddress{8600 Datapoint Drive}
%   \city{San Antonio}
%   \state{Texas}
%   \country{USA}
%   \postcode{78229}}
% \email{cpalmer@prl.com}

% \author{John Smith}
% \affiliation{%
%   \institution{The Th{\o}rv{\"a}ld Group}
%   \streetaddress{1 Th{\o}rv{\"a}ld Circle}
%   \city{Hekla}
%   \country{Iceland}}
% \email{jsmith@affiliation.org}

% \author{Julius P. Kumquat}
% \affiliation{%
%   \institution{The Kumquat Consortium}
%   \city{New York}
%   \country{USA}}
% \email{jpkumquat@consortium.net}

%%
%% By default, the full list of authors will be used in the page
%% headers. Often, this list is too long, and will overlap
%% other information printed in the page headers. This command allows
%% the author to define a more concise list
%% of authors' names for this purpose.
% \renewcommand{\shortauthors}{Trovato and Tobin, et al.}

%%
%% The abstract is a short summary of the work to be presented in the
%% article.
% \begin{abstract}
%   A clear and well-documented \LaTeX\ document is presented as an
%   article formatted for publication by ACM in a conference proceedings
%   or journal publication. Based on the ``acmart'' document class, this
%   article presents and explains many of the common variations, as well
%   as many of the formatting elements an author may use in the
%   preparation of the documentation of their work.
% \end{abstract}

%%
%% The code below is generated by the tool at http://dl.acm.org/ccs.cfm.
%% Please copy and paste the code instead of the example below.
%%
\begin{CCSXML}
% <ccs2012>
%  <concept>
%   <concept_id>00000000.0000000.0000000</concept_id>
%   <concept_desc>Do Not Use This Code, Generate the Correct Terms for Your Paper</concept_desc>
%   <concept_significance>500</concept_significance>
%  </concept>
%  <concept>
%   <concept_id>00000000.00000000.00000000</concept_id>
%   <concept_desc>Do Not Use This Code, Generate the Correct Terms for Your Paper</concept_desc>
%   <concept_significance>300</concept_significance>
%  </concept>
%  <concept>
%   <concept_id>00000000.00000000.00000000</concept_id>
%   <concept_desc>Do Not Use This Code, Generate the Correct Terms for Your Paper</concept_desc>
%   <concept_significance>100</concept_significance>
%  </concept>
%  <concept>
%   <concept_id>00000000.00000000.00000000</concept_id>
%   <concept_desc>Do Not Use This Code, Generate the Correct Terms for Your Paper</concept_desc>
%   <concept_significance>100</concept_significance>
%  </concept>
% </ccs2012>
\end{CCSXML}

% \ccsdesc[500]{Do Not Use This Code~Generate the Correct Terms for Your Paper}
% \ccsdesc[300]{Do Not Use This Code~Generate the Correct Terms for Your Paper}
% \ccsdesc{Do Not Use This Code~Generate the Correct Terms for Your Paper}
% \ccsdesc[100]{Do Not Use This Code~Generate the Correct Terms for Your Paper}

%%
%% Keywords. The author(s) should pick words that accurately describe
%% the work being presented. Separate the keywords with commas.
\keywords{}

%% A "teaser" image appears between the author and affiliation
%% information and the body of the document, and typically spans the
%% page.
% \begin{teaserfigure}
%   \includegraphics[width=\textwidth]{sampleteaser}
%   \caption{Seattle Mariners at Spring Training, 2010.}
%   \Description{Enjoying the baseball game from the third-base
%   seats. Ichiro Suzuki preparing to bat.}
%   \label{fig:teaser}
% \end{teaserfigure}

% \received{20 February 2007}
% \received[revised]{12 March 2009}
% \received[accepted]{5 June 2009}

%%
%% This command processes the author and affiliation and title
%% information and builds the first part of the formatted document.

\begin{abstract}
  
  The rapid proliferation of machine learning coupled with accelerating evolution of the hardware ecosystem has led to a surge in the demand for model inference on a variety of hardware.
  %While CPUs have been the mainstay for machine learning inference, the availability of GPUs holds promise to scale to bigger and more powerful models. 
  This paper is motivated by the problems encountered when targeting inference of decision tree based models, the most popular models on tabular data, to run at peak performance on 
  diverse CPU and GPU targets. We evaluated existing solutions and found that they do not provide portable performance across different hardware targets.
  %Decision tree based models are widely used in practice due to their robustness, interpretability, and ability to handle missing data.  
  
  To address this we present the design of \Treebeard{}, a schedule guided compiler  
  for decision tree based models that searches over a large design space 
  to automatically generate high-performance inference routines.
  %We re-architect the open-source \TreebeardOLD{} infrastructure and significantly extend it to enable high-performance code generation across target processors. 
  \Treebeard{} has two core components. A scheduling language 
  that encapsulates the large optimization space for decision tree inference, 
  and techniques to efficiently explore this space.
  \TODO{Change large optimization space}.
  %We also design a set of heuristics that can find near-optimal solutions in quick time.
  Second, an optimizing multi-level compiler that can generate code based on the schedule. 
  For the latter, we re-architect the open-source \TreebeardOLD{} CPU compiler to support schedule guided compilation and
  add support for GPU code generation. GPU code generation required fundamental new optimization interfaces for 
  caching, parallel reduction, and support for multiple in-memory representations of trees.
  
  %The compiler builds on an existing compiler for CPUs, \TreebeardOLD{}, also add support for caching, parallel reduction and a plug-in mechanism to explore different in-memory representations of trees.

  We evaluate \Treebeard{} on over seven hundred diverse models and demonstrate that the best schedule varies drastically with model, batch size, and target hardware. 
  Our scheduling heuristic is able to quickly find near optimal schedules while searching over a small number (\~50) of schedules.
  In terms of performance, \Treebeard{} generated code is an order of magnitude faster than XGBoost and
  about 2-3$\times$ faster on average than RAPIDs FIL and Tahoe. While these systems only target NVIDIA GPUs, \Treebeard{} achieves competent performance on AMD GPUs as well. 
  On CPUs, \Treebeard{} achieves better scaling compared to \TreebeardOLD{}.
  For models where \TreebeardOLD{} was only able to achieve diminishing returns with an 
  increasing number of threads, \Treebeard{} is able to scale linearly with the number of threads.
  \TODO{(numbers for CPU performance?)}
\end{abstract}

\maketitle
\section{Introduction}
\label{sec:intro}
We are in the midst of a hardware revolution, a new golden age for computer architecture~\cite{GoldenAge}. The last 
decade has seen a shift in architectural paradigms, with the rise of GPUs and accelerators. This shift has been driven
by the necessity to innovate in the post Moore's law and Dennard's scaling era. This transformation has also played 
a significant role in the success of modern deep learning models, as they enable scaling model training and inference 
to a massive number of threads. Such scalability would be essential for all performance critical applications, including
other machine learning models that need to scale with increasing data sizes and model complexities. 
%need to accelerate machine learning workloads. The success of modern AI models is due to their ability to scale both training and inference to a massive number of threads. This has been made possible by the rise of GPUs and accelerators.

%Despite these recent advancements, one dominant family of models, namely decision forest models have not seen the same level of acceleration.
Decision forest models remain the mainstay for machine learning over tabular data~\cite{DLNotAllYouNeed,TreebasedOutperformDL}. 
Their robustness, interpretability, and ability to handle missing data make them a popular choice for a wide range of applications. 
\TODO{a few more lines, classification, regression, etc.}
%We observe that while scalable performance on neural network models has received significant attention, decision tree inference has not seen the same level of acceleration.
A recent survey~\cite{kaggle}\dots. 
The survey also finds that, the cost of inference is the most critical factor in the overall cost of deploying a machine learning model.
This is because, in production settings, each model is trained once and often used for inference millions of times. 
However, inference is run on a variety of hardware platforms, ranging from low to high-end CPUs and GPUs. 
This paper is motivated by the need to accelerate decision tree inference to achieve portable performance on a variety of hardware.
% In particular, we focus on a range of commodity CPUs and GPUs, a class of hardware that has seen widespread 
% adoption across client/edge devices used for inference.

Decision forest models are composed of a large collection of decision trees (100-1000), and inference involves 
traversing down each tree in the forest and aggregating the predictions. Inference is typically done in a batched
setting, where multiple inputs are processed simultaneously.
Despite the simplicity of the model and the availability of multiple sources of coarse grain parallelism (parallelism 
across inputs in a batch and parallelism across trees), existing systems do not consistently scale well across different
models even on the limited set of targets they support. 
%This is because existing systems exploit a limited set of optimizations and often specialize the implementation to a specific hardware platform, and this limits their portability. 

%\TODO{cover Rapids, Tahoe, XGBoost and Treebeard}.  
Evaluation on a diverse set of models highlights that the best implementation often requires a careful composition 
of many optimization strategies like data layout optimizations, 
loop transformations, parallelization, and memory access optimizations. 
%For example, simultaneously traversing multiple trees with many threads, while managing the working set is 
%critical to get scalable performance. 
%This requires a combination of techniques like data layout optimizations, loop transformations, and memory access optimizations. 
%Existing systems each use specific data layouts for the underlying tree and use fixed loop structures. 
%Parallelism, loop structure and model representation together determine the accesses patterns and the working set. 
Existing systems today are mostly library based, and only support a predefined combination of optimizations. 
XGBoost~\cite{XGBoost} uses a sparse representation for the model and a loop structure that processes one tree for a block of rows before moving to 
the next tree. RAPIDS FIL~\cite{FIL} uses a reorg representation and partitions trees across a fixed number of threads. Tahoe~\cite{Tahoe} uses a
variation of the reorg representation and has four predefined inference strategies from which it picks one based on an analytical model.
%These systems are all library based and are hard to extend. 
\TreebeardOLD{}, the state-of-the-art model compiler for CPUs, supports two fixed loop structures and does not scale well with increasing number of threads. Additionally, it lacks GPU specific optimizations that are critical to scale performance to massive number of threads.

This paper presents \Treebeard{}, a novel schedule guided compilation infrastructure for decision tree inference on multiple target hardware. \Treebeard{} is able to generate high-performance code for decision tree inference by exploring a large optimization space. The code generation is guided by a \emph{schedule} written in a custom scheduling language that can represent a wide range of implementation strategies. We demonstrate that the language is sufficient to express the various optimizations proposed by prior work, and further generalizes them and incorporates several new optimizations. We also design and implement a heuristic that is able to quickly find high-performance schedules for the model being compiled.
\TODO{Performance evaluation summary}
\TODO{Oganization}
\subsection{Contributions}
\begin{itemize}
  \item We present the design for a multi-target decision tree compiler infrastructure and implement several optimizations within this framework.
   We are also the first to implement an optimizing compiler for decision tree inference on GPUs.
  \item We identify that an extensive optimization space exists for the problem of decision tree inference. We design a scheduling language that 
  allows us to effectively represent this solution space abstractly. This scheduling language is expressive enough to represent a wide range of  
  implementation strategies proposed by prior work.
  \item To the best of our knowledge, we perform the first extensive characterization of the optimization space for decision tree inference 
  on GPUs. Using some of the characteristics we identify, we design and implement a heuristic that is able to quickly find high-performance 
  schedules for the model being compiled. 
  \item We design and implement a general framework for expressing and optimizing reductions within MLIR. To the best of our knowledge, this is the first
  such framework.
  \item We evaluate our implementation by comparing it against RAPIDs and Tahoe, the state-of-the-art decision tree inference frameworks for GPU and 
  report significant speedups. We also show that our compiler can effectively target different GPUs, including both NVIDIA and AMD GPUs.
\end{itemize}
%\section{Motivation}

In this section, we first motivate the need for a scheduling language 
by showing how a model can be compiled in different ways
and subsequently, we show how drastically 
performance can vary across these variants for real benchmarks.

As an example, consider a model with four trees, two complete
trees of depth 1 and two of depth 2 (Figure \ref{Fig:HIRExample}).
% We use this model as our running example throughout this section.
We first describe a simple strategy that processes one tree at a time 
for all input rows and unrolls all tree walks. 
The loop over the trees is split into two loops -- one that
iterates over the first two trees (Trees 1 and 2 with depth 1) and 
the second that iterates over the last two trees (Trees 3 and 4 with
depth 2). The \Treebeard{} schedule then unrolls the tree walks for each tree.
\begin{lstlisting}[style=c++]
  reorder(tree, batch)
  // Fiss tree loop so trees with equal depth 
  // are processed together  
  split(tree, t_depth1, t_depth2, 2)
  // Unroll the tree walks
  unrollWalk(t_depth1, 1)
  unrollWalk(t_depth2, 2)
\end{lstlisting}
The concrete implementation of this schedule (in one of \Treebeard{}'s IRs) 
is as follows.
\begin{lstlisting}[style=c++]
  model = ensemble(...)
  for t_depth1 = 0 to 2 step 1 {
    T = getTree(ensemble, t_depth1)
    for batch = 0 to BATCH_SIZE step 1 {
      treePred = walkDecisionTree(T, 
                    input[batch]) <unrollDepth = 1>
      reduce(result[batch], treePred)
    }
  }
  for t_depth2 = 2 to 4 step 1 {
    T = getTree(ensemble, t_depth2)
    for batch = 0 to BATCH_SIZE step 1 {
      treePred = walkDecisionTree(T,
                    input[batch]) <unrollDepth = 2>
      reduce(result[batch], treePred) <'+', 0.0>
    }
  }
\end{lstlisting}
This schedule is ideally suited for a single-core CPU. It maximizes 
the reuse of trees in the L1 cache and also minimizes the amount of
branching by unrolling tree walks. However, it doesn't exploit  
any parallelism and is therefore ill-suited for parallel processors.

One form of parallelism that can be exploited is to process rows in 
parallel. However, with massively parallel processors like GPUs,
this may not yield sufficient parallel work. Another option is to also 
parallelize across trees. A possible strategy to accomplish this is 
encoded in the following schedule.
\begin{lstlisting}[style=c++]
  // Split the trees into two sets
  tile(tree, t0, t1, 2)
  reorder(batch, t1, t0)
  // Fiss loop so that trees with equal 
  // depth are processed together
  split(t0, t0_depth1, t0_depth2, 2)
  unrollWalk(t0_depth1, 1)
  unrollWalk(t0_depth2, 2)
  // Configure the GPU kernel dimensions
  gpuDimension(batch, grid.x)
  gpuDimension(t1, block.x)
\end{lstlisting}
This schedule generates an inference function that runs on the GPU. 
The inference routine processes one input row per thread block (since the \op{batch}
loop is mapped directly to \op{grid.x}).
It also splits the trees into two sets by tiling the \op{tree} loop.
Each of the two sets is processed in parallel. We unroll the tree walks 
for each tree. The IR generated is as follows. 
\begin{lstlisting}[style=c++]
  model = ensemble(...)
  par.for batch = 0 to BATCH_SIZE step 1 <grid.x> {
    par.for t1 = 0 to 2 step 1 <block.x> {
      for t0_depth1 = 0 to 2 step 2 {
        T = getTree(ensemble, t0_depth1 + t1)
        treePred = walkDecisionTree(T, 
                        input[batch]) <unrollDepth = 1>
        reduce(result[batch], treePred)
      }
      for t0_depth2 = 2 to 4 step 2 {
        T = getTree(ensemble, t0_depth2 + t1)
        treePred = walkDecisionTree(T,
                        input[batch]) <unrollDepth = 2>
        reduce(result[batch], treePred) <'+', 0.0>
      }
    }
  }
\end{lstlisting}

In the case of this schedule, the \op{reduce}
operation needs special consideration. In order to correctly generate 
code for this schedule, the compiler needs to determine that parallel 
iterations of the \op{t1} 
loop accumulate into the same element of the \op{result} array.
One possible solution is to rewrite the reduction so that each parallel 
iteration accumulates into a different array element by introducing 
a temporary buffer (\op{temp}) as follows.
\begin{lstlisting}[style=c++]
  float temp[2][BATCH_SIZE]
  model = ensemble(...)
  par.for batch = 0 to BATCH_SIZE step 1 <grid.x> {
    par.for t1 = 0 to 2 step 1 <block.x> {
      for t0_depth1 = 0 to 2 step 2 {
        T = getTree(ensemble, t0_depth1 + t1)
        treePred = walkDecisionTree(T, 
                      input[batch]) <unrollDepth = 1>
        reduce(temp[t1][batch], treePred)
      }
      for t0_depth2 = 2 to 4 step 2 {
        T = getTree(ensemble, t0_depth2 + t1)
        treePred = walkDecisionTree(T,
                      input[batch]) <unrollDepth = 2>
        reduce(temp[t1][batch], treePred) <'+', 0.0>
      }
    }
    result[batch] = reduce_dimension(temp[:][batch], 0)
  }
\end{lstlisting}
Here, partial results are accumulated into \op{temp} and then
reduced across the \op{t1} dimension to get the final result.

As is evident from these examples, it is possible to optimize 
the inference routine in different ways. Also, the structure of the loop 
nest in the inference routine can get quite complex even 
for simple schedules. Writing these routines by hand 
is error-prone and time-consuming.  We believe that designing a 
scheduling language to encapsulate these strategies and a principled code 
generator to automatically generate code based on the schedule is the 
best approach. 

\begin{figure}[]
  \centering
  \begin{subfigure}[b]{.3\textwidth}
    \subcaptionbox*{}{\includegraphics[width=\textwidth]{figures/batch_sensitivity_covtype.png}}
    \caption{\label{fig:sensitivitya} Batch sensitivity for \op{covtype}. Each point shows the slowdown when best schedule for the 
    x-axis batch size is used for the y-axis batch size.}
  \end{subfigure}
  % \hfill  % Add horizontal space between subfigures
  \begin{subfigure}[b]{.3\textwidth}
    \subcaptionbox*{}{\includegraphics[width=\textwidth]{figures/model_sensitivity_4096.png}}
    \caption{\label{fig:sensitivityb} Model sensitivity for batch size 4096. Each point shows the slowdown when best schedule for the
    x-axis model is used for the y-axis model.}
  \end{subfigure}
\end{figure}

To further complicate matters, we find that these different strategies can have significantly different
performance. Figures \ref{fig:sensitivitya} and \ref{fig:sensitivityb} show the
variation in performance when the same schedules are used across batch sizes 
and models respectively. These diagrams show that the best schedules to use 
vary across batch sizes and models. The largest slowdown is 5$\times$ when 
schedules are used across different batch sizes and 6$\times$ when schedules
are used across different models. Therefore, no one strategy can be used 
across all models and batch sizes. 
A system that is capable of specializing generated code for both batch 
size and model is required for the best performance.

Building such a configurable compiler and supporting code generation for CPUs and GPUs 
required us to solve several fundamental problems. We had to enable the
compiler to represent and optimize reductions, deal uniformly with different
in-memory representations of the model, design optimizations to effectively 
use the memory hierarchy of the target processor (shared memory on GPUs and 
cache on the CPU) and finally be able to generate target specific code. 
The rest of the paper describes these challenges in detail and how we
solved them in \Treebeard{}.

% Our main aim while designing \Treebeard{} was to unify the diverse set of implementation
% strategies that have been used in existing systems for decision tree inference. Some 
% differences in these systems are as follows:
% \begin{itemize}
%   \item Decision tree inference is run on several platforms including CPUs and GPUs. The 
%   implementations used on each of these platforms are different and the techniques used
%   to optimize them are also different.
%   \item A diverse set techniques have been proposed for optimization of decision tree 
%   inference on CPUs and GPUs \cite{VPred, Tahoe, Treelite, XGBoost, Hummingbird, QuickScorer, FIL}. 
%   No system exists that unifies the disparate optimizations implemented in these systems.
%   \item A very extensive design space of optimizations exists for decision tree inference
%   outside the few that have been proposed in the literature. However, currently no 
%   system exists that is capable of exploring this space and identifying the best set 
%   of parameters to use for a given model and platform.
%   \item Different systems use different in-memory representations for the model. For example,
%   XGBoost uses a sparse representation, RAPIDs FIL uses what is called the reorg representation 
%   and Tahoe uses a variation of the reorg representation. Currently, systems implement 
%   inference kernels that are tied to a single representation of the model. Again, this means
%   that no current system can explore different combinations of in-memory representations 
%   and optimizations.
% \end{itemize}
% At high-level, to make \Treebeard{} capable of unifying these differences, we design 1)  
% expressive intermediate representations that can represent and compose several proposed 
% optimizations 2) a scheduling language that specifies the structure of the
% generated code and 3) a plugin mechanism with which different in-memory representations
% can be composed with different optimizations. Finally, we develop a heuristic to
% explore the extensive optimization space that \Treebeard{}'s design enables.

% While a diverse set techniques have been proposed for optimization of decision tree 
% inference on CPUs and GPUs \cite{VPred, Tahoe, Treelite, XGBoost, Hummingbird, QuickScorer, FIL},
% a very extensive design space of optimizations exists 
% outside what has been proposed in the literature. Furthermore, decision tree inference 
% is run on several platforms including CPUs and GPUs. The implementations used on each of 
% these platforms are different and the techniques used to optimize them are different.
% To make matters even more complicated, several in-memory representations
% have been proposed for decision tree models. For example, XGBoost\cite{XGBoost} uses a sparse representation,
% RAPIDs FIL\cite{FIL} uses what is called the reorg representation and Tahoe uses a variation of the reorg
% representation. 
% \TODO{Can we add some numbers here to show that different models/batch sizes need different optimizations?}

% To solve the problems of exploring the design space of optimizations for decision tree
% inference and enabling portable performance, we build several techniques in \Treebeard{}, 
% an open source compiler infrastructure for decision tree inference. To make \Treebeard{}
% capable of unifying these different techniques and targets, we do the following. 
% \begin{itemize}
%   \item We design a scheduling language that encapsulates various optimization techniques
%   and controls the structure of the generated code.
%   \item We design an MLIR dialect to represent and optimize reductions and use this 
%   dialect within \Treebeard{} to enable the generation of different variants of 
%   inference routines.
%   \item We extend \Treebeard{}'s intermediate representations to include operations like caching.
%   We were able to easily reuse and extend \Treebeard{}'s IR as it was built as an MLIR dialect.
%   \item We design a plugin mechanism with which different in-memory representations
%   can be composed with different optimizations.  
% \end{itemize}
\section{Compiler Overview}
Figure \ref{Fig:CompilerStructure} shows the high level structure of Treebeard. The input to the compiler is a JSON file that describes the decision forest model and a set of options. The output of the compiler is a callable function that takes an array of rows and returns an array containing the model's prediction for each row. \TODO{Add a diagram, including IR stages}. The specified set of options includes information such as the batch size, the type of the input features, the type for node thresholds and the type for feature indices \TODO{There are also several optimization related inputs like tile size, type of tiling, pipeline depth etc. Should we mention those?}. Treebeard is written as a dialect in the MLIR framework \cite{MLIR}. At a high level, Treebeard first performs transformations on the trees in the model and subsequently generates and performs optimizations on a more traditional loop based IR. From an implementation perspective, Treebeard first constructs a high level representation of the decision forest inference operation and then progressively optimizes and lowers it to LLVM IR \cite{LLVM}. LLVM is then used to JIT compile the generated IR to executable code. The following sections describe each level of the IR and their lowering in more detail.

\begin{figure}
  \centering
  \includegraphics[scale = 0.3]{figures/CompilerStructure.PNG}
  \caption{Treebeard Compiler Structure}
  \label{Fig:CompilerStructure}
\end{figure}

\TODO{Should we describe the dialect's type system?}

\subsection{High Level IR}
Treebeard parses the input JSON file and generates a function with a single MLIR operation, \texttt{predictForest} that represents inference using the input model on a set of rows. The operation contains within it a tree based representation of the model that can be manipulated by optimizing transformations. Transformations on the model such as tiling, tree reordering and leaf padding are done at this level. The structure of the loop nest to walk the iteration space of trees and inputs is also decided at this level of the IR. \TODO{Should we mention that there is a scheduling language to decide this?}

\begin{lstlisting}[language=C++]
func Predict(float rows[batchSize]) {
  predictions = predictForest(rows) 
  return rows
}
\end{lstlisting}

\subsection{Mid Level IR}
The Mid Level IR makes the loop structures and tree walks more explicit. Firstly, the order in which the iteration space of trees and inputs is walked is explicitly specified in the IR through loop nests. Also, tree specific operations such as \texttt{isLeaf}, \texttt{traverseTile}, \texttt{getLeafValue} are introduced so that the traversal of trees explicitly represented. The following listing shows the IR for inference using a model with four trees on an input batch with two rows. The listed IR walks all trees for one input row before moving to the next row. One important point to note here is that details such as the data structure used for the trees are not explicitly encoded in the IR. This allows us to reuse optimization and lowering passes on this level of the IR regardless of what the final in memory representation of the model is.

\begin{lstlisting}[language=C++]
  // Constant that represents the model being compiled
  forest = ensemble(...)
  for i = 0 to 2 step 1 {
    prediction = 0
    for t = 0 to 4 step 1 {
      tree = getTree(forest, t) 
      node = getRoot(tree)
      while (isLeaf(tree, n)==false)  do {
        node = traverseTreeTile(tree, node, rows[i])
      }
      treePrediction = getLeafValue(tree, node)
      prediction = prediction + treePrediction
    }
    predictions[i] = prediction
  }
\end{lstlisting}

The IR listed above is a simplification of the actual IR. The actual IR is strongly typed and in SSA form.

\subsection{Low Level IR}
The IR is finally lowered to a form where the in memory representation of the model is made explicit. Buffers to hold the model values are inserted into the generated code and all tree operations in the mid-level IR are lowered to explicitly reference these buffers. The semantics of all operations are made explicit. For example, \texttt{traverseTreeTile} is lowered into a series of operations to load thresholds, feature indices and features, compare the features with the thresholds and compute the next node to evaluate. This IR is then lowered directly to LLVM IR and JITted.


% \section{\Treebeard{} IR Description}

% \begin{table*}[htb]
%   \centering
%   \resizebox{\linewidth}{!}{
%   \begin{tabularx}{\linewidth}{c l l l l}
%    \toprule
%    \textbf{Operation} & \textbf{Inputs} & \textbf{Outputs} & \textbf{Attributes} & \textbf{Description} \\
%    \midrule
%    \multirow{4}{*}{\texttt{predict\_ensemble}} & \textbf{data} & \textbf{result} & \textbf{ensemble} & Performs inference on the \textbf{data} using the model specified by\\
%                                                &               &                 & \textbf{predicate} & the \textbf{ensemble} attribute. The \textbf{schedule} attribute contains the \\
%                                                &               &                 & \textbf{schedule} & schedule described in Section \ref{sec:schedule}. \textbf{predicate} specifies the \\
%                                                 &               &                &                   & operator to use to evaluate nodes (Eg: $<$, $\leq$). \\
%    \midrule
%    \multirow{3}{*}{\texttt{walk\_decision\_tree}} & \textbf{trees[]} & \textbf{results[] }& \textbf{predicate}   & Represents an interleaved walk on all the element-wise pairs  \\ 
%                                                   & \textbf{rows[]}  &                    & \textbf{unrollDepth} & of \textbf{trees} and \textbf{rows}. \textbf{unrollDepth} specifies the number of \\
%                                                   &                  &                    &                      & hops to unroll. An array of tree walk results is returned.\\
%    \bottomrule
%   \end{tabularx}
%   }
%   \vskip 5pt
%   \caption{\label{Tab:Benchmarks}List of benchmark datasets and their parameters. 
%   The column \op{Leaf-biased} reports the number of leaf-biased trees per benchmark with $\langle\alpha =0.075, \beta =0.9 \rangle$. }
% \end{table*}

\begin{table*}[htb]
  \centering
  \resizebox{\linewidth}{!}{
  \begin{tabularx}{\linewidth}{c | l | l | l | l}
   \toprule
   \textbf{Operation} & \textbf{Inputs} & \textbf{Outputs} & \textbf{Attributes} & \textbf{Description} \\
   \midrule
   \multirow{4}{*}{\texttt{predictEnsemble}} & \textbf{rows[]} & \textbf{result} & \textbf{ensemble} & \multirow{4}{*}{\parbox{0.47\linewidth}{Performs inference on the data in \textbf{rows[]} 
                                                                                                     using the model specified by
                                                                                                     the \textbf{ensemble} attribute. The \textbf{schedule} attribute contains the 
                                                                                                     schedule described in Section \ref{sec:schedule}. \textbf{predicate} specifies the
                                                                                                     operator to use to evaluate nodes (Eg: $<$, $\leq$).}} \\
                                               &               &                 & \textbf{predicate} & \\
                                               &               &                 & \textbf{schedule} &  \\
                                               &               &                 &                   &  \\
   \midrule
   \midrule

   \multirow{3}{*}{\texttt{walkDecisionTree}} & \textbf{trees[]} & \textbf{results[] }& \textbf{predicate}   & \multirow{3}{*}{\parbox{0.47\linewidth}{Represents an interleaved walk on all the element-wise pairs 
                                                                                                                    of \textbf{trees} and \textbf{rows}. \textbf{unrollDepth} specifies the number of
                                                                                                                    hops to unroll. An array of tree walk results is returned.}} \\
                                                  & \textbf{rows[]}  &                    & \textbf{unrollDepth} &  \\
                                                  &                  &                    &                      & \\
   \midrule
   \multirow{2}{*}{\texttt{ensemble}} &  & \textbf{ensemble} & \textbf{model} & \multirow{2}{*}{\parbox{0.47\linewidth}{Represents the forest of trees that constitute the model. The   
                                                                                                                       \textbf{model} attribute contains the actual trees model.}} \\
                                      &  &                 &   & \\
   \midrule
   \multirow{2}{*}{\texttt{getTree}} & \textbf{ensemble} & \textbf{tree} &  & \multirow{2}{*}{\parbox{0.47\linewidth}{Get the tree at the specified index (\textbf{treeIndex}) from the \textbf{ensemble}.}} \\   
                                                                                                                       
                                      & \textbf{treeIndex} &                 &   & \\                                                                                                                       

   \midrule
   \multirow{2}{*}{\texttt{getTreeClassId}} & \textbf{ensemble} & \textbf{classId} &  & \multirow{2}{*}{\parbox{0.47\linewidth}{Get the class ID for the tree at index \textbf{treeIndex} 
                                                                                                                               in the \textbf{ensemble}. This is used for multi-class models.}} \\    
                                            & \textbf{treeIndex} &                 &   & \\                                                                                                                       

   \midrule
   \multirow{1}{*}{\texttt{getRoot}} & \textbf{tree} & \textbf{rootNode} &  & \multirow{1}{*}{\parbox{0.47\linewidth}{Get the root node of the specified tree.}} \\   

   \midrule
   \multirow{2}{*}{\texttt{isLeaf}} & \textbf{tree} & \textbf{bool} &  & \multirow{2}{*}{\parbox{0.47\linewidth}{Returns a boolean value indicating whether \textbf{node} is a leaf of \textbf{tree}.}} \\   
                                    & \textbf{node} &                 &   & \\                                                                                                                       

   \midrule
   \multirow{2}{*}{\texttt{getLeafValue}} & \textbf{tree} & \textbf{value} &  & \multirow{2}{*}{\parbox{0.47\linewidth}{Returns the value of the leaf \textbf{node} in \textbf{tree}.}} \\   
                                    & \textbf{node} &                 &   & \\                                                                                                                       

   \midrule
   \multirow{3}{*}{\texttt{traverseTreeTile}} & \textbf{trees[]} & \textbf{nodes[] }& \textbf{predicate}   & \multirow{3}{*}{\parbox{0.47\linewidth}{Represents an interleaved traversal of the   
                                                                                                                    nodes in \textbf{nodes} based on the data in \textbf{rows}. \textbf{predicate} specifies 
                                                                                                                    the operator to use to evaluate nodes.}} \\
                                                  & \textbf{nodes[]}  &             &               &  \\
                                                  & \textbf{rows[]}   &             &               & \\

   \midrule
   \multirow{3}{*}{\texttt{cacheTrees}} & \textbf{ensemble} & \textbf{ensemble}&     & \multirow{3}{*}{\parbox{0.47\linewidth}{Cache the trees in the \textbf{ensemble} between the specified \textbf{start} and \textbf{end} indices.   
                                                                                                                    The returned \textbf{ensemble} has the specified trees cached.}} \\
                                                  & \textbf{start}  &             &               &  \\
                                                  & \textbf{end}   &             &               & \\

   \midrule
   \multirow{3}{*}{\texttt{cacheRows}} & \textbf{rows[]} & \textbf{cachedRows[]}&     & \multirow{3}{*}{\parbox{0.47\linewidth}{Cache the rows in \textbf{rows[]} between the specified \textbf{start} and \textbf{end} indices.   
                                                                                                                    Returns an array of cached rows \textbf{cachedRows[]}.}} \\
                                                  & \textbf{start}  &             &               &  \\
                                                  & \textbf{end}   &             &               & \\

   \midrule
   \midrule
   \multirow{3}{*}{\texttt{loadThreshold}} & \textbf{buffer} & \textbf{threshold}&     & \multirow{3}{*}{\parbox{0.47\linewidth}{Load the threshold value for the node specified by \textbf{nodeIndex} 
                                                                                                                                      in the tree specified by \textbf{treeIndex} from \textbf{buffer}.
                                                                                                                                      Returns the loaded threshold.}} \\   
                                                  & \textbf{treeIndex}  &             &               &  \\
                                                  & \textbf{nodeIndex}   &             &               & \\

   \midrule
   \multirow{3}{*}{\texttt{loadFeatureIndex}} & \textbf{buffer} & \textbf{threshold}&     & \multirow{3}{*}{\parbox{0.47\linewidth}{Load the feature index for the node specified by \textbf{nodeIndex} 
                                                                                                                                      in the tree specified by \textbf{treeIndex} from \textbf{buffer}.
                                                                                                                                      Returns the loaded feature index.}} \\   
                                                  & \textbf{treeIndex}  &             &               &  \\
                                                  & \textbf{nodeIndex}   &             &               & \\

  \bottomrule
  \end{tabularx}
  }
  \vskip 5pt
  \caption{\label{Tab:IRSpecification} List of all the operations in the \Treebeard{} MLIR dialect. These operations are used 
  in conjunction with operations from other MLIR dialects like \op{scf}, \op{arith}, \op{gpu} etc. to represent and optimize 
  decision tree inference.}
\end{table*}

\section{Scheduling Language}
\label{sec:schedule}
As established in Section \ref{sec:motivation}, there are several 
different configurations and optimizations strategies for decision 
tree inference. The best ones vary significantly across
models, batch sizes and hardware platforms. Therefore, designing
any one hard-coded strategy is not feasible as this would 
make portable performance impossible. To address this, we 
design a scheduling language for \Treebeard{}. The scheduling 
language provides an abstract way to specify loop structure and 
other optimizations as an input to the compiler. The specified 
schedule controls the lowering of model inference to a set of 
loop nests. This configurability provided by the schedule allows us 
to build auto-schedulers and auto-tuners (Section \ref{sec:exploring}).
% Using a scheduling language will also significantly simplify 
% adding support for new hardware as this will likely require
% different locality optimizations and loop transformations.

% The goal of \Treebeard{}'s scheduling language is to declaratively
% express loop structures and the application of other optimizations 
% (tree walk unrolling, tree walk interleaving etc.). 

% \subsection{Language Definition}
The core construct of \Treebeard{}'s scheduling language is an 
\textbf{\emph{index variable}} which abstractly represents a loop. 
The language then provides directives to manipulate these index 
variables. There are two special index variables -- \op{batch} and
\op{tree} that are used to represent the batch and tree loops and all 
other index variables are derived from these. A schedule derives 
new index variables from these root index variables by applying
directives. 

\Treebeard{}'s scheduling language has three classes of directives. The first is a set 
of loop modifiers that are used to specify the structure of the loop nest to
walk the iteration space (Table \ref{Tab:LoopModifiers}). The second is a set of 
directives that enable optimizations on a loop (Table \ref{Tab:Optimizations}). 
Finally, we have a class of attributes that enable reduction specific optimizations
(Table \ref{Tab:ReductionOpts}).

The set of loops (index variables) are internally represented as nodes in a tree where the children
of a node represent immediately contained loops. Each schedule 
primitive modifies this tree in some way. The compiler tracks 
the lineage of each of the loops. This allows the compiler to 
automatically infer the ranges for all loops.

% \subsubsection{Loop Modifiers}
% The clauses modify these index variables or 
% index variables derived from these (through the application of clauses).
% \begin{itemize}
%   \item \textbf{tile}: Tile the passed index variable using a fixed tile size.
%   \item \textbf{split}: Split the range of the passed index variable into two parts. The range of the first part is specified
%   by an argument.
%   % \item \textbf{unroll}: Unroll an index completely
%   \item \textbf{reorder}: Reorder the specified indices. The specified indices must be successive indices in the current loop nest.
%   \item \textbf{specialize}: Generate separate code for each iteration of the specified index variable. This is useful 
%   while parallelizing across trees and these trees have different depths.
%   \item \textbf{gpuDimension}: Maps the specified index variable to represent a dimension in either the GPU kernel grid or thread block.
% \end{itemize}

\begin{table}[htb]
  \centering
  \resizebox{\linewidth}{!}{
  \begin{tabularx}{\linewidth}{c | l | l}
   \toprule
   \textbf{Directive} & \textbf{Inputs} & \textbf{Description} \\
   \midrule
   \multirow{4}{*}{\texttt{tile}} & \textbf{indexVar}  & \multirow{4}{*}{\parbox{0.55\linewidth}{Tile the loop corresponding to \textbf{indexVar}
   with the specified tile size. Resulting loops will be represented by \textbf{outer} and \textbf{inner}.}} \\
                                               &  \textbf{outer} & \\
                                               &  \textbf{inner} &  \\
                                               &  \textbf{tileSize} &  \\
   \midrule
   \multirow{6}{*}{\texttt{split}} & \textbf{indexVar}  & \multirow{6}{*}{\parbox{0.55\linewidth}{Fiss the loop represented by \textbf{indexVar}
   at iteration \textbf{splitIter}. Resulting loops will be represented by \textbf{first} and \textbf{second}. Returns a maps from nested 
   index variables to new ones created by splitting.}} \\
                                               &  \textbf{first} & \\
                                               &  \textbf{second} &  \\
                                               &  \textbf{splitIter} &  \\
                                               & & \\
                                               & & \\

   \midrule
   \multirow{4}{*}{\texttt{reorder}} & \textbf{indices[]}  & \multirow{4}{*}{\parbox{0.55\linewidth}{Permute loops corresponding to the specified index variables.
   The loops must be perfectly nested in the current loop structure.}} \\
                                               &   & \\
                                               &   &  \\
                                               &   &  \\
  %  \midrule
  %  \multirow{4}{*}{\texttt{specialize}} & \textbf{indexVar}  & \multirow{4}{*}{\parbox{0.55\linewidth}{Generate specialized code for each iteration of the loop. 
  %  Useful when different iterations of a loop need to execute different code.}} \\
  %                                              &   & \\
  %                                              &   &  \\
  %                                              &   &  \\
   \midrule
   \multirow{3}{*}{\texttt{gpuDimension}} & \textbf{indexVar}  & \multirow{3}{*}{\parbox{0.55\linewidth}{Map the passed index variable to a dimension of either the grid
   or thread block.}} \\
                                               & \textbf{gpuDim}  & \\
                                               &   &  \\

  \bottomrule
  \end{tabularx}
  }
  \vskip 5pt
  \caption{\label{Tab:LoopModifiers} List of all the loop modifiers in \Treebeard{}'s scheduling language. We use \emph{index variable}
  and \emph{loop} interchangeably in descriptions for clarity of exposition.}
\end{table}


% The following are examples of how the loop modifiers can be used.
% \begin{itemize}
%   \item The loop order used by XGBoost\cite{XGBoost} is (tree, batch) -- walk one tree for all inputs in the batch before 
%   moving to the next tree. The corresponding schedule would be
% \begin{lstlisting}[style=c++]
%   reorder(tree, batch)
% \end{lstlisting}

%   \item The below schedule computes 2 trees at a time over the whole batch.
% \begin{lstlisting}[style=c++]
%   tile(tree, t0, t1, 2)
%   reorder(t0, batch, t1)
% \end{lstlisting}

%   \item If we additionally only want to compute over 4 input rows (rather than the whole batch) for
%   every 2 tree, and then move onto the next 2 trees for the same set of inputs, then the schedule is as follows. 
% \begin{lstlisting}[style=c++]
%   tile(batch, b0, b1, 4) 
%   tile(tree, t0, t1, 2)
%   reorder(b0, t0, b1, t1) 
% \end{lstlisting}

% \end{itemize}

% \subsubsection{Optimizations}
% The following clauses provide ways to optimize the inference routine being generated.
% \begin{itemize}
%   \item \textbf{cache}: Cache the working set of one iteration of the specified loop corresponding
%   to this index. This can be specified on either batch or tree loops. Specifying it 
%   on a batch loop leads to all rows accessed in a single iteration of the loop 
%   being cached. Similarly, specifying it on a tree loop leads to all trees accessed in 
%   one iteration of that loop being cached.
%   \item \textbf{parallel}: Execute the loop corresponding to this index in parallel.
%   \item \textbf{interleave}: Interleave the execution of the tree walks within the current index (must be applied on an inner most index).
%   \item \textbf{unrollWalk}: Unroll tree walks at the current index. 
%   \item \textbf{peelWalk}: Peel the first n steps of the specified tree walk and don't check for leaves for that number of steps.
% \end{itemize}

\begin{table}[htb]
  \centering
  \resizebox{\linewidth}{!}{
  \begin{tabularx}{\linewidth}{c | l | l}
   \toprule
   \textbf{Directive} & \textbf{Inputs} & \textbf{Description} \\
   \midrule
   \multirow{4}{*}{\texttt{cache}} & \textbf{indexVar}  & \multirow{4}{*}{\parbox{0.55\linewidth}{Cache the working set of one iteration of the specified loop. 
   Cache rows for a batch loop and trees for a tree loop.}} \\
                                               &  &  \\
                                               &  &  \\
                                               &  &  \\
   \midrule
   \multirow{2}{*}{\texttt{parallel}} & \textbf{indexVar}  & \multirow{2}{*}{\parbox{0.55\linewidth}{Execute the iterations of the specified 
   loop in parallel.}} \\
                                               &  & \\

   \midrule                                               
   \multirow{3}{*}{\texttt{interleave}} & \textbf{indexVar}  & \multirow{3}{*}{\parbox{0.55\linewidth}{Interleave tree walks within the specified 
   loop (must be innermost loop).}} \\
                                               &  & \\
                                               &  & \\

   \midrule                                               
   \multirow{3}{*}{\texttt{unrollWalk}} & \textbf{indexVar}  & \multirow{3}{*}{\parbox{0.55\linewidth}{Unroll tree walks at the specified loop 
   for \textbf{unrollDepth} hops. Loop must be an innermost loop.}} \\
                                               & \textbf{unrollDepth} & \\
                                               &  & \\

  \bottomrule
  \end{tabularx}
  }
  \vskip 5pt
  \caption{\label{Tab:Optimizations} List of optimization directives in \Treebeard{}'s scheduling language. 
  We use \emph{index variable} and \emph{loop} interchangeably in descriptions for clarity of exposition.}
\end{table}


% \subsubsection{Reduction Optimization}
% \begin{itemize}
%   \item \textbf{atomicReduce}: Use atomic memory operations to accumulate values across 
%   parallel iterations of the specified loop. 
%   \item \textbf{sharedReduce}: Only applies to GPU compilation. Specifies that intermediate
%   results are to be stored in shared memory.
%   \item \textbf{vectorReduce}: Use vector instructions with the specified vector width 
%   to reduce intermediate values across parallel iterations of the specified loop.
% \end{itemize}

\begin{table}[htb]
  \centering
  \resizebox{\linewidth}{!}{
  \begin{tabularx}{\linewidth}{c | l | l}
   \toprule
   \textbf{Directive} & \textbf{Inputs} & \textbf{Description} \\
   \midrule
   \multirow{3}{*}{\texttt{atomicReduce}} & \textbf{indexVar}  & \multirow{3}{*}{\parbox{0.55\linewidth}{Use atomic memory operations to accumulate values across 
   parallel iterations of the specified loop.}} \\
                                               &  &  \\
                                               &  &  \\
   \midrule
   \multirow{3}{*}{\texttt{sharedReduce}} & \textbf{indexVar}  & \multirow{3}{*}{\parbox{0.55\linewidth}{Specifies that intermediate
   results are to be stored in shared memory (GPU only).}} \\
                                               &  & \\
                                               &  & \\

   \midrule                                               
   \multirow{4}{*}{\texttt{vectorReduce}} & \textbf{indexVar}  & \multirow{4}{*}{\parbox{0.55\linewidth}{Use vector instructions with the specified vector width 
   to reduce intermediate values across parallel iterations of the specified loop.}} \\
                                               & \textbf{width} & \\
                                               &  & \\
                                               &  & \\

  \bottomrule
  \end{tabularx}
  }
  \vskip 5pt
  \caption{\label{Tab:ReductionOpts} List of reduction optimization directives in \Treebeard{}'s scheduling language. 
  We use \emph{index variable} and \emph{loop} interchangeably in descriptions for clarity of exposition.}
\end{table}


%\TODO{Add examples for RAPIDS, Tahoe (Maybe show some strategies can be encoded?)}

\Treebeard{}'s scheduling language is expressive enough to represent a wide range of
strategies used in existing systems. We show examples of how it can be used to 
represent XGBoost and Tahoe's strategies. Before presenting the examples, we note that \Treebeard{}'s 
default loop order is [\op{batch}, \op{tree}], i.e, for each row in the input
batch, go over all trees.

% \subsection{The XBoost Schedule}
XGBoost\cite{XGBoost} implements inference on the CPU by going 
over a fixed number of rows (64 in the previous version)
for every tree and then moving to the next tree. When all trees 
have been walked for this set of rows, the next set of rows is 
taken up. Different sets of rows are processed in parallel.
The following schedule expresses XGBoost's strategy.
\begin{lstlisting}[style=c++]
  tile(batch, b0, b1, CHUNK_SIZE)
  reorder(b0, tree, b1)
  parallel(b0)
\end{lstlisting}

Tahoe\cite{Tahoe} has four strategies for inference on the GPU that it picks from for a given model. 
We show how two of these strategies can be encoded using \Treebeard{}'s scheduling language.
%The rest can be encoded similarly. 
\begin{itemize}
  \item In the \emph{direct method}, a single GPU thread walks all trees
  for a given input row. The schedule for this strategy is as follows.
\begin{lstlisting}[style=c++]
  tile(batch, b0, b1, ROWS_PER_TB)
  reorder(b0, b1, tree)
  gpuDimension(b0, grid.x)
  gpuDimension(b1, block.x)
\end{lstlisting}
  Here, \op{ROWS\_PER\_TB} is the number of rows that are processed by a single thread block.
  \item In the \emph{shared data} strategy, a thread block walks all the trees 
  for a given row in parallel. 
  % If threads walk multiple trees, each thread accumulates partial results. 
  Then, a thread block wide reduction is performed to compute 
  the prediction. The schedule for this strategy is as follows.
\begin{lstlisting}[style=c++]
  reorder(batch, tree)
  gpuDimension(batch, grid.x)
  gpuDimension(tree, block.x)
  cache(batch)
\end{lstlisting}
%   \item \textbf{Shared Forest}: In this strategy, the whole model is loaded into 
%   shared memory and subsequently, a single thread walks all trees for a particular
%   row. The schedule for this strategy is as follows.
% \begin{lstlisting}[style=c++]
%   tile(batch, b0, b1, ROWS_PER_TB)
%   tile(tree, t0, t1, N_TREES)
%   reorder(b0, b1, t0, t1)
%   cache(t0)
%   gpuDimension(b0, grid.x)
%   gpuDimension(b1, block.x)
% \end{lstlisting}
%   Here, we create a placeholder single iteration loop \op{t0} so that we can 
%   specify that all trees are to be cached.
%   \item \textbf{Shared Partial Forest}: In case the model is too large to fit into
%   shared memory, the model is split into chunks and each chunk is loaded into shared
%   memory. Again, as in the previous strategy, one thread walks all trees assigned to
%   a thread block for a row. The schedule for this strategy is as follows.
% \begin{lstlisting}[style=c++]
%   tile(batch, b0, b1, ROWS_PER_TB)
  
%   tile(tree, t0, t0Inner, TREES_PER_TB)
%   tile(t0Inner, t1, t2, TREES_PER_TB)
%   cache(t1)
%   reorder(b0, t0, b1, t1, t2)

%   gpuDimension(b0, grid.x)
%   gpuDimension(t0, grid.y)
%   gpuDimension(b1, block.x)
% \end{lstlisting}
\end{itemize}

% There are some simplifying assumptions and limitations in the current design
% of the scheduling language. Mainly, tree traversals are considered atomic and
% accumulation of tree predictions is done immediately (as opposed to, 
% for example, collecting all predictions and performing a reduction later).
% However, we find that these are not significant limitations in practice as
% the current design is able to express most strategies of interest.
\section{Overview of \Treebeard{} Compiler}

% \begin{itemize}
%   \item The compiler performs a number of optimizations on the HIR and MIR to improve the performance of the generated code.
%   \item Optimizations in HIR mainly transform the model itself. Tiling, tree reordering and padding are some of the 
%   optimizations that are performed on the HIR. \TODO{Should we use the MICRO paper overview for this part?}
%   \item Explain lowering to MIR using the schedule here.
%   \item Optimizations on the MIR include tree walk unrolling, tree walk interleaving, loop tiling, parallelization etc.
% \end{itemize}

\Treebeard{} takes a serialized decision tree ensemble as input (
XGBoost JSON, ONNX etc.) and automatically generates an optimized inference function
that can either target CPUs or GPUs. 
Figure \ref{Fig:CompilerStructure} shows the structure of the \Treebeard{} compiler. 
The inference computation is lowered through three intermediate representations
-- high-level IR (HIR), mid-level IR (MIR) and low-level IR (LIR). The LIR is
finally lowered to LLVM and then JIT'ed to the specified target processor. 
\Treebeard{} is built using the open-source \TreebeardOLD{} infrastructure~\cite{Treebeard}.
% Since the \TreebeardOLD{} infrastructure was originally designed to target CPUs,
% significant extensions were required to support schedule-guided compilation for CPUs and GPUs. 
The \TreebeardOLD{} infrastructure targets CPUs.
It lacks a scheduling language and does not support generating code for 
different implementation strategies. We enhance \TreebeardOLD{}  
to support schedule-guided compilation for CPUs and GPUs.
The parts \Treebeard{} that are new or significantly different compared to \TreebeardOLD{}
are shown as shaded boxes in Figure \ref{Fig:CompilerStructure}.
% However, it implements several optimizations on the HIR and MIR that 
% can be leveraged across target processors and we find that some these 
% are beneficial for GPUs as well. 
% This reuse of optimizations is possible 
% because the intermediate representations on which these optimizations are performed 
% are abstract and are designed to be target-independent. We briefly review these 
% optimizations below. 

\begin{figure}[htb]
  \centering
  \includegraphics[width=\linewidth]{figures/compiler.png}
  \caption{\Treebeard{} compiler structure.}
  \label{Fig:CompilerStructure}
\end{figure}

Table \ref{Tab:IRSpecification} lists the operations in the three IRs. 
In HIR, the model is represented as a collection of binary trees. This abstraction
allows the implementation of optimizations that require the manipulation of the model
or its constituent trees. We extend the \TreebeardOLD{} infrastructure with loop rewrites on 
the HIR that are implemented through the scheduling language (Section \ref{sec:schedule}).
The \emph{schedule} is implemented as an MLIR attribute on the \op{predictEnsemble} operation.
We use this object to implement the automatic scheduling described in Section \ref{sec:exploring}. 
We reuse HIR transformations to reorder and pad trees from \TreebeardOLD{}.
The tree transformations that reorder and pad trees are used in conjunction 
with loop transformations like splitting to specialize inference code as the example 
in Section \ref{sec:motivation} shows. 
% Also, we enable tree padding only if the schedule 
% specifies unrolling of tree walks.
% Figure \ref{Fig:HIRExample} shows this representation for a model 
% with four trees and how the model can be transformed. The model contains two trees of 
% depth 1 and two trees of depth 2 (\circled{A}). The right side of the diagram shows the models after 
% padding and reordering (\circled{B}). Padding makes all leaves in a tree the same depth (Trees 2 and
% 4 are padded). Trees are reordered so that trees of the same depth are together (Trees 
% 2 and 3 are swapped). We use this model as a running example for the rest of the section.
% After these model-level optimizations are performed on the HIR, the 
% code is lowered to the mid-level IR (MIR) as dictated by a  
% \emph{\textbf{schedule}}. The schedule determines how to traverse
% the iteration space that goes over the trees and input rows 
% by specifying how loops are to be tiled, 
% parallelized, mapped to GPU grid and block dimensions etc.
% (Details in Section \ref{sec:schedule}). While MIR is a loop-based IR that 
% explicitly encodes details of the iteration space has to be traversed, 
% it still abstracts details about the in-memory representation of the model. 

The HIR is lowered to MIR as dictated by the \emph{schedule}.
Optimizations like tree-walk unrolling and tree-walk interleaving
are performed on the MIR. 
% These optimizations are beneficial for GPUs as well.
% and the design of \TreebeardOLD{} allows us to reuse the tree-walk 
% unrolling and tree-walk interleaving optimizations on the MIR for GPUs.
% While building \Treebeard{}, we found that
One surprising thing we found while developing \Treebeard{} was that 
we could use ILP to improve performance on GPUs. 
One of the performance bottlenecks in inference code targeted 
to GPUs was that warps spent significant time being stalled. We were able to 
alleviate this bottleneck by interleaving tree walks. 
% This significantly improved performance of generated 
% code. The fact that exploiting ILP improved performance on GPUs was surprising. 

In the generated MIR, the compiler uses the \op{reduce} op from 
the \op{reduction} dialect we design (details in Section \ref{sec:reduction})
to represent reduction operations. The lowering of the \op{reduce} operation 
involves introducing temporary buffers and splitting the operation  
to correctly implement reduction in the presence of 
parallel loops. This process, that we call \textbf{\emph{legalization}}, is 
described in Section \ref{sec:reduction}. 

% \TODO{Should we talk about how interleaving is implemented as a 
% statemachine and therefore it can be used across representations 
% and tile traversal techniques?}

The MIR is further lowered to a low-level IR (LIR). 
Significant changes to \TreebeardOLD{} were required to get
LIR to correctly lower to GPU code. The most important of these was 
changing how the compiler implements support for in-memory 
representations of models (Section \ref{sec:representations}).
% With these design changes, we were able to reuse much of the CPU 
% implementation while customizing some parts for GPUs (for example,
% buffers need to be allocated differently for CPU and GPU, caching 
% is implemented differently etc.). 
Also, when the target processor is a GPU, the required memory transfers and kernel
invocations are inserted into the LIR. Buffers 
to hold model values are inserted and abstract tree operations are lowered to
explicitly refer to these buffers.
% This lowering is controlled by 
% a plugin mechanism which enables different in-memory representations to 
% be added to the compiler by implementing an interface 
% (Section \ref{sec:representations}). Vectorization of tree traversals
% is also explicitly represented in LIR.
Subsequently, the LIR is lowered to LLVM and then JIT'ed to the 
specified target processor.
% \subsection{Optimizations on High-Level IR}
% We augment the existing \TreebeardOLD{} infrastructure with loop rewrites on 
% the HIR that are implemented through the scheduling language (Section \ref{sec:schedule}).
% We use these to implement the automatic scheduling described in Section \ref{sec:exploring}. 
% Additionally, the \TreebeardOLD{} infrastructure implements HIR transformations to reorder and pad 
% trees. It also implements tree tiling transformations on the HIR \cite{Treebeard}. We reuse the 
% reordering and padding transformations on the HIR for GPUs. However, we found that 
% tiling trees was not beneficial for GPUs. This is because the tiling transformation
% introduces redundant computation inorder to vectorize computation on CPUs where 
% all lanes need to follow the same control flow. However, on SIMT GPUs, we find that 
% the benefits of tiling (coalescing memory accesses) do not outweigh the cost of
% redundant computation. We leave an investigation of this for future work.

% \subsection{Optimizations on Mid-Level IR}
% The original \TreebeardOLD{} infrastructure implements optimizations like 
% tree-walk unrolling, tree-walk interleaving, and parallelization
% on the MIR. These optimizations are beneficial for GPUs as well.
% and the design of \TreebeardOLD{} allows us to reuse the tree-walk 
% unrolling and tree-walk interleaving optimizations on the MIR for GPUs.

% While building \Treebeard{}, we found that one of the performance bottlenecks on the 
% GPU was that warps spent significant time being stalled. Since GPUs 
% implement scoreboarding~\cite{HennesseyPatterson}, we were able to alleviate this bottleneck by
% interleaving tree walks. This significantly improved performance of generated 
% code. We found it surprising that the use of ILP could benefit 
% performance on the GPU.  
% \TODO{Should we talk about how interleaving is implemented as a 
% statemachine and therefore it can be used across representations 
% and tile traversal techniques?}

% \subsection{A Note on Low-level IR}
% Significant changes to the original \TreebeardOLD{} design were required to get
% LIR to correctly lower to GPU code. The most important of these was 
% the change to how the compiler implements support for in-memory 
% representations of models (Section \ref{sec:representations}).
% With these design changes, we were able to reuse much of the CPU 
% implementation while customizing some parts for GPUs (for example,
% buffers need to be allocated differently for CPU and GPU, caching 
% is implemented differently etc.). 
\section{Reductions : Representation, Optimization and Lowering}
\label{sec:reduction}
\Treebeard{} needs to sum up individual tree predictions to compute 
the prediction of the model while performing inference. However,
generating fused reductions within arbitrary loop nests specified 
using \Treebeard{}'s scheduling language is non-trivial. We found 
that existing reduction support in MLIR is insufficient to code
generate and optimize these reductions. MLIR only supports reductions
of value types and does not provide ways to lower reductions to GPUs. 
To address this gap, we design an MLIR dialect that allows us to
specify accumulating values into an element of a multi-dimensional array
and can be lowered to CPU or GPU. 

The main abstraction we introduce is the \op{reduce} op. It models 
atomically accumulating values into an element of a 
multi-dimensional array (represented by an MLIR \op{memref}).
The following example shows how the \op{reduce} op can be used to 
sum up the elements of an array in parallel.

\begin{lstlisting}[style=c++]
  float arr[10], result[1]
  par.for i0 = 0  to 10 step 5 {
    for i1 = 0 to 5 
      reduce(result[0], arr[i0 + i1]) <"+", 0.0>
  }
\end{lstlisting}

The semantics of the \op{reduce} op is exactly the semantics of 
an atomic accumulation, i.e. it guarantees that all accumulations 
are correctly performed even in the presence of parallel loops. 
The \op{reduce} op is defined for all associative and commutative
reduction operations with a well-defined initial value. The 
reduction operator and the initial value are attributes applied
on the \op{reduce} op. 

Having modeled the reductions with an abstract operation, the 
aim now is to lower this to a correct and optimized 
implementation on both CPU and GPU. In order to do this, we 
first determine if any parallel loop iterations can accumulate 
into the same array element. We call such loops 
\emph{\textbf{reduction loops}}. If such loops exist, we 
\textbf{\emph{privatize}} the array for each iteration of the
loop. We call this process \textbf{\emph{legalization}}.
Subsequently, each privatized dimension 
can be reduced at the end of the reduction loop it was inserted for. 
\TODO{We cannot do better than this in terms of memory usage} \TODO{Need a proof}.


In our example above, parallel iterations of the \op{i0} loop accumulate 
into the same element of the \op{result} array.
We would therefore privatize the \op{result} array for each iteration of 
the \op{i0} loop as follows. 
\begin{lstlisting}[style=c++]
  float arr[10], result[1]
  float resultPriv[2][1]
  par.for i0 = 0  to 10 step 5 {
    for i1 = 0 to 5 
      reduce(resultPriv[i0/5][0], arr[i0 + i1]) <"+", 0.0>
  }
  result = reduce_dimension(resultPriv, 0)
\end{lstlisting}

The op \op{reduce\_dimension} reduces values across the specified
dimension of an n-dimensional array. In the above example, 
the \op{reduce\_dimension} op is reducing across all elements 
of the first dimension (dimension 0). Therefore, in this case, it 
produces a result memref with a single element (the first dimension
with size 2 is collapsed). 

% \begin{definition}
%   A parallel loop surrounding one or more \op{reduce} ops is 
%   a \textbf{conflicting loop} for a target multi-dimensional array if this 
%   loop has a non-zero dependence distance for the dependence between
%   any of the contained \op{reduce} ops.
% \end{definition} 

% \begin{definition}
%   \op{\textbf{reduce\_dimension(targetMemref, memref, dim, [indices], [rangeStart], [rangeEnd])}}:
%    Computes the reduction over the dimension specified by \op{dimension} and stores the 
%    result in \op{targetMemref}. \op{[indices]} must be a vector of \op{dim} elements
%    (or empty if the dimension being reduced is the first dimension). \op{[rangeStart]} 
%    and \op{[rangeEnd]} represent the range of indices following the reduction dimension and 
%    must have the same number of elements. If both are \op{null} (not passed), 
%    all elements of these dimensions are reduced. The computation performed by the op is as follows.

%   $targetMemref[\vec{\boldsymbol{indices}}, \vec{\boldsymbol{k}}] = \sum_{i=0}^{shape[dim]} memref[\vec{\boldsymbol{indices}}, i, \vec{\boldsymbol{k}}]\quad   \forall \vec{\boldsymbol{k}} \in \left[[rangeStart_0, rangeEnd_0), ... , [rangeStart_n, rangeEnd_n)\right]$
% \end{definition}

% Consider the following code with nested parallel loops. 
% (A situation where trees are split across both
% threads and thread blocks could result in such generated 
% code in \Treebeard{}.)
% \begin{lstlisting}[style=c++]
%   builtin.func @ReduceVector(%arr: memref<num_elemsxf64>, %result: memref<1xf64>) -> void {
%     par.for i0 = range(0 : num_elems/2 : num_elems) {
%       par.for i1 = range(0 : num_elems/4 : num_elems/2) {
%         for i2 = range(0 : num_elems/4) 
%           reduce(%result, 0, arr[i0 + i1 + i2]) <"+", 0.0>
%       }
%     }
%   }
% \end{lstlisting}

% Here, the \op{i0} and \op{i1} loops are conflicting loops wrt 
% the \op{result} memref. We \textbf{legalize} the reduction 
% by privatizing the \op{result} array wrt the \op{i0} and \op{i1} loops.
% However, there are now two privatized dimensions and therefore, 
% two dimensions need to be reduced to compute the final result.
% This multi-stage reduction is what enables us to model 
% hierarchical reductions.

% The following code shows how the reduction above is legalized. 
% We introduce a new op, \op{reduce\_dimension\_inplace} which 
% reduces a dimension of the input memref and stores results 
% in the same array. This helps saves memory by removing the 
% need to create multiple intermediate arrays to store results.
% Only the final dimension reduction uses the \op{reduce\_dimension} op.

% \begin{lstlisting}[style=c++]
%   builtin.func @ReduceVector(%arr: memref<num_elemsxf64>, %result: memref<1xf64>) -> void {
%     results_1 = memref<2x2x1xf64>
%     par.for i0 = range(0 : num_elems/2 : num_elems) {
%       par.for i1 = range(0 : num_elems/4 : num_elems/2) {
%         for i2 = range(0 : num_elems/4) 
%           index0 = i0/(num_elems/2)
%           index1 = i1/(num_elems/4)
%           reduce(%result_1[index0, index1, 0], arr[i0 + i1 + i2]) <"+", 0.0>
%       }
%       // result_1[i0/(num_elems/2), 0] = sum(result_1[i0/(num_elems/2), :])
%       reduce_dimension_inplace(%result_1, 1, i0/(num_elems/2)) 
%     }
%     // result = sum(result[:, 0])
%     %result = reduce_dimension(%result_1, 0)
%   }
% \end{lstlisting}

The behavior of the \op{reduce\_dimension\_inplace} op is similar to the 
\op{reduce\_dimension} op except that it updates the input array inplace
rather than writing results to a target array. The definition of the 
op is as follows. 

\begin{definition}
 \textbf{\op{reduce\_dimension\_inplace(memref, dim, [indices], [rangeStart], [rangeEnd])}}:
  Computes the reduction over the dimension specified by \op{dimension} and stores the 
  result at index 0 of that dimension. \op{[indices]} must be a vector of \op{dim} elements
   (or empty if the dimension being reduced is the first dimension). \op{[rangeStart]} 
   and \op{[rangeEnd]} must have the same number of elements. If both are \op{null} (not passed), 
   all elements of the corresponding dimension are reduced. 
  
  The computation performed by the op is defined by the following equation.
  
  $memref[\vec{\boldsymbol{indices}}, 0, \vec{\boldsymbol{k}}] = \sum_{i=0}^{shape[dim]} memref[\vec{\boldsymbol{indices}}, i, \vec{\boldsymbol{k}}]\quad   \forall \vec{\boldsymbol{k}} \in \left[[rangeStart_0, rangeEnd_0), ... , [rangeStart_n, rangeEnd_n)\right]$  
\end{definition}

\subsection{Lowering Reduction Operations}
We implement lowering of the operations defined above to both the CPU and GPU.
Since the compilation pipeline diverges after the reductions are legalized, 
we can implement lowering and optimization of our reduction dialect to CPUs and
GPUs simply using different MLIR rewrite patterns. We now briefly describe 
how these operations are lowered to the CPU and GPU. 

In the context of \Treebeard{}, the set of reduction loops is exactly the set 
of surrounding parallel loops that are iterating over trees. The 
results can be privatized for each loop iteratively 
and reductions along each privatized dimension can be inserted  
immediately following the loop the dimension was inserted due to.

\subsubsection{Lowering to CPU}
The lowering of the reduction operations to CPU is fairly straightforward. We lower the 
two operations listed above, \op{reduce\_dimension\_inplace} and 
\op{reduce\_dimension} to a simple loop nest that goes over the specified
subset of the input array, performs the reduction and writes 
the result into the appropriate location of the target array. 
If the schedule specifies that the reduction is to be vectorized,
then as many elements as specified by the vector width are read 
from the input array as a vector, accumulated as a vector, and 
finally written back to the target array. In general, this works 
well because reductions are typically being performed on dimensions
other than the inner-most dimension and therefore, this strategy
loads successive elements from memory maximizing memory bandwidth 
utilization. 

\TODO{explain atomic reduction}

\subsubsection{Lowering to GPU}
The lowering on GPU is slightly more involved than the lowering on CPUs.
However, we can lower the same abstractions to efficient implementations
and therefore simplify high-level code generation. The lowering for 
the inplace and non-inplace operations are essentially the same, except 
for the target array and we do not distinguish between them except 
for finally storing the result. 

The lowering of the \op{reduce\_dimension\_*} ops is distinct from existing 
work on implementing reductions efficiently on GPUs \cite{NVIDIAReductions}
because our abstractions potentially represent several independent reductions
(independent for different output elements).
Therefore, we can either exploit parallelism across the independent reductions or 
the inherent parallelism in the reduction by performing a divide and conquer 
reduction.

The reduction pass for GPU can follow one of two paths. If the lowering pass
determines that there are enough independent reductions to keep all threads
in a thread block busy, then it simply generates code that performs one (or 
multiple) reductions completely in a thread. If however there are not 
enough independent reductions, then the lowering pass generates a tree 
style reduction where multiple threads cooperate to perform a single reduction
using inter-thread shuffles.

Another feature specific to GPU reductions is the use of shared memory. 
If the schedule specifies that the reduction needs to be performed 
using shared memory, the privatized buffer is allocated in shared memory. 
Also, the compiler ensures that only as much shared memory is allocated 
as needed to hold values processed by a single thread-block and 
index offsets are appropriately rewritten to handle the differences between 
the indexing of the target memref and the shared memory array.
Our abstractions allow our lowering passes to be written completely 
independent of whether we use shared memory and therefore allow 
us to enable or disable shared memory use independently from the other 
parts of the compiler. 


\subsection{Use in \Treebeard{}}
We now present an example specific to \Treebeard{}. The schedule with
which code is generated is as below. \op{N\_t} is the number of trees 
and \op{batch\_size} is the batch size. The schedule tiles both the 
batch loop and the tree loop and parallelizes the outer batch 
and tree loops.

\begin{lstlisting}[style=c++]
IndexVar i0, i1, t0, t1;
auto& batch = schedule.GetBatchIndex();
auto& tree = schedule.GetTreeIndex();
schedule.Tile(batch, i0, i1, batch_size/2);
schedule.Tile(tree, t0, t1, N_t/2);
schedule.Reorder({i0, t0, t1, i1});
schedule.Parallel(t0);
schedule.Parallel(i0);
\end{lstlisting}

The loop-nest generated by \Treebeard{} for the above schedule is as follows. 
\begin{lstlisting}[style=c++]
  builtin.func @Prediction_Function(%arg0: memref<batch_sizexnum_featuresxf64>) -> memref<batch_sizexf64> {
    %result = memref.alloc <batch_sizexf64>
    %0 = #decisionforest<ReductionType = 0, #Trees = N_t, resultType = memref<batch_sizexf64>> 
    par.for i0 = range(0 : batch_size/2 : batch_size) {
      par.for t0 = range(0 : N_t/2 : N_t) {
        for t1 = range(0 : N_t/2) {
          for i1 = range(0 : batch_size/2) {
            %2 = GetTree(%0, t0 + t1) 
            %3 = WalkDecisionTree(%2, %arg0[i0+i1])
            reduce(%result, i0+i1, %3)
          }
        }
      }
    }
  }
\end{lstlisting}

\Treebeard{} determines that the \op{t0} loop is a conflicting loop for the \op{result}
array and therefore legalizes the reduction by inserting a privatized array 
\op{result\_1}. The privatized dimension of this array is reduced at the end 
of the \op{t0} loop.

\begin{lstlisting}[style=c++]
  builtin.func @Prediction_Function(%arg0: memref<batch_sizexnum_featuresxf64>) -> memref<batch_sizexf64> {
    %result = memref.alloc <batch_sizexf64>
    %result_1 = memref.alloc <2xbatch_sizexf64>
    %0 = #decisionforest<ReductionType = 0, #Trees = N_t, resultType = memref<batch_sizexf64>> 
    par.for i0 = range(0 : batch_size/2 : batch_size) {
      par.for t0 = range(0 : N_t/2 : N_t) {
        for t1 = range(0 : N_t/2) {
          for i1 = range(0 : batch_size/2) {
            %2 = GetTree(%0, t0 + t1) 
            %3 = WalkDecisionTree(%2, %arg0[i0+i1])
            reduce(%result, i0+i1, %3)
          }
        }
      }
      %result[i0 : i0+step] = reduce_dimension(%result_1, 0, i0 : i0+step)
    }
  }
\end{lstlisting}

While legalizing the reduction, the compiler determines that the 
\op{reduce\_dimension} operation must only process a subset of the final 
result that is computed within the current parallel iteration of the 
\op{i0} loop. Once this process is complete, the \op{reduce} ops in 
the result IR can be lowered to a simple ``read-accumulate-write''
sequence of instructions

Finally, we note that in our experiments, we found that our 
current implementation of lowering the reduction operations 
was sufficient and reduction is not the bottleneck in our 
generated code. However, we believe this approach to enabling 
higher level code generators to easily generate reductions 
through simple abstractions and then having the compiler 
automatically lower them to efficient implementation is an
important area for future work with applicability in several 
domains. 

\TODO{Should we mention how we handle multi-class models?}
\section{Model Representations}
\label{sec:representations}

The design of the \Treebeard{} compiler allows the implementation of different strategies for 
the in-memory representation of the model. The compiler currently has implementations for 
the three representations shown in Figure \ref{Fig:Representations}. The array and sparse 
representations are the ones described in the \TreebeardOLD{} paper\cite{Treebeard}. The reorg 
representation is the representation used by the RAPIDs library\cite{FIL}. 
The \textbf{array representation} is the simplest representation where the trees are stored in an array
in level order. The \textbf{sparse representation} stores the trees in a sparse format where 
memory is allocated only for nodes present in the tree and nodes contain pointers to their children. The \textbf{reorg representation}
interleaves the array representation of each tree in the model: all root nodes are stored first, then 
the left children of all the roots and so on. This representation was 
designed to improve memory coalescing when tree nodes are being loaded. 
\begin{figure}[htb]
  \centering
  \includegraphics[width=0.7\linewidth]{figures/representations.png}
  \caption{The three representations supported by \Treebeard{}.}
  \label{Fig:Representations}
\end{figure}

One of the major changes we make to the original design of \TreebeardOLD{} \cite{Treebeard} is to
separate the implementation of representations from the rest of the compiler. This allows
us to implement representations as plugins to the compiler. We define an interface 
that representations implement. The code generator is implemented using
this interface thus hiding details of the actual representation from the core 
compiler. Curcially, the interface abstracts how and what buffers 
are allocated, how to move from a node to its child, how trees 
are cached, reading the value of leaves and now threshold and 
feature indices are read from the allocated buffers.
% The interface abstracts the following details. 
% \begin{itemize}
%   \item \textbf{Buffer allocation:} The representation object exposes methods that generate 
%   buffer allocations in the IR. 
%   \item \textbf{Moving to child nodes:} Given the value of the predicate at a node (or tile),
%   generate code to move to the appropriate child node of the current node.
%   \item \textbf{Leaf representation:} The interface abstracts determining whether the current
%   node is a leaf (which is needed to terminate walks) and how to get the value of the leaf
%   (which is needed to get a tree's prediction).
%   \item \textbf{Caching trees:} Since reading the trees into shared memory on GPU (or prefetching)
%   them on CPU require knowledge of the buffer layout, the task of generating caching code for trees
%   is delegated to the representation object. 
%   \item \textbf{Loading thresholds and feature indices:} The representation object provides MLIR 
%   rewrite patterns to lower operations that load thresholds and feature indices to LLVM IR. This
%   is necessary because the details of how to load the thresholds and feature indices from the model 
%   buffers is representation-specific.
% \end{itemize}

In summary, the representation interface abstracts the details of how the model is stored in memory
and allows the compiler to generate code without having to explicitly know the details of the
representation. This design allows us to implement new representations without changing the core
compiler infrastructure. Implementing the representations as plugins also allows us to reuse
the implementations across different lowering pipelines. 

% For example, the code for the array 
% and sparse representations is almost fully shared between the CPU and GPU lowering pipelines.   

\section{Caching}
\label{sec:caching}

% \begin{itemize}
%   \item The compiler exposes caching of both trees and input rows in a unified manner.
%   \item This is independent of the final target on which the inference is to be run.
%   \item The caching is done at the granularity of a tree or a row.
%   \item Caching is encoded in the mid-level IR using the \op{cacheTrees} and \op{cacheRows} operations. 
%   These operations are generated when the HIR lowered to MIR and \op{Cache()} is specified on an 
%   index variable in the schedule. While the HIR is being lowered and a cached index variable is 
%   encountered, the compiler generates a \op{cacheTrees} or \op{cacheRows} operation depending on 
%   whether the index variable is a tree or a batch index variable.
%   \TODO{Should we talk about how the lowering from HIR to MIR is actually implemented as a tree walk?}
%   \item The lowering of these ops is done by the target-specific code generator.
%   \item For the \op{cacheRows} operation, \Treebeard{} uses pre-implemented lowerings for both
%   CPU and GPU. This is possible because the input is currently assumed to be a dense 
%   array format. 
%   \item For the \op{cacheTrees} operation, the lowering is representation-specific.
%   Each representation provides a lowering to the target-specific code generator
%   to lower the \op{cacheTrees} op when that representation is used.
%   \item The cache operations are lowered to reads into shared memory while compiling to GPU 
%   and to prefetches while compiling to CPU.
% \end{itemize}

\Treebeard{} provides mechanisms to cache both trees and input rows 
on both the CPU and GPU. As described in Section \ref{sec:schedule}, 
the user can specify that the working set of an iteration of a loop
needs to be cached using the \op{cache} directive. This 
provides a unified way to specify caching of both trees and input rows.
\Treebeard{} implements caching at the granularity of a tree or a row.

% \subsection{IR Representation of Caching}
Caching is encoded in the mid-level IR using the \op{cacheTrees} and \op{cacheRows} operations (Table \ref{Tab:IRSpecification}). 
These operations are generated when the HIR lowered to MIR and \op{cache} is specified on an 
index variable in the schedule. While the HIR is being lowered and a cached index variable is 
encountered, the compiler generates a \op{cacheTrees} or \op{cacheRows} operation depending on 
whether the index variable is a tree or a batch index variable.
\Treebeard{} also determines the working set of the loop  
and generates a caching operation with the appropriate limits.

% Each of the caching operations take parameters that specify the set of trees or rows that need 
% to be cached. The caching operations are defined as follows. 
% \begin{itemize}
%   \item \textbf{\op{cacheTrees(forest, start, end)}:} This operation caches the trees in ensemble
%   \op{forest} from \op{start} to \op{end}. The trees are cached in the order in which they are
%   specified in the ensemble. 
%   \item \textbf{\op{cacheRows(data, start, end)}:} This operation caches the rows in the input 
%   array \op{data} from \op{start} to \op{end}. The rows are cached in the same order as in the
%   input array.
% \end{itemize} 

%\subsection{Lowering of Caching Operations}
When the MIR is lowered to LIR, the cache ops are lowered to target-specific code. Each of 
the two caching operations is lowered differently for the CPU and the GPU. On CPU, the cahce operations are lowered to preteches while
on the GPU they are lowered to reads into shared memory.

Lowering the \op{cacheRows} operation is straightforward
% \Treebeard{} uses pre-implemented lowerings for both
% CPU and GPU. This is possible 
because the input is currently assumed to be a dense
array format. The lowering for the \op{cacheRows} operation is implemented directly in the \Treebeard{} compiler.

For the \op{cacheTrees} operation, the lowering is representation-specific. Each representation
provides a lowering to the target-specific code 
generator to lower the \op{cacheTrees} op when
that representation is used.
%  However, the \Treebeard{} infrastructure does provide helpers 
% to generate caching code to cache contiguous regions of memory. These helpers are reused 
% as required across different representations. 

\section{Exploring the Schedule Space}
\label{sec:exploring}

% \begin{itemize}
%   \item The scheduling language described in Section \ref{sec:schedule} allows
%   for numerous possible schedules for a given model. 
%   Finding a schedule with good performance is a non-trivial task.
%   \item We identify a reasonable template schedule for GPUs which encompasses
%   all strategies published in prior work.
%   \item Describe the template schedule.
%   \item Even within the variants of this template schedule, there is a significant 
%   amount of variation in performance. Show the histogram. 
%   \item Extensively exploring all possible parameter values for the template schedule 
%   is infeasible due to the large number of parameters.
%   Cite some numbers on how long it takes to explore the schedule space.
%   \item List the observations we make and the heuristic we design based on these.
% \end{itemize}

The set of schedules that can be constructed using the scheduling language described in 
Section \ref{sec:schedule} is unbounded. Searching this schedule space to find a 
high-performance schedule is a non-trivial task. To simplify this process, we design a
template schedule for GPUs that encompasses several strategies published in prior work.
Our template schedule assigns a configurable number of rows to each thread block and to each thread.
It distributes the trees across a specified number of threads and can cache trees and 
input rows if required while unrolling and interleaving of tree walks.
Table \ref{tab:schedparams} lists the parameter values we tried. 
% \begin{itemize}
%   \item \textbf{Number of rows per thread block (Integer):} The number of rows that are processed by each thread block.
%   \item \textbf{Number of rows per thread (Integer):} The number of rows processed by each thread.
%   \item \textbf{Number of tree threads (Integer):} The number of threads across which the trees are distributed.
%   \item \textbf{Cache rows (Boolean):} Whether the input rows are cached in shared memory.
%   \item \textbf{Cache trees (Boolean):} Whether the trees are cached in shared memory.
%   \item \textbf{Unroll walks (Boolean):} Whether the tree walks are unrolled.
%   \item \textbf{Tree walk interleave factor:} The number of tree walks that are interleaved.
%   \item \textbf{Shared memory reduction:} Whether the reduction across tree threads is done in shared memory.
% \end{itemize}

\begin{table}[htb]
  \centering
  \resizebox{\linewidth}{!}{
  \begin{tabularx}{\linewidth}{c | c }
  \toprule
  \textbf{Parameter} & \textbf{Values} \\
  \midrule
  \textbf{Rows per thread block} & $\{8, 32, 64\}$ \\
  \textbf{Rows per thread} & $\{1, 2, 4\}$ \\
  \textbf{Number of tree threads} & $\{2, 10, 20, 50\}$ \\
  \textbf{Cache rows} & $\{True, False\}$ \\
  \textbf{Cache trees} & $\{True, False\}$ \\
  \textbf{Unroll walks} & $\{True, False\}$ \\
  \textbf{Tree walk interleave factor} & $\{1, 2, 4\}$ \\
  \textbf{Shared memory reduction} & $\{True, False\}$ \\
  \bottomrule
  \end{tabularx}
  }
  \vskip 5pt
  \caption{\label{tab:schedparams} List of parameter values we explored for the template GPU schedule.}
\end{table}

%While the template schedule simplifies optimization of generated inference code,
It is important to note that the \Treebeard{} compiler itself does not place any 
restrictions on the schedule. The user is free to specify any schedule they wish.
The compiler pass that implements the template schedule is also implemented as a 
module outside the core \Treebeard{} compiler. 
% Users are also free to implement 
% other auto-schedulers that generate schedules different from the template schedule.

\begin{figure}[htb]
  \centering
  \includegraphics[width=\linewidth]{figures/normalized_kernel_histogram_lt5.png}
  \caption{Distribution of normalized execution times for all benchmark models
  with the template schedule using parameter values as shown in Table \ref{tab:schedparams}}. 
  \label{Fig:ExecTimeDistribution}
\end{figure}

While the template schedule simplifies code generation, finding a good set of 
parameter values is still hard.
Figure \ref{Fig:ExecTimeDistribution} shows the distribution of normalized execution times
for all benchmark models with different parameter values for the template schedule (inference
times normalized w.r.t fastest time for that model).
There is a significant amount of variation in performance even within the variants of the
template schedule,. Very few schedules perform close to the best while a vast majority of
schedules perform poorly.

Exploring the schedule space extensively even for a reasonable set of parameter values
is very expensive. We explored the set of parameter values listed in Table \ref{tab:schedparams}
for our benchmarks and found that it took anywhere between thirty minutes up to a few hours
to explore the entire space for each model. 
%Performing this extensive search for every model being compiled is infeasible in practise. 
We therefore
%need a better mechanism to guide the search for a good schedule.
design a heuristic to narrow down the set of schedules to explore based on the following 
observations.
\begin{itemize}
  \item For small batch sizes, the best schedules tend to have a small number of rows
  per thread block and partition the trees across a larger number of threads. This is 
  intuitive since the amount of data parallelism across the rows is limited for small batch sizes.
  \item Always cache rows in shared memory and never cache trees. We find that caching rows 
  %when possible (i.e., when the number of features is small enough to fit in shared memory)
  almost always improves performance. Caching trees on the other hand almost always degrades 
  performance. This is because the one time cost of loading trees into shared memory is 
  not sufficiently amortized when the whole of the tree is not accessed during inference. 
  \item Models with a large number of features tend to benefit from partitioning the 
  trees across more threads even at larger batch sizes. This is because processing fewer rows at a time 
  allows us to keep them in shared memory.
  % We empirically find that the threshold for when 
  % we should start partitioning the trees across more threads is when the number of features
  % is greater than 100. 
  \item We find that when a model prefers schedules with shared reduction, the same schedules 
  without shared reduction are among the best performing schedules without shared reduction.
  % We therefore are able to separate the evaluation of shared reduction by collecting the 
  % best schedules without shared reduction and only evaluating shared reduction on them. 
  Evaluating the top 3 schedules for shared reduction is sufficient in practice.  
\end{itemize}

\Treebeard{} uses these observations to narrow down the set of schedules to explore. 
The pseudo-code for the heuristic is shown in Algorithm \ref{alg:heuristic}.
The algorithm first computes a subset of thread block configurations in the function 
\op{TBConfigs}. A set of schedules based on these thread block configurations 
is then computed (\op{schedules}). The model is compiled with each of these schedules 
and then the resulting inference code is profiled. The three best performing schedules 
are collected and shared reduction is enabled on them and the resulting schedules evaluated.
The best schedule among all the evaluated schedules is selected as the schedule to use.
We find that this heuristic is able to find schedules that are close to the best schedules
but improves the search time by two orders of magnitude as we show in Section \ref{sec:results}.

\begin{algorithm}
  \caption{Heuristic to find a good schedule}
  \label{alg:heuristic}
  \begin{algorithmic}[1]
  \Procedure{TBConfigs}{$N_{batch}$, $N_f$}
    \State $T_{batch} \gets 2048,\; T_f \gets 128$
    \If {$N_{batch} \leq T_{batch}$ \textbf{or} $N_{f} > T_{f}$}
      \State $rowsPerBlock \gets \{8, 32\}$
      \State $treeThreads \gets \{20, 50\}$
    \Else
      \State $rowsPerBlock \gets \{32, 64\}$
      \State $treeThreads \gets \{2, 10\}$
    \EndIf
    \State \Return $rowsPerBlock, treeThreads$
  \EndProcedure
  \\
  \State $bestSchedules \gets shMemSchedules \gets \emptyset$
  \State $rowsPerTB, treeThds \gets TBConfigs(N_{batch}, N_f)$
  \State $cacheRows \gets \text{True},\; cacheTrees \gets \text{False}$
  \State $interleave \gets \{1, 2, 4\}$
  % \State $reps \gets \{array, sparse, reorg\}$
  \State $schedules \gets (rowsPerTB, treeThds, cacheRows,$ \\
                          \hspace{2cm}$cacheTrees, interleave)$
                          %\hspace{2cm}$$
  \For{$(sched, rep) \in schedules \times \{array, sparse, reorg\}$}
    \State $time \gets EvaluateSchedule(sched, rep)$
    \State $bestSchedules.insert(time, sched, rep)$
  \EndFor
  \State 
  \For{$sched, rep \in Top3(bestSchedules)$}
    \State $EnableSharedReduction(sched)$
    \State $time \gets EvaluateSchedule(sched, rep)$
    \State $shMemSchedules.insert(time, sched, rep)$
  \EndFor
  \State \Return $min(shMemSchedules \cup bestSchedules)$
  \end{algorithmic} 
\end{algorithm}
\section{Experimental Evaluation}
\label{sec:results}

\begin{figure*}[ht]
  \centering
  \begin{subfigure}[b]{.45\textwidth}
    \subcaptionbox*{}{\includegraphics[width=\textwidth]{figures/RandomModels/kernel_speedup_b512_depth8.png}}
    \caption{Batch size 512, depth 8}
  \end{subfigure}
  \begin{subfigure}[b]{.45\textwidth}
    \subcaptionbox*{}{\includegraphics[width=\textwidth]{figures/RandomModels/kernel_speedup_b4096_depth6.png}}
    \caption{Batch size 4096, depth 6}
  \end{subfigure}
  \hfill
  \caption{\Treebeard{} vs RAPIDs Kernel Time Speedup on NVIDIA RTX 4060 for several randomly generated models.}
\end{figure*}

\begin{figure}[htb]
  \centering
  \includegraphics[width=0.75\linewidth]{figures/geomean_speedup_4060_kernel_time.png}
  \caption{\Treebeard{} vs RAPIDs and Tahoe Kernel Time Speedup on NVIDIA RTX 4060}
  \label{Fig:TBvsRAPIDsTahoe_4060_Speedup}
\end{figure}

\begin{figure}[htb]
  \centering
  \includegraphics[width=0.75\linewidth]{figures/geomean_speedup_4060_total_time.png}
  \caption{\Treebeard{} vs RAPIDs Total Time Speedup on NVIDIA RTX 4060.}
  \label{Fig:TBvsRAPIDs_4060_TotalTimeSpeedup}
\end{figure}

\begin{figure*}[ht]
  \centering
  \begin{subfigure}[b]{.45\textwidth}
    \subcaptionbox*{}{\includegraphics[width=\textwidth]{figures/speedup_bar_graph_1024.png}}
    \caption{Batch size 1024}
  \end{subfigure}
  \begin{subfigure}[b]{.45\textwidth}
    \subcaptionbox*{}{\includegraphics[width=\textwidth]{figures/speedup_bar_graph_8192.png}}
    \caption{Batch size 8192}
  \end{subfigure}
  \hfill
  \caption{Kernel time speedup of \Treebeard{} vs RAPIDs on NVIDIA RTX 4060. Numbers on the bars are 
  inference times per sample in $\mu$s for RAPIDs and Tahoe.}
\end{figure*}

\begin{figure}[htb]
  \centering
  \includegraphics[width=0.75\linewidth]{figures/geomean_speedup_T400_kernel_time.png}
  \caption{\Treebeard{} vs RAPIDs and Tahoe Kernel Time Speedup on NVIDIA T400.}
  \label{Fig:TBvsRAPIDsTahoe_T400_Speedup}
\end{figure}

\begin{figure*}[ht]
  \centering
  \begin{subfigure}[b]{.45\textwidth}
    \subcaptionbox*{}{\includegraphics[width=\textwidth]{figures/abs_times_bar_graph_1024.png}}
    \caption{Batch size 1024.}
  \end{subfigure}
  \begin{subfigure}[b]{.45\textwidth}
    \subcaptionbox*{}{\includegraphics[width=\textwidth]{figures/abs_times_bar_graph_8192.png}}
    \caption{Batch size 8192.}
  \end{subfigure}
  \hfill
  \caption{\Treebeard{} vs RAPIDs total time comparison on NVIDIA RTX 4060. Numbers on the bars are the times 
  per sample in $\mu$s for \Treebeard{} and RAPIDs. Times for each benchmark are normalized w.r.t the RAPIDs time for that benchmark.}
\end{figure*}

\begin{figure}[htb]
  \centering
  \includegraphics[width=0.75\linewidth]{figures/AutotuningSpeedupvs4060Sched_T400.png}
  \caption{Autotuning heuristics speedup vs best 4060 schedule on T400.}
  \label{Fig:AutotuningSpeedupvs4060Sched_T400}
\end{figure}

\begin{figure}[htb]
  \centering
  \includegraphics[width=0.75\linewidth]{figures/HeuristicVsFullExplore_Speedup.png}
  \caption{Autotuning heuristic compile time speedup vs full schedule exploration.}
  \label{Fig:HeuristicVsFullExplore_Speedup}
\end{figure}

\begin{figure}[htb]
  \centering
  \includegraphics[width=0.75\linewidth]{figures/TBvsXGB_TotalTime.png}
  \caption{\Treebeard{} vs XGBoost Speedup on RTX 4060.}
  \label{Fig:TBvsXGBoost_Speedup}
\end{figure}

\begin{figure}[htb]
  \centering
  \includegraphics[width=0.75\linewidth]{figures/AMD_MI210_ATHeuristicVs4060Sched_speedup.png}
  \caption{Autotuning heuristics speedup vs best 4060 schedule on MI210.}
  \label{Fig:AMD_MI210_ATHeuristicVs4060Sched_speedup}
\end{figure}


\section{Related Work}
\label{Sec:Related}
While several optimization strategies for decision tree based models have been 
studied in the literature, to the best of our knowledge, no systems that are 
capable of exploring the full optimization space exist. We describe related work 
and compare these systems to \Treebeard{} in this section.

\emph{Decision Tree Inference Systems:} 
Tahoe\cite{Tahoe} is a system that implements high-performance library routines and a 
performance model for tree inference on GPUs. Tahoe is a library-based system that picks 
between four predefined strategies to implement decision tree inference on GPUs.
In comparison, \Treebeard{} explores a much larger set of implementation options 
because it is a compiler. \Treebeard{} can also explore different in-memory 
representations for models. Also, \Treebeard{} generates code that is specific to 
a particular model, specializing both the parallelism (by deciding the thread block 
structure on a per model basis) and the kernel code itself by 
performing optimizations like tree walk unrolling and interleaving.
% In contrast, Tahoe 
% uses a library-based approach, and cannot generate code tailored to a model 
% like \Treebeard{} does.

RAPIDS FIL\cite{FIL} is a library that implements decision tree inference on GPUs
and is the most widely used production system for decision tree inference. While 
FIL does implement some heuristics to pick a good configuration for every model, 
these techniques are limited and the library essentially uses a single strategy 
and in-memory representation for all models. XGBoost \cite{XGBoost} also implements GPU 
support\cite{XGBGPU} but uses a single strategy and in-memory representation. 
% In contrast, \Treebeard{} is a compiler that can explore a much larger
% optimization space and can generate code that is tailored to a specific model.

% Finally, all of Tahoe's optimizations are designed to address GPU specific problems. 
% For example, it uses an elaborate and opaque heuristic based on locality sensitive hashing 
% to coalesce accesses to the GPU global memory. We believe that the tree tiling
% infrastructure described in this paper will allow us to deterministically 
% coalesce accesses to GPU global memory.
%As described previously, Hummingbird\cite{Hummingbird} uses tensor operations to perform decision tree inference. 
% While compiling decision tree inference to GPUs presents a distinct set of challenges,
% we believe, we can achieve this while reusing much of \Treebeard{}'s current infrastructure.
% Specifically, we expect that most of the HIR and MIR
% optimizations described in this paper will carry over directly while LIR 
% optimizations and in-memory representations will need to be retargeted.
% This can be the subject of a separate future work.
On CPUs, XGBoost\cite{XGBoost}, LightGBM\cite{LightGBM} and
scikit-learn\cite{Sklearn} are extremely popular. However, 
as mentioned in Section \ref{sec:intro}, none of these systems
provide portable performance across different target machines.
\TODO{Write about the PACT paper and whether our scheduling language can represent 
all the schedules they propose.}
Other systems that hide dependency stalls by interleaving tree walks\cite{VPred},
implement optimized algorithms for tree inference\cite{QuickScorer, QuickScorer1}
and improve cache performance of decision tree ensembles on CPUs\cite{CacheConscious1, CacheConscious2}
have been proposed in prior work. However, these systems are limited to CPUs.
Some systems have been proposed to parallelize decision tree training 
on CPUs and GPUs\cite{Jansson2014gpuRFAG, Nasridinov2013DecisionTC}.


\emph{Decision Tree Ensemble Compilers:}
Several compilers for decision tree ensembles have been proposed in the 
literature \cite{Treelite, Treebeard, Hummingbird}. \TreebeardOLD{} and Treelite
exclusively target CPUs and all their optimizations are designed purely for 
performance on CPUs. 
Treelite\cite{Treelite} is a model compiler that only  
generates \op{if-else} code for each tree in the model. 
% Its code generator is hard-coded 
% to expand trees in an ensemble into a series of \op{if-else} statements (one for 
% each node in a tree). Due to this, extending Treelite and reusing it to 
% target GPUs is not possible.

\TreebeardOLD{} is the work most closely related to \Treebeard{}. While we 
build on top of \TreebeardOLD{}, \Treebeard{} is a significant enhancement 
over \TreebeardOLD{}. 
% Most importantly, we re-architect \TreebeardOLD{} in 
% order to generate code for multiple target processors.
Specifically, we introduce the scheduling language and schedule exploration 
while also enhancing the IRs and support for parallelizing 
across trees through the implementation of a novel MLIR reduction dialect.
% It cannot perform any of the optimizations \Treebeard{} is designed to perform. 

Hummingbird\cite{Hummingbird} is a compiler that compiles traditional ML models
to tensor operations, thereby enabling them to be run on tensor-based frameworks like
TensorFlow\cite{TensorFlow}. Hummingbird can target both CPUs
and GPUs, but, as was shown earlier~\cite{Treebeard},
tensor operations are not the most efficient way to implement decision tree inference
and the performance of Hummingbird is significantly lower than 
that of other frameworks.

% Hummingbird\cite{Hummingbird} compiles traditional ML models to make use of tensor primitives so that
% they can be integrated into tensor-based frameworks like TensorFlow \cite{TensorFlow}.
% While Hummingbird does compile decision tree based models to both CPUs and GPUs, the primary aim of 
% the system is to leverage progress in tensor compilers by representing ML models as tensor operations. 
% Consequently, Hummingbird does not perform any optimizations tailored to decision trees. It 
% also does not perform model specific optimizations that are enabled by \Treebeard{}'s abstractions. 

% Asadi et. al.\cite{VPred} optimize tree walks by hiding dependency stalls 
% by interleaving tree walks. In contrast, \Treebeard{} is an extensible 
% optimizing compiler that carefully implements this and many other
% optimizations at different levels of abstraction.
% QuickScorer\cite{QuickScorer, QuickScorer1} is an algorithm that uses 
% bit manipulation to compute tree predictions. Even though QuickScorer is 
% extremely fast for smaller models, it does not scale well to larger 
% models\cite{ProbBasedLayout}. The goal of QuickScorer is orthogonal 
% to the goals of \Treebeard{} and the QuickScorer algorithm can easily  
% be integrated into \Treebeard{} as another traversal strategy for the 
% system to explore. Tang et. al.\cite{CacheConscious1} and Jin et. al.\cite{CacheConscious2}
% build models to predict cache performance of decision tree ensembles on CPUs.
% This work is again orthogonal to the work described in this paper. Some systems have been proposed 
% to parallelize decision tree training on CPUs and GPUs\cite{Jansson2014gpuRFAG, Nasridinov2013DecisionTC}.

%Two recently published systems explore 
%using GPUs for decision tree inference\cite{Tahoe, Hummingbird}.

\emph{Other Systems and Techniques:} Ren et. al.~\cite{PortableVM} design an
intermediate language and a virtual machine to enable vector execution of decision tree
inference. However, this virtual machine is itself implemented by 
hand on different target processors. This is clearly 
more expensive than \Treebeard{}'s approach.
% Additionally, even though they perform layout optimizations, their system does
% not perform any model specific optimizations.
Jo et. al.\cite{MilindTreeVectorization} describe code transformations and runtime 
techniques that help vectorize tree-based applications. However, they do not 
study optimizations specific to decision trees.
% Both these works vectorize tree walks by performing different tree walks on each
% vector lane. The main issue with this approach is the divergence of tree walks.
% Another issue is that memory accesses are \op{gather}'s rather than vector
% loads. \Treebeard{}'s approach to vectorization solves both these issues. Also,
% these approaches are not precluded by \Treebeard{}'s tiling based vectorization.
% Multiple tiled tree walks can be combined into a single vectorized walk. We
% leave an exploration of this to future work.  FAST\cite{FAST} is a system that
% accelerates tree structured index search on CPUs and GPUs. FAST defines a layout
% for the index tree that enables vectorization of the tree walk. FAST uses a tree
% tiling approach to vectorize tree walks. However, FAST only uses a single
% triangular tile shape. \Treebeard{}'s basic tiling algorithm is a generalization
% of the tiling used in FAST. If given a perfectly balanced tree, the basic tiling
% algorithm would return exactly the tiling used by FAST.  Also, the tree walks in
% FAST are hand coded using intrinsics on the CPU and CUDA on the GPU and
% therefore need repeated effort to implement on each target. 
Inspector-executor systems \cite{TaoOfParallelism,HybridCPUGPU} have been 
developed to parallelize tree walks but are not a good fit 
for decision tree inference as the individual node predicates are 
simple and the overhead of an inspector-executor system would be prohibitive.

\emph{Code Generation Systems from Other Domains:}
Several optimizing compilers and code generation techniques have been developed 
for other domains. TVM\cite{TVM}, Tiramisu\cite{Tiramisu}, and Tensor 
Comprehensions\cite{TensorComprehensions} are optimizing compilers 
for DNNs that can target a variety of processors. Similarly, 
Halide\cite{Halide} is a DSL and compiler primarily designed for 
image processing applications. The concept of separating the computation 
from the schedule was pioneered by Halide and has since been adopted 
by several other systems~\cite{TVM,Tiramisu,GraphIt}. However, to 
the best of our knowledge, \Treebeard{} is the first system to design
a scheduling language for decision tree inference optimization
and to build a system capable of state-of-the-art performance 
across different processors.
% {\sc {Cortex}} transforms 
% recursive computations in DNNs to loop based computations and optimizes them.
% However, the model properties {\sc {Cortex}} assumes to optimize
% generated code do not hold in the context of decision tree inference. For example, 
% {\sc {Cortex}} assumes that control flow depends purely on data structure 
% connectivity. This is not the case with decision trees. Control flow is determined 
% by the input value and not by the structure of the tree. This makes the 
% problems addressed by \Treebeard{} very different from the ones handled
% by {\sc {Cortex}}.
Libraries that compose or generate optimized implementations  
for BLAS\cite{BLIS, atlas_sc98, CUTLASS} and signal processing\cite{FFTW, SPIRAL}
have also been developed.
% BLIS\cite{BLIS} and ATLAS\cite{atlas_sc98}
% are systems to instantiate high performance BLAS routines on multiple 
% target architectures. CUTLASS\cite{CUTLASS} provides building blocks 
% in the form of C++ templates to quickly instantiate high performance 
% BLAS functions on different GPU architectures. SPIRAL\cite{SPIRAL}
% is a domain specific compiler for signal processing applications
% that instantiates high performance routines for linear transformations 
% like FFTs and FIR filters. FFTW\cite{FFTW} is a fast fourier transform 
% compiler that generates high performance FFT routines by customizing 
% the routine based on FFT size and the target machine. 
% However, no such frameworks for decision tree ensembles exist currently.
% \Treebeard{} is a first step in this direction and unlocks several future 
% optimization opportunities.

\emph{Reductions:} CUB\cite{CUB} and Thrust\cite{Thrust} are libraries 
that implement high-performance parallel reductions on GPUs. While they 
provide highly-tuned implementations to perform large reductions, 
it is not possible to fuse these functions with other computations  
as required in \Treebeard{}. Reddy et. al.~\cite{ChandanReduction}
describe language constructs in {\sc {Pencil}}~\cite{Pencil}
to express reductions and to represent and optimize them using the polyhedral 
framework. It is not clear how these techniques can be fused with 
other computations in arbitrary loop nests as required in \Treebeard{}.
Additionally, their system does not express the hierarchical nature of 
reductions and also only targets GPUs. Suriana et. al.~\cite{HalideReductions}
extend Halide to add support for factoring reductions in the Halide 
scheduling language and to synthesize reduction operators. De Gonzalo 
et. al.~\cite{TangramReduction} describe a system based on Tangram 
that composes several partial reduction implementations into different 
reduction implementations for GPUs and then searches through these
alternate implementations to find the best ones. In summary, none of 
these systems provide abstractions and a general framework to generate 
and optimize reductions across different target processors as \Treebeard{} does.

\section{Citations and Bibliographies}

Artifacts:
  \cite{MLIR} and \cite{LLVM}.


%%
%% The next two lines define the bibliography style to be used, and
%% the bibliography file.
\bibliographystyle{ACM-Reference-Format}
\bibliography{refs}

%%
%% If your work has an appendix, this is the place to put it.

\end{document}
\endinput
%%
%% End of file `sample-sigplan.tex'.
