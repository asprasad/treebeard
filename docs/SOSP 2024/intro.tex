\section{Introduction}
\label{sec:intro}
We are in the midst of a hardware revolution, a new golden age for computer architecture~\cite{GoldenAge}. The last 
decade has seen a shift in architectural paradigms, with the rise of GPUs and accelerators. This shift has been driven
by the necessity to innovate in the post Moore's law and Dennard's scaling era. This transformation has also played 
a significant role in the success of modern deep learning models, as they enable scaling 
%% RG - minor rewording
%% model training and inference to a massive number of threads. 
training and inference to models with billions of parameters across a massive number of threads. 
Such scalability would be essential for all performance critical applications, including
other machine learning models that need to scale with increasing data sizes and model complexities. 
%need to accelerate machine learning workloads. The success of modern AI models is due to their 
%ability to scale both training and inference to a massive number of threads. This has been made
%possible by the rise of GPUs and accelerators.

%Despite these recent advancements, one dominant family of models, namely decision forest models have not seen the same level of acceleration.
Decision forest models remain the mainstay for machine learning over tabular data~\cite{DLNotAllYouNeed,TreebasedOutperformDL}. 
Their robustness, interpretability, and ability to handle missing data make them a popular choice for a wide 
range of applications\cite{DecisionTreesOverview, Med1, Med2, Facebook, LHCModel, Finance}. 
% \TODO{a few more lines, classification, regression, etc.}
% We observe that while scalable performance on neural network models has received significant attention, decision tree inference has not seen the same level of acceleration.
The Kaggle AI report for 2023 \cite{ABCD} highlights the continuing \emph{dominance of 
gradient boosted trees as the algorithm of choice} for tabular data. It also estimates that \emph{between
50\% and 90\% of practicing data scientists use tabular data as their primary type of data in their professional setting.}
% An analysis of ML workloads at a large scale web company found that these models are widely used~\cite{LookingGlass}.
This has also been observed in other surveys~\cite{KaggleSurvey,LookingGlass}.
Recent work has noted that the cost of inference is the most critical factor in the overall cost of 
deploying a machine learning model~\cite{Hummingbird, SageMaker}.
This is because, in production settings, each model is trained once and often used for inference millions of times. 
%% RG: Minor reword
%% However, 
Further, inference is run on a variety of hardware platforms, ranging from low to high-end CPUs and GPUs. 
This paper is motivated by the need to accelerate decision tree inference to achieve portable performance on 
%% RG: reword -- specifically say CPUs and GPUs.
%% a variety of hardware.
commodity platforms with CPUs and GPUs. 
% In particular, we focus on a range of commodity CPUs and GPUs, a class of hardware that has seen widespread 
% adoption across client/edge devices used for inference.

Decision forest models are composed of a large collection of decision trees (typically 100--1000), and inference involves 
traversing down each tree in the forest and aggregating the predictions. Inference is typically done in a batched
setting, where multiple inputs are processed simultaneously.
Despite the simplicity of the model and the availability of multiple sources of coarse-grain parallelism (parallelism 
across inputs in a batch and parallelism across trees), existing systems do not consistently perform well across different
models even on the limited set of targets they support. 
%This is because existing systems exploit a limited set of optimizations and often specialize the implementation to a specific hardware platform, and this limits their portability. 

%\TODO{cover Rapids, Tahoe, XGBoost and Treebeard}.  
Evaluation on a diverse set of models highlights that the best implementation often requires a careful composition 
of many optimization strategies like data layout optimizations, 
loop transformations, parallelization, and memory access optimizations. 
%For example, simultaneously traversing multiple trees with many threads, while managing the working set is 
%critical to get scalable performance. 
%This requires a combination of techniques like data layout optimizations, loop transformations, and memory access optimizations. 
%Existing systems each use specific data layouts for the underlying tree and use fixed loop structures. 
%Parallelism, loop structure and model representation together determine the accesses patterns and the working set. 
Existing systems today are mostly library based, only support a predefined combination of optimizations
and typically only target a single platform. 
On the CPU, XGBoost~\cite{XGBoost} uses a sparse representation for the model 
and a loop structure that processes one tree for a block of rows before moving to 
the next tree. On the GPU, RAPIDS FIL~\cite{FIL} uses a reorg representation~\cite{FIL}
and partitions trees across a fixed number of threads. 
%% RG: 
% \TODO{Should we say RAPIDS is only for nVidia GPUs?} 
%%
Tahoe~\cite{Tahoe} uses a variation of the reorg representation and 
has four predefined inference strategies from which it picks one based on an analytical model.
The GPU systems are CUDA based and only work on NVIDIA GPUs.
%These systems are all library based and are hard to extend. 
\TreebeardOLD{}, the state-of-the-art decision tree model compiler for CPUs
%% RG
, supports two fixed loop structures 
% and does not scale well with increasing number of threads. Additionally, it 
and lacks GPU specific optimizations that are critical to scale performance to massive number of threads.
% \TODO{Shouldn't we reverse this? First say no GPU support and then the scaling issue.}

This paper presents \Treebeard{}, a novel schedule-guided compiler
for decision tree inference on multiple hardware targets. \Treebeard{} is able to 
generate high-performance code for decision tree inference by exploring a large optimization space. 
%% RG - rewording
%% The code generation is guided by a \emph{schedule} written in a custom scheduling language that can represent a wide range of implementation strategies. 
This is achieved by a compilation framework consisting of a 
custom scheduling language that can represent a wide range of 
implementation strategies and techniques to efficiently explore
the optimization space. We demonstrate that the language is 
capable of expressing all optimizations proposed by prior work and more.
Further, our schedule exploration heuristic can quickly find a near 
optimal schedule for the model being compiled. 
%% Further we generalizes them and incorporates several new optimizations. 
% \TODO{RG: it would be good to expand on this and also talk about the retargetable component. 
% We could also say that the schedule framework and the retargetable compiler work in an 
% interwined manner, each benefitting from the other.}  
The second component of \Treebeard{} is a \emph{retargetable} multi-level compiler that can generate efficient code 
for any specified schedule for both CPUs and GPUs. For this purpose, we re-architect \TreebeardOLD{} to
support schedule-guided code generation, and incorporate several new optimizations that are critical to target GPUs. 
These two components of \Treebeard{} are intertwined, each benefitting from the other.
%Last, to address the challenge of selecting a near-optimal schedule from the vast optimization space, 
%%
%we design and implement a heuristic that is able to quickly find high-performance schedules for the model being compiled.

We evaluated \Treebeard{} on a large number of synthetic and real-world 
benchmark models and compared against state-of-the-art systems RAPIDS~\cite{RAPIDS}, Tahoe~\cite{Tahoe}
, XGBoost~\cite{XGBoost} and \TreebeardOLD{}~\cite{Treebeard}. 
% While these systems can only run on 
% specific hardware, \Treebeard{} outperforms all these systems while providing portable performance across 
% a range of target hardware.
The geomean speedup (across all benchmarks) of \Treebeard{} over
Tahoe and RAPIDS is in the range $2$--$5\times$ over several batch sizes. The geomean speedup over
XGBoost is more than $10\times$. While other systems can only run on specific hardware,
\Treebeard{} can provide portable performance across a range of target hardware
including AMD GPUs and CPUs. \Treebeard{} also enables better scalability on CPUs by parallelizing 
across multiple dimensions. We also demonstrate that \Treebeard{}'s scheduling heuristic is able to 
find near-optimal schedules in less than 30 seconds on average.

%% RG: This may not be needed -- can be removed to save space!
%% \TODO{Oganization}
In summary, this paper makes the following important contributions.
\begin{itemize}
  \item We identify that there are many different ways to implement  
  decision tree inference and that the best implementation depends both on the model
  and the target processor. We design a scheduling language that 
  allows us to represent this solution space. 
  \item We propose a retargetable compiler that can generate code as specified by 
  the schedule. The compiler also includes new abstractions for caching,
  reduction and representing models in memory. These are critical to generate 
  efficient GPU code.
  \item Since the scheduling language can represent an unbounded number of schedules, 
  we design and implement techniques to search over this space and find high-performance 
  schedules in tens of seconds. 
  \item We exhaustively evaluate \Treebeard{} and report that it achieves 
  portable performance across a range of hardware targets and consistently outperforms
  state-of-the-art decision tree inference frameworks for both CPU and GPU.
\end{itemize}
