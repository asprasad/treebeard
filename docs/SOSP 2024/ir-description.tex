% \section{\Treebeard{} IR Description}

% \begin{table*}[htb]
%   \centering
%   \resizebox{\linewidth}{!}{
%   \begin{tabularx}{\linewidth}{c l l l l}
%    \toprule
%    \textbf{Operation} & \textbf{Inputs} & \textbf{Outputs} & \textbf{Attributes} & \textbf{Description} \\
%    \midrule
%    \multirow{4}{*}{\texttt{predict\_ensemble}} & \textbf{data} & \textbf{result} & \textbf{ensemble} & Performs inference on the \textbf{data} using the model specified by\\
%                                                &               &                 & \textbf{predicate} & the \textbf{ensemble} attribute. The \textbf{schedule} attribute contains the \\
%                                                &               &                 & \textbf{schedule} & schedule described in Section \ref{sec:schedule}. \textbf{predicate} specifies the \\
%                                                 &               &                &                   & operator to use to evaluate nodes (Eg: $<$, $\leq$). \\
%    \midrule
%    \multirow{3}{*}{\texttt{walk\_decision\_tree}} & \textbf{trees[]} & \textbf{results[] }& \textbf{predicate}   & Represents an interleaved walk on all the element-wise pairs  \\ 
%                                                   & \textbf{rows[]}  &                    & \textbf{unrollDepth} & of \textbf{trees} and \textbf{rows}. \textbf{unrollDepth} specifies the number of \\
%                                                   &                  &                    &                      & hops to unroll. An array of tree walk results is returned.\\
%    \bottomrule
%   \end{tabularx}
%   }
%   \vskip 5pt
%   \caption{\label{Tab:Benchmarks}List of benchmark datasets and their parameters. 
%   The column \op{Leaf-biased} reports the number of leaf-biased trees per benchmark with $\langle\alpha =0.075, \beta =0.9 \rangle$. }
% \end{table*}

\begin{table*}[htb]
  \footnotesize 
  \centering
  \resizebox{\linewidth}{!}{
  \begin{tabularx}{\linewidth}{c | l | l | l | l}
   \toprule
   \textbf{Operation} & \textbf{Inputs} & \textbf{Outputs} & \textbf{Attributes} & \textbf{Description} \\
   \midrule
   \multirow{4}{*}{\texttt{predictEnsemble}} & \textbf{rows[]} & \textbf{result} & \textbf{ensemble} & \multirow{4}{*}{\parbox{0.47\linewidth}{Performs inference on the data in \textbf{rows[]} 
                                                                                                     using the model specified by
                                                                                                     the \textbf{ensemble} attribute. The \textbf{schedule} attribute contains the 
                                                                                                     schedule described in Section \ref{sec:schedule}. \textbf{predicate} specifies the
                                                                                                     operator to use to evaluate nodes (Eg: $<$, $\leq$).}} \\
                                               &               &                 & \textbf{predicate} & \\
                                               &               &                 & \textbf{schedule} &  \\
                                               &               &                 &                   &  \\
   \midrule
   \midrule

   \multirow{3}{*}{\texttt{walkDecisionTree}} & \textbf{trees[]} & \textbf{results[] }& \textbf{predicate}   & \multirow{3}{*}{\parbox{0.47\linewidth}{Represents an interleaved walk on all the element-wise pairs 
                                                                                                                    of \textbf{trees} and \textbf{rows}. \textbf{unrollDepth} specifies the number of
                                                                                                                    hops to unroll. An array of tree walk results is returned.}} \\
                                                  & \textbf{rows[]}  &                    & \textbf{unrollDepth} &  \\
                                                  &                  &                    &                      & \\
   \midrule
   \multirow{2}{*}{\texttt{ensemble}} &  & \textbf{ensemble} & \textbf{model} & \multirow{2}{*}{\parbox{0.47\linewidth}{Represents the forest of trees that constitute the model. The   
                                                                                                                       \textbf{model} attribute contains the actual trees model.}} \\
                                      &  &                 &   & \\
   \midrule
   \multirow{2}{*}{\texttt{getTree}} & \textbf{ensemble} & \textbf{tree} &  & \multirow{2}{*}{\parbox{0.47\linewidth}{Get the tree at the specified index (\textbf{treeIndex}) from the \textbf{ensemble}.}} \\   
                                                                                                                       
                                      & \textbf{treeIndex} &                 &   & \\                                                                                                                       

   \midrule
   \multirow{2}{*}{\texttt{getTreeClassId}} & \textbf{ensemble} & \textbf{classId} &  & \multirow{2}{*}{\parbox{0.47\linewidth}{Get the class ID for the tree at index \textbf{treeIndex} 
                                                                                                                               in the \textbf{ensemble}. This is used for multi-class models.}} \\    
                                            & \textbf{treeIndex} &                 &   & \\                                                                                                                       

   \midrule
   \multirow{1}{*}{\texttt{getRoot}} & \textbf{tree} & \textbf{rootNode} &  & \multirow{1}{*}{\parbox{0.47\linewidth}{Get the root node of the specified tree.}} \\   

   \midrule
   \multirow{2}{*}{\texttt{isLeaf}} & \textbf{tree} & \textbf{bool} &  & \multirow{2}{*}{\parbox{0.47\linewidth}{Returns a boolean value indicating whether \textbf{node} is a leaf of \textbf{tree}.}} \\   
                                    & \textbf{node} &                 &   & \\                                                                                                                       

   \midrule
   \multirow{2}{*}{\texttt{getLeafValue}} & \textbf{tree} & \textbf{value} &  & \multirow{2}{*}{\parbox{0.47\linewidth}{Returns the value of the leaf \textbf{node} in \textbf{tree}.}} \\   
                                    & \textbf{node} &                 &   & \\                                                                                                                       

   \midrule
   \multirow{3}{*}{\texttt{traverseTreeTile}} & \textbf{trees[]} & \textbf{nodes[] }& \textbf{predicate}   & \multirow{3}{*}{\parbox{0.47\linewidth}{Represents an interleaved traversal of the   
                                                                                                                    nodes in \textbf{nodes} based on the data in \textbf{rows}. \textbf{predicate} specifies 
                                                                                                                    the operator to use to evaluate nodes.}} \\
                                                  & \textbf{nodes[]}  &             &               &  \\
                                                  & \textbf{rows[]}   &             &               & \\

   \midrule
   \multirow{3}{*}{\texttt{cacheTrees}} & \textbf{ensemble} & \textbf{ensemble}&     & \multirow{3}{*}{\parbox{0.47\linewidth}{Cache the trees in the \textbf{ensemble} between the specified \textbf{start} and \textbf{end} indices.   
                                                                                                                    The returned \textbf{ensemble} has the specified trees cached.}} \\
                                                  & \textbf{start}  &             &               &  \\
                                                  & \textbf{end}   &             &               & \\

   \midrule
   \multirow{3}{*}{\texttt{cacheRows}} & \textbf{rows[]} & \textbf{cachedRows[]}&     & \multirow{3}{*}{\parbox{0.47\linewidth}{Cache the rows in \textbf{rows[]} between the specified \textbf{start} and \textbf{end} indices.   
                                                                                                                    Returns an array of cached rows \textbf{cachedRows[]}.}} \\
                                                  & \textbf{start}  &             &               &  \\
                                                  & \textbf{end}   &             &               & \\

   \midrule
   \midrule
   \multirow{3}{*}{\texttt{loadThreshold}} & \textbf{buffer} & \textbf{threshold}&     & \multirow{3}{*}{\parbox{0.47\linewidth}{Load the threshold value for the node specified by \textbf{nodeIndex} 
                                                                                                                                      in the tree specified by \textbf{treeIndex} from \textbf{buffer}.
                                                                                                                                      Returns the loaded threshold.}} \\   
                                                  & \textbf{treeIndex}  &             &               &  \\
                                                  & \textbf{nodeIndex}   &             &               & \\

   \midrule
   \multirow{3}{*}{\texttt{loadFeatureIndex}} & \textbf{buffer} & \textbf{threshold}&     & \multirow{3}{*}{\parbox{0.47\linewidth}{Load the feature index for the node specified by \textbf{nodeIndex} 
                                                                                                                                      in the tree specified by \textbf{treeIndex} from \textbf{buffer}.
                                                                                                                                      Returns the loaded feature index.}} \\   
                                                  & \textbf{treeIndex}  &             &               &  \\
                                                  & \textbf{nodeIndex}   &             &               & \\

  \bottomrule
  \end{tabularx}
  }
\vskip 2pt
  \caption{\label{Tab:IRSpecification} List of all the operations in the \Treebeard{} MLIR dialect. These operations are used 
  in conjunction with operations from other MLIR dialects like \op{scf}, \op{arith}, \op{gpu} etc. to represent and optimize 
  decision tree inference. Different IR levels (HIR, MIR and LIR) are separated by double lines.}
\end{table*}
